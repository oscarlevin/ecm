\documentclass{book}

\input{../activities-preamble.tex}
\begin{document}
\setcounter{project}{128}
\addtocounter{project}{-1}
\begin{activity}[]\label{compositionagian}
\hypertarget{p-914}{}%
A \terminology{composition} of the integer \(k\) into \(n\) parts is a list of \(n\) positive integers that add to \(k\).  How many compositions are there of an integer \(k\) into \(n\) parts?%
~\hfill{\tiny\hyperlink{a-128}{[hint]}\hypertarget{q-128}{}}\par\smallskip%
\noindent\textbf{Solution.}\hypertarget{solution-102}{}\quad%
\hypertarget{p-916}{}%
There is a bijection between compositions of \(k\) into \(n\) parts and arrangements of \(k\) identical books on \(n\) shelves so that each shelf gets a book. Namely, the number of books on shelf \(i\) is the \(i\)th element of the list. Thus the number of compositions of \(k\) into \(n\) parts is \(\binom{k-1}{n-1}\).%
\end{activity}
\end{document}
