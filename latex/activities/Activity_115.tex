\documentclass{book}

\input{../activities-preamble.tex}
\begin{document}
\setcounter{project}{115}
\addtocounter{project}{-1}
\begin{activity}[]\label{activity-108}
\hypertarget{p-829}{}%
How many anagrams of the word ``anagram'' are there? (An anagram is a rearrangement of \emph{all} of the letters of a word.)%
~\hfill{\tiny\hyperlink{a-115}{[hint]}\hypertarget{q-115}{}}\par\smallskip%
\noindent\textbf{Solution.}\hypertarget{solution-89}{}\quad%
\hypertarget{p-831}{}%
Using the quotient principle, you can treat each ``a'' as distinct to list all \(7!\) arrangements.  But then group arrangements if they have their a's in the same positions: there are \(3!\) permutations in each block.  So there are \(\frac{7!}{3!}\) anagrams.%
\par
\hypertarget{p-832}{}%
Another approach is to first pick three of the seven positions for the a's.  This can be done in \(\binom{7}{3}\) ways.  Then you must permute the remaining 4 letters in the remaining 4 spots, in \(4!\) ways.  This give \(\binom{7}{3}4!\) anagrams.%
\end{activity}
\end{document}
