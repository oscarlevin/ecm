\documentclass{book}

\input{../activities-preamble.tex}
\begin{document}
\setcounter{cpjt}{274}
\addtocounter{cpjt}{-1}
\begin{activity}\label{activity-267}
\hypertarget{p-1401}{}%
How many ways are there to partition \([5]\) into three sets?%
\begin{enumerate}[font=\bfseries,label=(\alph*),ref=\alph*]
\item\label{task-254} \hypertarget{p-1402}{}%
How many partitions of \([5]\) have two blocks of size 1 and one block of size 3?%
\item\label{task-255} \hypertarget{p-1403}{}%
How many partitions of \([5]\) have one block of size 1 and two of size 2?%
\par\smallskip%
\noindent\textbf{Hint}.\hypertarget{hint-177}{}\quad%
\hypertarget{p-1404}{}%
Careful here.  The answer is NOT \(\binom{4}{2}\binom{4}{2}\).  Do you see why?%
\item\label{task-256} \hypertarget{p-1405}{}%
Are there any other types of partitions we need to consider?  What is \(S(5,3)\)?%
\item\label{task-257} \hypertarget{p-1406}{}%
Generalize! Find a formula for \(S(k, k-2)\).%
\item\label{task-258} \hypertarget{p-1407}{}%
A friend tells you that \(S(k,k-2) = \binom{k}{3} + 3 \binom{k}{4}\).  Prove this is correct also.  If this is the formula you found for the previous part, see the hint.%
\par\smallskip%
\noindent\textbf{Hint}.\hypertarget{hint-178}{}\quad%
\hypertarget{p-1408}{}%
If you found this formula for \(S(k,k-2)\) in part (e), then pretend your friend claims that \(S(k, k-2) = \binom{k}{3} + \frac{1}{2}\binom{k}{2}\binom{k-2}{2}\) and prove why that is correct too.%
\end{enumerate}
\end{activity}

\clearpage\end{document}
