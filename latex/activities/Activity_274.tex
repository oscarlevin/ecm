\documentclass{book}

\input{../activities-preamble.tex}
\begin{document}
\setcounter{project}{274}
\addtocounter{project}{-1}
\begin{activity}[]\label{activity-267}
\hypertarget{p-1442}{}%
How many ways are there to partition \([5]\) into three sets?%
\begin{enumerate}[font=\bfseries,label=(\alph*),ref=\alph*]
\item\label{task-257} \hypertarget{p-1443}{}%
How many partitions of \([5]\) have two blocks of size 1 and one block of size 3?%
\item\label{task-258} \hypertarget{p-1444}{}%
How many partitions of \([5]\) have one block of size 1 and two of size 2?%
\par\smallskip%
\noindent\textbf{Hint.}\hypertarget{hint-177}{}\quad%
\hypertarget{p-1445}{}%
Careful here.  The answer is NOT \(\binom{4}{2}\binom{4}{2}\).  Do you see why?%
~\hfill{\tiny\hyperlink{a-274.b}{[hint]}\hypertarget{q-274.b}{}}\item\label{task-259} \hypertarget{p-1446}{}%
Are there any other types of partitions we need to consider?  What is \(S(5,3)\)?%
\item\label{task-260} \hypertarget{p-1447}{}%
Generalize! Find a formula for \(S(k, k-2)\).%
\item\label{task-261} \hypertarget{p-1448}{}%
A friend tells you that \(S(k,k-2) = \binom{k}{3} + 3 \binom{k}{4}\).  Prove this is correct also.  If this is the formula you found for the previous part, see the hint.%
\par\smallskip%
\noindent\textbf{Hint.}\hypertarget{hint-178}{}\quad%
\hypertarget{p-1449}{}%
If you found this formula for \(S(k,k-2)\) in part (e), then pretend your friend claims that \(S(k, k-2) = \binom{k}{3} + \frac{1}{2}\binom{k}{2}\binom{k-2}{2}\) and prove why that is correct too.%
~\hfill{\tiny\hyperlink{a-274.e}{[hint]}\hypertarget{q-274.e}{}}\end{enumerate}
\end{activity}
\end{document}
