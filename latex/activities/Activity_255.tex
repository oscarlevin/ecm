\documentclass{book}

\input{../activities-preamble.tex}
\begin{document}
\setcounter{cpjt}{255}
\addtocounter{cpjt}{-1}
\begin{activity}\label{candygenfn}
\hypertarget{p-1322}{}%
Write down the generating function for the number of ways to distribute identical pieces of candy to three children so that no child gets more than 4 pieces. Write this generating function as a quotient of polynomials. Using both the extended binomial theorem and the original binomial theorem, find out in how many ways we can pass out exactly ten pieces.%
\par\smallskip%
\noindent\textbf{Hint 1}.\hypertarget{hint-163}{}\quad%
\hypertarget{p-1323}{}%
Look for a power of a polynomial to get started.%
\par\smallskip%
\noindent\textbf{Hint 2}.\hypertarget{hint-164}{}\quad%
\hypertarget{p-1324}{}%
The polynomial referred to in the first hint is a quotient of two polynomials.  The power of the denominator can be written as a power series.%
\par\smallskip%
\noindent\end{activity}

\clearpage\end{document}
