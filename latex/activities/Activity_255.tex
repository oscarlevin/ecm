\documentclass{book}

\input{../activities-preamble.tex}
\begin{document}
\setcounter{project}{255}
\addtocounter{project}{-1}
\begin{activity}[]\label{candygenfn}
\hypertarget{p-1389}{}%
Write down the generating function for the number of ways to distribute identical pieces of candy to three children so that no child gets more than 4 pieces. Write this generating function as a quotient of polynomials. Using both the extended binomial theorem and the original binomial theorem, find out in how many ways we can pass out exactly ten pieces.%
~\hfill{\tiny\hyperlink{a-255}{[hint]}\hypertarget{q-255}{}}\par\smallskip%
\noindent\textbf{Solution.}\hypertarget{solution-191}{}\quad%
\hypertarget{p-1392}{}%
\((1+x+x^2+x^3+x^4)^3\). We can write%
\begin{align*}
\amp (1+x+x^2+x^3+x^4)^3 = \left(\frac{1-x^5}{1-x}\right)^3\\
=\amp (1-x^5)^3(1-x)^{-3}\\
=\amp (1-3x^5+3x^{10}-x^{15})\sum_{i=0}^\infty \binom{3+i-1}{i}x^i\\
=\amp (1-3x^5+3x^{10}-x^{15})\sum_{i=0}^\infty \binom{2+i}{i}x^i
\end{align*}
%
\par
\hypertarget{p-1393}{}%
The coefficient of \(x^{10}\) is the number of ways to pass out ten pieces of candy, and is \(\binom{12}{10}-3\binom{7}{5} +3\binom{2}{0}\).  Thus the number of ways to pass out ten pieces of candy is \(66-3\cdot21+3=6\).)%
\end{activity}
\end{document}
