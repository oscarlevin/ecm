\documentclass{book}

\input{../activities-preamble.tex}
\begin{document}
\setcounter{project}{89}
\addtocounter{project}{-1}
\begin{activity}[]\label{SubsetsFirstTime}
\hypertarget{p-660}{}%
How many subsets does a set \(S\) with \(n\) elements have?%
\par\smallskip%
\noindent\textbf{Hint.}\hypertarget{hint-42}{}\quad%
\hypertarget{p-661}{}%
Try applying the product principle in the case \(n = 2\) and \(n = 3\). How might you apply it in general?%
~\hfill{\tiny\hyperlink{a-89}{[hint]}\hypertarget{q-89}{}}\par\smallskip%
\noindent\textbf{Solution.}\hypertarget{solution-47}{}\quad%
\hypertarget{p-662}{}%
For each of the \(n\) elements of \(S\), we have two options: either we put the element into the subset or we do not. Thus, the general product principle tells us that there are \(2^n\) subsets of \(S\).%
\end{activity}

\clearpage\end{document}
