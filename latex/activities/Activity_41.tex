\documentclass{book}

\input{../activities-preamble.tex}
\begin{document}
\setcounter{project}{41}
\addtocounter{project}{-1}
\begin{activity}[]\label{activity-34}
\hypertarget{p-386}{}%
Radio stations broadcast their signal at certain frequencies. However, there are a limited number of frequencies to choose from, so nationwide many stations use the same frequency. This works because the stations are far enough apart that their signals will not interfere; no one radio could pick them up at the same time.%
\par
\hypertarget{p-387}{}%
Suppose 10 new radio stations are to be set up in a currently unpopulated (by radio stations) region. The radio stations that are close enough to each other to cause interference are recorded in the table below. What is the fewest number of frequencies the stations could use.%
\begin{sidebyside}{1}{0}{0}{0}
\begin{sbspanel}{1}
\resizebox{\linewidth}{!}{{
\begin{tabular}{c|c|c|c|c|c|c|c|c|c|c|}
 & {\tiny KQEA}&{\tiny KQEB}&{\tiny KQEC}&{\tiny KQED}&{\tiny KQEE}&{\tiny KQEF}&{\tiny KQEG}&{\tiny  KQEH}&{\tiny  KQEI}&{\tiny KQEJ } \\ \hline
{\tiny KQEA }&      &      &   x  &      &      &   x  &   x  &      &      &  x   \\ \hline
{\tiny KQEB }&      &      &   x  &   x  &      &      &      &      &      &      \\ \hline
{\tiny KQEC }&   x  &      &      &      &      &   x  &   x  &      &      &  x   \\ \hline
{\tiny KQED }&      &  x   &      &      &  x   &  x   &      &  x   &      &      \\ \hline
{\tiny KQEE }&      &      &      &  x   &      &      &      &      &  x   &      \\ \hline
{\tiny KQEF }&  x   &      &  x   &  x   &      &      &   x  &      &      &  x   \\ \hline
{\tiny KQEG }&  x   &      &  x   &      &      &  x   &      &      &      &  x   \\ \hline
{\tiny KQEH }&      &      &      &  x   &      &      &      &      &  x   &      \\ \hline
{\tiny KQEI }&      &      &      &      &  x   &      &      &  x   &      &  x   \\ \hline
{\tiny KQEJ }&  x   &      &  x   &      &      &  x   &   x  &      &  x   &      \\ \hline
\end{tabular}
}
}
\end{sbspanel}
\end{sidebyside}
\par\smallskip%
\noindent\textbf{Solution.}\hypertarget{solution-38}{}\quad%
\hypertarget{p-388}{}%
Represent the problem as a graph with vertices as the stations and edges when two stations are close enough to cause interference. We are looking for the chromatic number of the graph. Vertices that are colored identically represent stations that can have the same frequency.%
\par
\hypertarget{p-389}{}%
This graph has chromatic number 5. A proper 5-coloring is shown on the right. Notice that the graph contains a copy of the complete graph \(K_5\) so no fewer than 5 colors can be used.%
\begin{sidebyside}{2}{0.0875}{0.0875}{0.175}
\begin{sbspanel}{0.35}[bottom]
\resizebox{\linewidth}{!}{{
\begin{tikzpicture}[scale=.9]
	\coordinate (A) at (90:2);
	\coordinate (B) at (90-36:2);
	\coordinate (C) at (90-2*36:2);
	\coordinate (D) at (90-3*36:2);
	\coordinate (E) at (90-4*36:2);
	\coordinate (F) at (90-5*36:2);
	\coordinate (G) at (90-6*36:2);
	\coordinate (H) at (90-7*36:2);
	\coordinate (I) at (90-8*36:2);
	\coordinate (J) at (90-9*36:2);

	\draw (A) -- (F) -- (D) -- (H) -- (I) (G) -- (J) -- (C) -- (F) (C) -- (G) -- (A) -- (J);

	\draw (A) \va{\tiny KQEA} -- (C) \vr{\tiny KQEC} -- (B) \va{\tiny KQEB} -- (D) \vr{\tiny KQED} -- (E) \vb{\tiny KQEE} -- (I) \vl{\tiny KQEI} -- (J) \va{\tiny KQEJ} -- (F) \vb{\tiny KQEF} -- (G) \vl{\tiny KQEG} (H) \vl{\tiny KQEH};
	\end{tikzpicture}
}
}
\end{sbspanel}
\begin{sbspanel}{0.3}[bottom]
\resizebox{\linewidth}{!}{{
\begin{tikzpicture}[scale=.9]
	\coordinate (A) at (90:2);
	\coordinate (B) at (90-36:2);
	\coordinate (C) at (90-2*36:2);
	\coordinate (D) at (90-3*36:2);
	\coordinate (E) at (90-4*36:2);
	\coordinate (F) at (90-5*36:2);
	\coordinate (G) at (90-6*36:2);
	\coordinate (H) at (90-7*36:2);
	\coordinate (I) at (90-8*36:2);
	\coordinate (J) at (90-9*36:2);

	\draw (A) -- (F) -- (D) -- (H) -- (I) (G) -- (J) -- (C) -- (F) (C) -- (G) -- (A) -- (J);
	\draw[line width=1.25pt] (A) -- (C) -- (F) -- (G) -- (J) -- (A) -- (F) -- (J) -- (C) -- (G) -- (A);

	\draw (A) \va{ R} -- (C) \vr{ B} -- (B) \va{ G} -- (D) \vr{ B} -- (E) \vb{ G} -- (I) \vl{ B} -- (J) \va{ P} -- (F) \vb{ G} -- (G) \vl{ Y} (H) \vl{ R};
	\end{tikzpicture}
}
}
\end{sbspanel}
\end{sidebyside}
\end{activity}
\end{document}
