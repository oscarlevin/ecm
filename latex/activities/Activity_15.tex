\documentclass{book}

\input{../activities-preamble.tex}
\begin{document}
\setcounter{project}{15}
\addtocounter{project}{-1}
\begin{activity}[]\label{activity-10}
\hypertarget{p-180}{}%
Suppose we have already proved \hyperref[thm-eulercircuit]{Theorem~\ref{thm-eulercircuit}}, that a graph has an Euler circuit if and only if every vertex has even degree.  Suppose now we want to prove \hyperref[thm-eulerpath]{Theorem~\ref{thm-eulerpath}}, that a graph has an Euler path if and only if it has at most two odd degree vertices.%
\begin{enumerate}[font=\bfseries,label=(\alph*),ref=\alph*]
\item\label{task-13} \hypertarget{p-181}{}%
Given a graph with an Euler path, how can you get a graph that definitely has an Euler circuit?  How did this affect the number of odd degree vertices?%
~\hfill{\tiny\hyperlink{a-15.a}{[hint]}\hypertarget{q-15.a}{}}\item\label{task-14} \hypertarget{p-186}{}%
Given a graph with at most two odd degree vertices, build a graph that has only even degree vertices.  If you have an Euler circuit of your new graph, what would that Euler circuit become if you looked at your original graph?%
~\hfill{\tiny\hyperlink{a-15.b}{[hint]}\hypertarget{q-15.b}{}}\item\label{task-15} \hypertarget{p-190}{}%
What do each of the above tasks demonstrate?  That is, write a proof of \hyperref[thm-eulerpath]{Theorem~\ref{thm-eulerpath}} assuming \hyperref[thm-eulercircuit]{Theorem~\ref{thm-eulercircuit}}.  Make sure you specify how each ``direction'' (implication and converse) of the proof is established by the tasks above.%
\end{enumerate}
\end{activity}
\end{document}
