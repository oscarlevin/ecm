\documentclass{book}

\input{../activities-preamble.tex}
\begin{document}
\setcounter{cpjt}{320}
\addtocounter{cpjt}{-1}
\begin{activity}\label{activity-313}
\hypertarget{p-1591}{}%
In \hyperref[genfunpartitions]{Activity~\ref{genfunpartitions}}, we multiplied together infinitely many power series. Here are two notations for infinite products that look rather similar:%
\begin{equation*}
\prod_{i=1}^\infty 1 + x + x^2 +\cdots+ x^i\qquad\mbox{and}\qquad
\prod_{i=1}^\infty 1 +x^i +x^{2i} +\cdots + x^{i^2}.
\end{equation*}
%
\par
\hypertarget{p-1592}{}%
However, one makes sense and one doesn't. Figure out which one makes sense and explain why it makes sense and the other one doesn't. If we want a product of the form%
\begin{equation*}
\prod_{i=1}^\infty 1 +p_i(x),
\end{equation*}
where each \(p_i(x)\) is a nonzero polynomial in \(x\) to make sense, describe a relatively simple assumption about the polynomials \(p_i(x)\) that will make the product make sense. If we assumed the terms \(p_i(x)\) were nonzero power series, is there a relatively simple assumption we could make about them in order to make the product make sense? (Describe such a condition or explain why you think there couldn't be one.)%
\par\smallskip%
\noindent\textbf{Hint}.\hypertarget{hint-211}{}\quad%
\hypertarget{p-1593}{}%
If infinitely many of the polynomials had a nonzero coefficient for \(q\), would the product make any sense?%
\par\smallskip%
\noindent\end{activity}

\clearpage\end{document}
