\documentclass{book}

\input{../activities-preamble.tex}
\begin{document}
\setcounter{project}{307}
\addtocounter{project}{-1}
\begin{activity}[]\label{rectanglecomplement}
\hypertarget{p-1586}{}%
Show that the number of partitions of \(k\) into four parts equals the number of partitions of \(3k\) into four parts of size at most \(k-1\) (or \(3k-4\) into four parts of size at most \(k-2\) or \(3k-4\) into four parts of size at most \(k\)).%
~\hfill{\tiny\hyperlink{a-307}{[hint]}\hypertarget{q-307}{}}\par\smallskip%
\noindent\textbf{Solution.}\hypertarget{solution-235}{}\quad%
\hypertarget{p-1588}{}%
Think about putting the Young diagram of the partition into the upper left corner of a rectangle that is \(k\) units wide and four units high. Subdivide the rectangle into \(4k\) squares of unit area. The Young diagram covers \(k\) of these squares. The uncovered squares are in rows of length \(r_1\le r_2\le r_3\le r_4\). Thus if we list these lengths in the opposite order, we have a decreasing list representation of a partition of \(3k\). Even \(r_1\) will have to be positive, because the first part of the original partition will be at most \(k-3\). To get partitions of \(3k+4\), use a rectangle of width \(k+1\), and to get partitions of \(3k-4\), use a rectangle of width \(k-1\). Since the first row of the Young diagram has at most \(k-3\) squares, we will still have four nonzero parts in the partition that results.%
\end{activity}
\end{document}
