\documentclass{book}

\input{../activities-preamble.tex}
\begin{document}
\setcounter{project}{318}
\addtocounter{project}{-1}
\begin{activity}[]\label{atmostmparts}
\hypertarget{p-1590}{}%
In \hyperref[largestpartatmostm]{Task~\ref{activity-310}.\ref{largestpartatmostm}} you gave the generating function for the number of partitions of an integer into parts of size at most \(m\). Explain why this is also the generating function for partitions of an integer into at most \(m\) parts. Notice that this is the generating function for the number of partitions whose Young diagram fits into the space between the line \(y=0\) and the line \(y=m\).%
\par\smallskip%
\noindent\textbf{Hint.}\hypertarget{hint-207}{}\quad%
\hypertarget{p-1591}{}%
Think about conjugate partitions.%
~\hfill{\tiny\hyperlink{a-318}{[hint]}\hypertarget{q-318}{}}\par\smallskip%
\noindent\textbf{Solution.}\hypertarget{solution-212}{}\quad%
\hypertarget{p-1592}{}%
Conjugation is a bijection between partitions with largest part at most \(m\) and partitions with at most \(m\) parts. Thus the coefficient of \(q^i\) (the number of partitions of \(i\) into parts of size at most \(m\)) in the generating function for the number of partitions of integers into parts of size at most \(m\) will be the coefficient of \(q^i\) (the number of partitions of \(i\) with at most \(m\) parts) in the generating function for the number of partitions of integers into parts of size at most \(m\). Thus the two generating functions are the same.%
\end{activity}

\clearpage\end{document}
