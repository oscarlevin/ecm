\documentclass{book}

\input{../activities-preamble.tex}
\begin{document}
\setcounter{cpjt}{318}
\addtocounter{cpjt}{-1}
\begin{activity}\label{atmostmparts}
\hypertarget{p-1574}{}%
In \hyperref[largestpartatmostm]{Task~\ref{activity-310}.\ref{largestpartatmostm}} you gave the generating function for the number of partitions of an integer into parts of size at most \(m\). Explain why this is also the generating function for partitions of an integer into at most \(m\) parts. Notice that this is the generating function for the number of partitions whose Young diagram fits into the space between the line \(y=0\) and the line \(y=m\).%
\par\smallskip%
\noindent\textbf{Hint}.\hypertarget{hint-207}{}\quad%
\hypertarget{p-1575}{}%
Think about conjugate partitions.%
\par\smallskip%
\noindent\end{activity}

\clearpage\end{document}
