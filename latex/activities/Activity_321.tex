\documentclass{book}

\input{../activities-preamble.tex}
\begin{document}
\setcounter{project}{321}
\addtocounter{project}{-1}
\begin{activity}[]\label{activity-314}
\hypertarget{p-1623}{}%
What is the generating function (using \(q\) for the variable) for the number of partitions of an integer in which each part is even?%
\par\smallskip%
\noindent\textbf{Hint.}\hypertarget{hint-212}{}\quad%
\hypertarget{p-1624}{}%
\((1 + q^2 + q^4 )(1 + q^3 + q^9 )\) is the generating function for partitions of an integer into at most two twos and at most two threes.%
~\hfill{\tiny\hyperlink{a-321}{[hint]}\hypertarget{q-321}{}}\par\smallskip%
\noindent\textbf{Solution.}\hypertarget{solution-229}{}\quad%
\hypertarget{p-1625}{}%
\((1+q^2+q^4+\cdots)(1+q^4+q^8+\cdots)(1+q^6+q^{12}+\cdots)\cdots\), which can be written more precisely as \(\displaystyle\prod_{i=1}^\infty
\sum_{j=0}^\infty q^{2ij}\).%
\end{activity}
\end{document}
