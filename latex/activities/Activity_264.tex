\documentclass{book}

\input{../activities-preamble.tex}
\begin{document}
\setcounter{cpjt}{264}
\addtocounter{cpjt}{-1}
\begin{activity}\label{partialfractionsintro}
\hypertarget{p-1356}{}%
In \hyperref[simplifysumoffractions]{Activity~\ref{simplifysumoffractions}} you see that when we added numerical multiples of the reciprocals of first degree polynomials we got a fraction in which the denominator is a quadratic polynomial. This will always happen unless the two denominators are multiples of each other, because their least common multiple will simply be their product, a quadratic polynomial. This leads us to ask whether a fraction whose denominator is a quadratic polynomial can always be expressed as a sum of fractions whose denominators are first degree polynomials. Find numbers \(c\) and \(d\) so that%
\begin{equation*}
\frac{5x+1}{(x-3)(x+5)} = \frac{c}{x-3} + \frac{d}{x+5}.
\end{equation*}
%
\par\smallskip%
\noindent\textbf{Hint}.\hypertarget{hint-169}{}\quad%
\hypertarget{p-1357}{}%
%
\begin{equation*}
\frac{5x+1}{(x-3)(x-5)} = \frac{cx+5c+dx-3d}{(x-3)(x-5)}
\end{equation*}
gives us%
\begin{align*}
5x \amp = cx+dx\\
1 \amp= 5c-3d\text{.}
\end{align*}
%
\par\smallskip%
\noindent\end{activity}

\clearpage\end{document}
