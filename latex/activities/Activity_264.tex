\documentclass{book}

\input{../activities-preamble.tex}
\begin{document}
\setcounter{project}{264}
\addtocounter{project}{-1}
\begin{activity}[]\label{partialfractionsintro}
\hypertarget{p-1397}{}%
In \hyperref[simplifysumoffractions]{Activity~\ref{simplifysumoffractions}} you see that when we added numerical multiples of the reciprocals of first degree polynomials we got a fraction in which the denominator is a quadratic polynomial. This will always happen unless the two denominators are multiples of each other, because their least common multiple will simply be their product, a quadratic polynomial. This leads us to ask whether a fraction whose denominator is a quadratic polynomial can always be expressed as a sum of fractions whose denominators are first degree polynomials. Find numbers \(c\) and \(d\) so that%
\begin{equation*}
\frac{5x+1}{(x-3)(x+5)} = \frac{c}{x-3} + \frac{d}{x+5}.
\end{equation*}
%
~\hfill{\tiny\hyperlink{a-264}{[hint]}\hypertarget{q-264}{}}\end{activity}
\end{document}
