\documentclass{book}

\input{../activities-preamble.tex}
\begin{document}
\setcounter{project}{23}
\addtocounter{project}{-1}
\begin{activity}[]\label{activity-18}
\hypertarget{p-256}{}%
Recall that a \terminology{tree} is a connected graph with no cycles.  We wish to prove that every tree with \(v = n\) vertices has \(e = n-1\) edges.  (This is actually a special case of Euler's formula for planar graphs, as a tree will always be a planar graph with 1 face).%
\begin{enumerate}[font=\bfseries,label=(\alph*),ref=\alph*]
\item\label{task-28} \hypertarget{p-257}{}%
First, prove a lemma (not using induction): every tree contains at least two vertices of degree 1.%
\par\smallskip%
\noindent\textbf{Hint.}\hypertarget{hint-4}{}\quad%
\hypertarget{p-258}{}%
Look at paths in the tree.  Let \(v_0, v_1, \ldots, v_n\) be a path of maximal length.  What can you say about \(v_0\) and \(v_n\)?%
~\hfill{\tiny\hyperlink{a-23.a}{[hint]}\hypertarget{q-23.a}{}}\item\label{task-29} \hypertarget{p-259}{}%
We will give a proof by induction on the number of vertices in a tree.  What should the base case be?  Prove it.%
\par\smallskip%
\noindent\textbf{Hint.}\hypertarget{hint-5}{}\quad%
\hypertarget{p-260}{}%
What is the smallest number of vertices that make sense?  You will need to prove that \emph{all} trees with one vertex have the correct number of edges.  This should be very easy.%
~\hfill{\tiny\hyperlink{a-23.b}{[hint]}\hypertarget{q-23.b}{}}\item\label{task-30} \hypertarget{p-261}{}%
Now for the inductive case.  Start with an arbitrary tree \(T\) with \(n\) vertices and assume that \emph{all} trees with \(n-1\) vertices have \(n-2\) edges (why is the the right thing to assume)?  Prove that \(T\) has \(n-1\) edges.%
\par\smallskip%
\noindent\textbf{Hint.}\hypertarget{hint-6}{}\quad%
\hypertarget{p-262}{}%
To get to a tree with \(n-1\) vertices, you need to \emph{trim} \(T\) somehow.  Which vertex should you get rid of?  What will this do to the number of edges?%
~\hfill{\tiny\hyperlink{a-23.c}{[hint]}\hypertarget{q-23.c}{}}\end{enumerate}
\end{activity}
\end{document}
