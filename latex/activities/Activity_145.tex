\documentclass{book}

\input{../activities-preamble.tex}
\begin{document}
\setcounter{project}{145}
\addtocounter{project}{-1}
\begin{activity}[]\label{activity-138}
\hypertarget{p-981}{}%
Given any sequence \((a_n)_{n \ge 1}\), we can form the \terminology{sequence of partial sums} \((b_n)_{n \ge 1}\) given by \(b_n = \sum_{i = 1}^n a_n\).  That is, \(b_n\) tells you what you get if you add up the first \(n\) terms of \((a_n)\).%
\begin{enumerate}[font=\bfseries,label=(\alph*),ref=\alph*]
\item\label{task-176} \hypertarget{p-982}{}%
In general, what is a recurrence relation for \((b_n)\)?%
~\hfill{\tiny\hyperlink{a-145.a}{[hint]}\hypertarget{q-145.a}{}}\item\label{task-177} \hypertarget{p-984}{}%
If \((a_n)_{n \ge 1}\) is the sequence \(1, 3, 5, 7, 9, \ldots\), find the sequence of partial sums \(b_n\).%
\item\label{task-178} \hypertarget{p-985}{}%
If \(a_n = \binom{n}{2}\), find the sequence of partial sums \(b_n\).%
\item\label{task-179} \hypertarget{p-986}{}%
If \(a_n = F_n\), the \(n\)-th Fibonacci number, find the sequence of partial sums \(b_n\).%
\end{enumerate}
\end{activity}
\end{document}
