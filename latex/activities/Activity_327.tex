\documentclass{book}

\input{../activities-preamble.tex}
\begin{document}
\setcounter{project}{327}
\addtocounter{project}{-1}
\begin{activity}[]\label{qtorialformula}
\hypertarget{p-1684}{}%
In \hyperref[largestpartatmostm]{Task~\ref{activity-310}.\ref{largestpartatmostm}} and \hyperref[atmostmparts]{Activity~\ref{atmostmparts}} you gave the generating functions for, respectively, the number of partitions of \(k\) into parts the largest of which is at most \(m\) and for the number of partitions of \(k\) into at most \(m\) parts. In this problem we will give the generating function for the number of partitions of \(k\) into at most \(n\) parts, the largest of which is at most \(m\). That is we will analyze \(\sum_{i=0}^\infty a_kq^k\) where \(a_k\) is the number of partitions of \(k\) into at most \(n\) parts, the largest of which is at most \(m\). Geometrically, it is the generating function for partitions whose Young diagram fits into an \(m\) by \(n\) rectangle, as in \hyperref[rectanglecomplement]{Activity~\ref{rectanglecomplement}}. This generating function has significant analogs to the binomial coefficient \(\binom{m+n}{n}\), and so it is denoted by \(\qchoose{m+n}{n}\). It is called a \terminology{\(q\)-binomial coefficient}.\index{\(q\)-binomial coefficient}\index{binomial coefficient!\(q\)-binomial}%
\begin{enumerate}[font=\bfseries,label=(\alph*),ref=\alph*]
\item\label{task-282} \hypertarget{p-1685}{}%
Compute \(\qchoose{4}{2}=\qchoose{2+2}{2}\).%
~\hfill{\tiny\hyperlink{a-327.a}{[hint]}\hypertarget{q-327.a}{}}\par\smallskip%
\noindent\textbf{Solution.}\hypertarget{solution-263}{}\quad%
\hypertarget{p-1687}{}%
A partition with up to two parts of size up to two can have no parts, one part of size 1, which makes it a partition of 1, one part of size 2 which makes it a partition of 2, two parts of size 1, which makes it a partition of 2, a part of size 2 and a part of size 1, which makes it a partition of 3, or two parts of size 2, which makes it a partition of 4. Thus \(\qchoose{4}{2}=1+q+2q^2+q^3+q^4\).%
\item\label{task-283} \hypertarget{p-1688}{}%
Find explicit formulas for \(\qchoose{n}{1}\) and \(\qchoose{n}{n-1}\).%
~\hfill{\tiny\hyperlink{a-327.b}{[hint]}\hypertarget{q-327.b}{}}\par\smallskip%
\noindent\textbf{Solution.}\hypertarget{solution-264}{}\quad%
\hypertarget{p-1690}{}%
Both are \(1+q+q^2+\cdots+q^{n-1} = \frac{1-q^{n-1}}{1-q}\), because they are the generating function for the number of partitions whose Young diagram fits into a rectangle \(n-1\) units wide and 1 unit deep or into a rectangle 1 unit wide and \(n-1\) units deep respectively.%
\item\label{task-284} \hypertarget{p-1691}{}%
How are \(\qchoose{m+n}{n}\) and \(\qchoose{m+n}{n}\) related? Prove it. (Note this is the same as asking how \(\qchoose{r}{s}\) and \(\qchoose{r}{r-s}\) are related.)%
~\hfill{\tiny\hyperlink{a-327.c}{[hint]}\hypertarget{q-327.c}{}}\par\smallskip%
\noindent\textbf{Solution.}\hypertarget{solution-265}{}\quad%
\hypertarget{p-1693}{}%
By conjugation, \(\qchoose{m+n}{n}=\qchoose{m+n}{m}\).%
\item\label{task-285} \hypertarget{p-1694}{}%
So far the analogy to \(\binom{m+n}{n}\) is rather thin! If we had a recurrence like the Pascal recurrence, that would demonstrate a real analogy. Is \(\qchoose{m+n}{n}= \qchoose{m+n-1}{n-1}+\qchoose{m+n-1}{n}\)?%
\par\smallskip%
\noindent\textbf{Solution.}\hypertarget{solution-266}{}\quad%
\hypertarget{p-1695}{}%
No. \(\qchoose{4}{2}=1+q+2q^2+q^3+q^4\), but \(\qchoose{3}{1}=1+q+q^2\) and \(\qchoose{3}{2}=1+q+q^2\).%
\item\label{task-286} \hypertarget{p-1696}{}%
Recall the two operations we studied in \hyperref[numberpartitionrecurrence]{Activity~\ref{numberpartitionrecurrence}}.%
\begin{enumerate}[font=\bfseries,label=(\roman*),ref=\theenumi.\roman*]
\item\label{task-287} \hypertarget{p-1697}{}%
The largest part of a partition counted by \(\qchoose{m+n}{n}\) is either \(m\) or is less than or equal to \(m-1\).  In the second case, the partition fits into a rectangle that is at most \(m-1\) units wide and at most \(n\) units deep.  What is the generating function for partitions of this type?  In the first case, what kind of rectangle does the partition we get by removing the largest part sit in?  What is the generating function for partitions that sit in this kind of rectangle?  What is the generating function for partitions that sit in this kind of rectangle after we remove a largest part of size \(m\)?  What recurrence relation does this give you?%
\par\smallskip%
\noindent\textbf{Solution.}\hypertarget{solution-267}{}\quad%
\hypertarget{p-1698}{}%
The generating function for partitions that arise in the second case is \(\qchoose{m+n-1}{m-1}=\qchoose{m+n-1}{n}\). In the first case, after we delete the largest part, the Young diagram sits in a rectangle of width \(m\) and depth \(n-1\). The generating function for partitions that arise in the first case \emph{after} we have deleted a part of size \(m\) is \(\qchoose{m+n-1}{n-1}\). Thus the generating function for partitions that arise in the first case (\emph{before} we delete the part of size \(m\)) is \(q^m\qchoose{m+n-1}{n-1}\). Thus by the sum principle the generating function for all partitions that fit into a rectangle of width \(m\) and depth \(n\) is given by%
\begin{equation*}
\qchoose{m+n}{n}= q^m\qchoose{m+n-1}{n-1}+\qchoose{m+n-1}{n}.
\end{equation*}
%
\par
\hypertarget{p-1699}{}%
If you don't use the symmetry \(\qchoose{m+n-1}{m-1}=\qchoose{m+n-1}{n}\), you get%
\begin{equation*}
\qchoose{m+n}{n}= q^m\qchoose{m+n-1}{n-1}+\qchoose{m+n-1}{m-1}.
\end{equation*}
%
\par
\hypertarget{p-1700}{}%
While this doesn't \emph{quite} look like the Pascal recurrence, it is still a correct answer.%
\item\label{task-288} \hypertarget{p-1701}{}%
What recurrence do you get from the other operation we studied in \hyperref[numberpartitionrecurrence]{Activity~\ref{numberpartitionrecurrence}}?%
\par\smallskip%
\noindent\textbf{Solution.}\hypertarget{solution-268}{}\quad%
\hypertarget{p-1702}{}%
We either have exactly \(k\) parts or fewer than \(k\) parts. In the first case, removing one from each part gives us a partition whose Young diagram fits into a \(m-1\) by \(n\) box, while in the second case doing nothing gives us a partition that fits into a \(m\) by \(n-1\) box. In the first case the partition we get partitions \(k-n\), and in the second case it still partitions \(k\). Thus we get%
\begin{equation*}
\qchoose{m+n}{n}= q^n\qchoose{m+n-1}{n}+\qchoose{m+n-1}{n-1}.
\end{equation*}
%
\item\label{task-289} \hypertarget{p-1703}{}%
It is quite likely that the two recurrences you got are different.  One would expect that they might give different values for \(\qchoose{m+n}{n}\).  Can you resolve this potential conflict?%
~\hfill{\tiny\hyperlink{a-327.e.iii}{[hint]}\hypertarget{q-327.e.iii}{}}\par\smallskip%
\noindent\textbf{Solution.}\hypertarget{solution-269}{}\quad%
\hypertarget{p-1705}{}%
Yes, by conjugation.%
\end{enumerate}
\item\label{task-290} \hypertarget{p-1706}{}%
Define \([n]_q\) to be \(1+q+\cdots+q^{n-1}\) for \(n>0\) and \([0]_q =1\).  We read this simply as \(n\)-sub-\(q\). Define \([n]!_q\) to be \([n]_q[n-1]_q\cdots [3]_q[2]_q[1]_q\). We read this as \(n\) cue-torial, and refer to it as a \terminology{\(q\)-ary factorial}.\index{factorial!\(q\)-ary}\index{\(q\)-ary factorial} Show that%
\begin{equation*}
\qchoose{m+n}{n} = \frac{[m+n]!_q}{[m]!_q[n]!_q}.
\end{equation*}
%
~\hfill{\tiny\hyperlink{a-327.f}{[hint]}\hypertarget{q-327.f}{}}\par\smallskip%
\noindent\textbf{Solution.}\hypertarget{solution-270}{}\quad%
\hypertarget{p-1708}{}%
Note that \(\qchoose{n}{0}= \qchoose{n}{n} = 1\), because only the partition with no parts sits in a rectangle of width or depth 0. But \(\frac{[m+0]!_q}{[m]!_q[0]!_q} =1\) and \(\frac{[0+n]!_q}{[0]!_q[n]!_q} =1\). Thus the formula holds when \(m=0\) or \(n=0\). But%
\begin{align*}
\amp q^m\frac{[m+n-1]!_q}{[n-1]!_q[m]!_q}+\frac{[m+n-1]!_q}{[n]!_q[m-1]!_q}\\
=\amp \relax[m+n-1]!_q\left(\frac{q^m}{[n-1]!_q[m]!_q}
+\frac{1}{[n]!_q[m-1]!_q}\right)\\
\amp \relax[m+n-1]!_q\left(\frac{[n]_qq^m}{[n]!_q[m]_q}+\frac{[m]_q}{[n]!_q[m]!_q}\right)\\
=\amp \relax\frac{[m+n-1]!_q}{[n]!_q[m]!_q}\left((1+q+\cdots+q^{n-1})q^m +
1+q+\cdots+q^{m-1}\right)\\
=\amp \relax\frac{[m+n-1]!_q[m+n]_q}{[n]!_q[m]!_q}\\
=\amp \qchoose{m+n}{n}.
\end{align*}
%
\par
\hypertarget{p-1709}{}%
Thus \(\frac{[m+n]!_q}{[m]!_q[n]!_q}\) satisfies our recurrence and so by the principle of mathematical induction, \(\qchoose{m+n}{n} = \frac{[m+n]!_q}{[m]!_q[n]!_q}.\)%
\item\label{task-291} \hypertarget{p-1710}{}%
Now think of \(q\) as a variable that we will let approach \(1\). Find an explicit formula for \leavevmode%
\begin{enumerate}[label=(\roman*)]
\item\hypertarget{li-62}{}\(\displaystyle\lim_{q\rightarrow 1} [n]_q\).%
\item\hypertarget{li-63}{}\(\displaystyle\lim_{q\rightarrow 1} [n]!_q\).%
\item\hypertarget{q-binomial-lim}{}\(\displaystyle\lim_{q\rightarrow 1} \qchoose{m+n}{n}\).%
\end{enumerate}
 Why is the limit in \hyperlink{q-binomial-lim}{Part~iii} equal to the number of partitions (of any number) with at most \(n\) parts all of size most \(m\)? Can you explain bijectively why this quantity equals the formula you got?%
~\hfill{\tiny\hyperlink{a-327.g}{[hint]}\hypertarget{q-327.g}{}}\par\smallskip%
\noindent\textbf{Solution.}\hypertarget{solution-271}{}\quad%
\hypertarget{p-1712}{}%
\leavevmode%
\begin{enumerate}[label=(\roman*)]
\item\hypertarget{li-65}{}\(n\)%
\item\hypertarget{li-66}{}\(n!\)%
\item\hypertarget{li-67}{}\(\binom{m+n}{n}\)%
\end{enumerate}
 Since the generating function is a finite sum (we are talking about partitions whose Young diagram into a finite rectangle), the limit is obtained by setting \(q=1\), and this sums the number of partitions of each possible number \(k\) that have at most \(n\) parts all of size at most \(m\). We want a bijection between such partitions and the \(n\) element subsets of an \(m+n\) element set. Recall that there is a bijection between subsets of an \(n\) element set and lattice paths from \((0,0)\) to \(m,n\) in a coordinate plane. If we draw our rectangle of width \(m\) and depth \(n\) with its lower left corner at \((0,0)\), then each Young diagram gives us such a lattice path and each such lattice path gives us a Young diagram.%
\item\label{task-292} \hypertarget{p-1713}{}%
What happens to \(\qchoose{m+n}{n}\) if we let \(q\) approach \(-1\)?%
~\hfill{\tiny\hyperlink{a-327.h}{[hint]}\hypertarget{q-327.h}{}}\par\smallskip%
\noindent\textbf{Solution.}\hypertarget{solution-272}{}\quad%
\hypertarget{p-1716}{}%
\(\qchoose{m+n}{n}\) is the generating function for the number of partitions whose Young diagram fits into an \(m\) by \(n\) rectangle. That is,%
\begin{equation*}
\qchoose{m+n}{n}=\sum_{k=0}^\infty a_kq^k,
\end{equation*}
where \(a_k\) is the number of partitions of \(k\) whose Young diagram fits into an \(m\) by \(n\) rectangle. In particular \(a_k=0\) if \(k>mn\), because the Young diagram of a partition of a number larger than \(mn\) certainly cannot fit into an \(m\) by \(n\) rectangle. Thus%
\begin{equation*}
\qchoose{m+n}{n}=\sum_{k=0}^{mn} a_kq^k.
\end{equation*}
%
\par
\hypertarget{p-1717}{}%
Furthermore, \(a_k=a_{mn-k}\), because the complement in an \(m\) by \(n\) rectangle of a partition of \(k\) is a partition of \(mn-k\), and complementation in an \(m\) by \(n\) rectangle is a bijection between partitions of \(k\) that fit into the rectangle and partitions of \(mn-k\) that fit into the rectangle. Thus \(\qchoose{m+n}{n}\) is a polynomial of degree \(mn\) in which the coefficient of \(q_k\) equals the coefficient of \(q^{mn-k}\). For this reason we can compute the limit as \(q\) approaches \(-1\) simply by substituting \(-1\) for \(q\) in the polynomial; the only trouble being that the formula we know for \(\qchoose{m+n}{n}\) is quotient of polynomials that has zero in both the numerator and denominator when we substitute in \(q=-1\). Nonetheless, when we substitute \(-1\) for \(q\) we get the alternating sum \(\sum_{i=0}^{mn} (-1)^ia_i\).  Thus if \(a_i\) and \(a_{mn-i}\) have opposite sign, they will cancel out. If \(mn\) is even, \(i\) is even exactly when \(mn-i\) is even, and so \(a_i\) and \(a_{mn-i}\) have the same sign. However when \(mn\) is odd, \(a_i\) and \(a_{mn-i}\) have opposite signs in the sum \(\sum_{i=0}^{mn} (-1)^ia_i\), and so the sum is zero.  Thus the polynomial \(\qchoose{m+n}{n}\) is zero at \(q=-1\) whenever \(mn\) is odd.%
\par
\hypertarget{p-1718}{}%
Our experience with binomial coefficients might lead us to hope that the alternating sum of the coefficients \(a_k\) will always be zero. However, we computed that \(\qchoose{4}{2}=1+q+2q^2+q^3+q^4\), so%
\begin{equation*}
\lim_{q\rightarrow -1}\qchoose{4}{2}=1-1+2-1+1=1.
\end{equation*}
%
\par
\hypertarget{p-1719}{}%
We could compute some more values of the limit by going back to the definition in this way, but it seems unlikely that we could get enough data to make a good conjecture. However we have the recurrence, and setting \(q=-1\) in the recurrence gives us \hyperref[q-binomial-tab]{Table~\ref{q-binomial-tab}}.  From the table, it is clear that we get binomial coefficients interspersed with 0s in the even numbered rows of our table and repeated binomial coefficients in the odd numbered rows of our table. In particular when \(m+n\) and \(n\) are both even, \(\negchoose{m+n}{n}=\binom{(m+n)/2}{n/2}\), and if \(m+n\) is even and \(n\) is odd, \(\negchoose{m+n}{n}=0\). In our table, at least, if \(m+n\) is odd, it appears that we get \(\negchoose{m+n}{n}=\binom{\lfloor(m+n)/2\rfloor}{\lfloor n/2\rfloor}\). In fact, using the recurrence just as we used it to construct the table, we can prove by induction that%
\begin{equation*}
\negchoose{m+n}{n}=\left\{
\begin{array}{ll}0\amp \mbox{if \(mn\) is odd} \\
\binom{\lfloor(m+n)/2\rfloor}{\lfloor n/2\rfloor}\amp \mbox{otherwise.}
\end{array} \right.
\end{equation*}
%
\begin{table}
\centering
\begin{tabular}{cAllllllllll}
\(m+n\backslash n\)&0&1&2&3&4&5&6&7&8&9\tabularnewline\hrulethin
0&1&0&0&0&0&0&0&0&0&0\tabularnewline[0pt]
1&1&1&0&0&0&0&0&0&0&0\tabularnewline[0pt]
2&1&0&1&0&0&0&0&0&0&0\tabularnewline[0pt]
3&1&1&1&1&0&0&0&0&0&0\tabularnewline[0pt]
4&1&0&2&0&1&0&0&0&0&0\tabularnewline[0pt]
5&1&1&2&2&1&1&0&0&0&0\tabularnewline[0pt]
6&1&0&3&0&3&0&1&0&0&0\tabularnewline[0pt]
7&1&1&3&3&3&3&1&1&0&0\tabularnewline[0pt]
8&1&0&4&0&6&0&4&0&1&0\tabularnewline[0pt]
9&1&1&4&4&6&6&4&4&1&1
\end{tabular}
\caption{\(-1\)-binomial coefficients\label{q-binomial-tab}}
\end{table}
\hypertarget{p-1720}{}%
(Note that \(mn\) is odd if and only if \(m+n\) is even and \(n\) is odd.) Wouldn't it be fascinating to know what we are counting here? It is the number of lattice paths from \((0, 0)\) to \((n - k, k)\) that are symmetric under 180 degree rotation. This and other results are discussed in the paper by John R. Stembridge ``Some hidden relations involving the ten symmetry classes of plane partitions.'' \textsl{J. Combin. Theory Ser. A} 68 (1994), no. 2, 372–409.%
\end{enumerate}
\end{activity}
\end{document}
