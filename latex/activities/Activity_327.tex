\documentclass{book}

\input{../activities-preamble.tex}
\begin{document}
\setcounter{project}{327}
\addtocounter{project}{-1}
\begin{activity}[]\label{qtorialformula}
\hypertarget{p-1665}{}%
In \hyperref[largestpartatmostm]{Task~\ref{activity-310}.\ref{largestpartatmostm}} and \hyperref[atmostmparts]{Activity~\ref{atmostmparts}} you gave the generating functions for, respectively, the number of partitions of \(k\) into parts the largest of which is at most \(m\) and for the number of partitions of \(k\) into at most \(m\) parts. In this problem we will give the generating function for the number of partitions of \(k\) into at most \(n\) parts, the largest of which is at most \(m\). That is we will analyze \(\sum_{i=0}^\infty a_kq^k\) where \(a_k\) is the number of partitions of \(k\) into at most \(n\) parts, the largest of which is at most \(m\). Geometrically, it is the generating function for partitions whose Young diagram fits into an \(m\) by \(n\) rectangle, as in \hyperref[rectanglecomplement]{Activity~\ref{rectanglecomplement}}. This generating function has significant analogs to the binomial coefficient \(\binom{m+n}{n}\), and so it is denoted by \(\qchoose{m+n}{n}\). It is called a \terminology{\(q\)-binomial coefficient}.\index{\(q\)-binomial coefficient}\index{binomial coefficient!\(q\)-binomial}%
\begin{enumerate}[font=\bfseries,label=(\alph*),ref=\alph*]
\item\label{task-282} \hypertarget{p-1666}{}%
Compute \(\qchoose{4}{2}=\qchoose{2+2}{2}\).%
~\hfill{\tiny\hyperlink{a-327.a}{[hint]}\hypertarget{q-327.a}{}}\item\label{task-283} \hypertarget{p-1669}{}%
Find explicit formulas for \(\qchoose{n}{1}\) and \(\qchoose{n}{n-1}\).%
~\hfill{\tiny\hyperlink{a-327.b}{[hint]}\hypertarget{q-327.b}{}}\item\label{task-284} \hypertarget{p-1672}{}%
How are \(\qchoose{m+n}{n}\) and \(\qchoose{m+n}{n}\) related? Prove it. (Note this is the same as asking how \(\qchoose{r}{s}\) and \(\qchoose{r}{r-s}\) are related.)%
~\hfill{\tiny\hyperlink{a-327.c}{[hint]}\hypertarget{q-327.c}{}}\item\label{task-285} \hypertarget{p-1675}{}%
So far the analogy to \(\binom{m+n}{n}\) is rather thin! If we had a recurrence like the Pascal recurrence, that would demonstrate a real analogy. Is \(\qchoose{m+n}{n}= \qchoose{m+n-1}{n-1}+\qchoose{m+n-1}{n}\)?%
\item\label{task-286} \hypertarget{p-1677}{}%
Recall the two operations we studied in \hyperref[numberpartitionrecurrence]{Activity~\ref{numberpartitionrecurrence}}.%
\begin{enumerate}[font=\bfseries,label=(\roman*),ref=\theenumi.\roman*]
\item\label{task-287} \hypertarget{p-1678}{}%
The largest part of a partition counted by \(\qchoose{m+n}{n}\) is either \(m\) or is less than or equal to \(m-1\).  In the second case, the partition fits into a rectangle that is at most \(m-1\) units wide and at most \(n\) units deep.  What is the generating function for partitions of this type?  In the first case, what kind of rectangle does the partition we get by removing the largest part sit in?  What is the generating function for partitions that sit in this kind of rectangle?  What is the generating function for partitions that sit in this kind of rectangle after we remove a largest part of size \(m\)?  What recurrence relation does this give you?%
\item\label{task-288} \hypertarget{p-1682}{}%
What recurrence do you get from the other operation we studied in \hyperref[numberpartitionrecurrence]{Activity~\ref{numberpartitionrecurrence}}?%
\item\label{task-289} \hypertarget{p-1684}{}%
It is quite likely that the two recurrences you got are different.  One would expect that they might give different values for \(\qchoose{m+n}{n}\).  Can you resolve this potential conflict?%
~\hfill{\tiny\hyperlink{a-327.e.iii}{[hint]}\hypertarget{q-327.e.iii}{}}\end{enumerate}
\item\label{task-290} \hypertarget{p-1687}{}%
Define \([n]_q\) to be \(1+q+\cdots+q^{n-1}\) for \(n>0\) and \([0]_q =1\).  We read this simply as \(n\)-sub-\(q\). Define \([n]!_q\) to be \([n]_q[n-1]_q\cdots [3]_q[2]_q[1]_q\). We read this as \(n\) cue-torial, and refer to it as a \terminology{\(q\)-ary factorial}.\index{factorial!\(q\)-ary}\index{\(q\)-ary factorial} Show that%
\begin{equation*}
\qchoose{m+n}{n} = \frac{[m+n]!_q}{[m]!_q[n]!_q}.
\end{equation*}
%
~\hfill{\tiny\hyperlink{a-327.f}{[hint]}\hypertarget{q-327.f}{}}\item\label{task-291} \hypertarget{p-1691}{}%
Now think of \(q\) as a variable that we will let approach \(1\). Find an explicit formula for \leavevmode%
\begin{enumerate}[label=(\roman*)]
\item\hypertarget{li-62}{}\(\displaystyle\lim_{q\rightarrow 1} [n]_q\).%
\item\hypertarget{li-63}{}\(\displaystyle\lim_{q\rightarrow 1} [n]!_q\).%
\item\hypertarget{q-binomial-lim}{}\(\displaystyle\lim_{q\rightarrow 1} \qchoose{m+n}{n}\).%
\end{enumerate}
 Why is the limit in \hyperlink{q-binomial-lim}{Part~iii} equal to the number of partitions (of any number) with at most \(n\) parts all of size most \(m\)? Can you explain bijectively why this quantity equals the formula you got?%
~\hfill{\tiny\hyperlink{a-327.g}{[hint]}\hypertarget{q-327.g}{}}\item\label{task-292} \hypertarget{p-1694}{}%
What happens to \(\qchoose{m+n}{n}\) if we let \(q\) approach \(-1\)?%
~\hfill{\tiny\hyperlink{a-327.h}{[hint]}\hypertarget{q-327.h}{}}\end{enumerate}
\end{activity}
\end{document}
