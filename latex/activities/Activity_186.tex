\documentclass{book}

\input{../activities-preamble.tex}
\begin{document}
\setcounter{project}{186}
\addtocounter{project}{-1}
\begin{activity}[]\label{activity-179}
\hypertarget{p-1083}{}%
Here is a way to establish the closed formula for \(C_n\) using Dyck words.  For simplicity, consider Dyck words using the symbols 0 and 1, insisting that no initial segment contains more 1's than 0's.%
\begin{enumerate}[font=\bfseries,label=(\alph*),ref=\alph*]
\item\label{task-200} \hypertarget{p-1084}{}%
Of all \(2n\)-bit strings of weight \(n\), some are Dyck words and some are not.  Explain which are not.  What do all of these have in common?%
\item\label{task-201} \hypertarget{p-1085}{}%
Suppose some initial segment of a bit string contains one more 1 than 0.  If you changed every 1 to a 0 and every 0 to a 1 in this initial segment, how many 0's and how many 1's would the entire bit string have?%
\item\label{task-202} \hypertarget{p-1086}{}%
Describe a bijection between the set of \(2n\)-bit strings of weight \(n-1\) and the set of \(2n\)-bit strings of weight \(n\) that are NOT Dyck words.%
\item\label{task-203} \hypertarget{p-1087}{}%
Why does this prove that \(C_n = \binom{2n}{n} - \binom{2n}{n-1}\)?%
\end{enumerate}
\end{activity}
\end{document}
