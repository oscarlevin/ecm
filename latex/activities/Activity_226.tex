\documentclass{book}

\input{../activities-preamble.tex}
\begin{document}
\setcounter{cpjt}{226}
\addtocounter{cpjt}{-1}
\begin{activity}\label{relaxedmenage}
\hypertarget{p-1208}{}%
A group of \(n\) married couples comes to a group discussion session where they all sit around a round table. In how many ways can they sit so that no person is next to his or her spouse? (Note that two people of the same sex can sit next to each other.)%
\par\smallskip%
\noindent\textbf{Hint 1}.\hypertarget{hint-146}{}\quad%
\hypertarget{p-1209}{}%
Start with two questions that can apply to any inclusion-exclusion problem. Do you think you would be better off trying to compute the size of a union of sets or the size of a complement of a union of sets? What kinds of sets (that are conceivably of use to you) is it easy to compute the size of? (The second question can be interpreted in different ways, and for each way of interpreting it, the answer may help you see something you can use in solving the problem.)%
\par\smallskip%
\noindent\textbf{Hint 2}.\hypertarget{hint-147}{}\quad%
\hypertarget{p-1210}{}%
Suppose we have a set \(S\) of couples whom we want to seat side by side. We can think of lining up \(|S|\) couples and \(2n - 2|S|\) individual people in a circle.  In how many ways can we arrange this many items in a circle?%
\par\smallskip%
\noindent\end{activity}

\clearpage\end{document}
