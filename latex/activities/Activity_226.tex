\documentclass{book}

\input{../activities-preamble.tex}
\begin{document}
\setcounter{project}{226}
\addtocounter{project}{-1}
\begin{activity}[]\label{relaxedmenage}
\hypertarget{p-1275}{}%
A group of \(n\) married couples comes to a group discussion session where they all sit around a round table. In how many ways can they sit so that no person is next to his or her spouse? (Note that two people of the same sex can sit next to each other.)%
~\hfill{\tiny\hyperlink{a-226}{[hint]}\hypertarget{q-226}{}}\par\smallskip%
\noindent\textbf{Solution.}\hypertarget{solution-162}{}\quad%
\hypertarget{p-1278}{}%
Let \(A\) be the set of all seating arrangements for \(2n\) people around a round table. Let \(A_i\) be the set of arrangements in which couple \(i\) sits together. We are interested in \(|\overline{A_1\cup A_2\cup \cdots \cup A_n}|\). Thus, for a set \(S\subseteq [n]\), we need to compute \(\left|\bigcup_{i\colon i\in S} A_i\right|\). If we let each couple described by \(S\) sit together, we will seat \(|S|\) couples and \(2n-2|S|\) individuals around the table. We can do this in \(2^{|S|}(|S| + 2n-2 |S|-1)!\) ways, because once we chose a place for a couple (i.e.\@, two adjacent seats) there are two ways the couple can sit down. Thus as long as \(S\) is nonempty, we have the right formula for \(\left|\bigcap_{i\colon i\in S} A_i\right|\). Notice that in the case where \(S=\emptyset\) the formula gives us \((2n-1)!\) seating arrangements, which is exactly the number of ways to seat \(2n\) people around a round table. This is the size of our set \(A\). Therefore,%
\begin{equation*}
\left|\bigcap_{i\colon i\in S} A_i\right| = 2^{|S|}(2n-|S|-1)!\text{.}
\end{equation*}
Substituting this into the formula from \hyperref[compunion]{Activity~\ref{compunion}} gives us%
\begin{align*}
\left|\overline{\bigcup_{i=1}^n A_i}\right| &=  \sum_{S:S\subseteq [n]}
(-1)^{|S|} \left|\bigcap_{i\colon i\in S} A_i\right|\\
&= \sum_{s=0}^n(-1)^s\binom{n}{s}2^{s}(2n- s-1)!\text{.}
\end{align*}
%
\end{activity}
\end{document}
