\documentclass{book}

\input{../activities-preamble.tex}
\begin{document}
\setcounter{project}{79}
\addtocounter{project}{-1}
\begin{activity}[]\label{act-pascalrowsum-binom}
\hypertarget{p-626}{}%
What does the binomial theorem say again?  That is, expand \((x+y)^n\).  What happens when you substitute \(x = y = 1\)?%
\par\smallskip%
\noindent\textbf{Hint.}\hypertarget{hint-36}{}\quad%
\hypertarget{p-627}{}%
Just do it.%
~\hfill{\tiny\hyperlink{a-79}{[hint]}\hypertarget{q-79}{}}\par\smallskip%
\noindent\textbf{Solution.}\hypertarget{solution-62}{}\quad%
\hypertarget{p-628}{}%
We have \((x+y)^n = \sum_{k=0}^n \binom{n}{k} x^{n-k}y^k\) by the binomial theorem.  When you substitute \(x = y = 1\) you get%
\begin{equation*}
(1+1)^n = \sum_{k=0}^n \binom{n}{k} 1^{n-k}1^k
\end{equation*}
%
\begin{equation*}
2^n = \sum_{k=0}^n \binom{n}{k}.
\end{equation*}
%
\end{activity}
\end{document}
