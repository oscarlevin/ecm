\documentclass{book}

\input{../activities-preamble.tex}
\begin{document}
\setcounter{cpjt}{206}
\addtocounter{cpjt}{-1}
\begin{activity}\label{BellNumberIntro}
\leavevmode%
\begin{enumerate}[font=\bfseries,label=(\alph*),ref=\alph*]
\item\label{task-213} \hypertarget{p-1108}{}%
Show, by explicitly exhibiting the partitions, that \(B_1 = 1\), \(B_2 = 2\), and \(B_3 = 5\).%
\par\smallskip%
\noindent\item\label{task-214} \hypertarget{p-1110}{}%
Why is \(B_k = \sum_{n=1}^{k} S(k,n)\), but \(n^k \ne \sum_{n=1}^k S(k,n)n!\)?  Why is this a meaningful question?%
\par\smallskip%
\noindent\textbf{Hint}.\hypertarget{hint-130}{}\quad%
\hypertarget{p-1111}{}%
As to why this is a reasonable question, think of distributing \(k\) items to \(n\) recipients.  All four expressions count the number of ways to do this under different restrictions.%
\item\label{task-215} \hypertarget{p-1112}{}%
Find a recurrence that expresses \(B_k\) in terms of \(B_n\) for \(n\lt  k\) and prove your formula correct in as many ways as you can.%
\par\smallskip%
\noindent\textbf{Hint}.\hypertarget{hint-131}{}\quad%
\hypertarget{p-1113}{}%
Here it is helpful to think about what happens if you delete the entire block containing \(k\) rather than thinking about whether \(k\) is in a block by itself or not.%
\par\smallskip%
\noindent\item\label{task-216} \hypertarget{p-1116}{}%
Find \(B_k\) for \(k=4,5,6\).%
\par\smallskip%
\noindent\end{enumerate}
\end{activity}

\clearpage\end{document}
