\documentclass{book}

\input{../activities-preamble.tex}
\begin{document}
\setcounter{project}{237}
\addtocounter{project}{-1}
\begin{activity}[]\label{activity-230}
\hypertarget{p-1324}{}%
Substitute \(x\) for all of \(A\), \(P\) and \(B\) (or for the corresponding pictures) in the formula you got in \hyperref[twopiecesoffruit]{Activity~\ref{twopiecesoffruit}} and expand the result in powers of \(x\). Give an interpretation of the coefficient of \(x^n\).%
\par\smallskip%
\noindent\textbf{Solution.}\hypertarget{solution-172}{}\quad%
\hypertarget{p-1325}{}%
\(x^3+3x^4+4x^5+3x^6+x^7\). There is one way to choose three pieces of fruit, there are three ways to choose four pieces, four ways to chose 5 pieces, three ways to choose 6 pieces , and there is one way to choose 7 pieces of fruit. The coefficient of \(x^n\) is the number of ways to choose \(n\) pieces of fruit.%
\end{activity}
\end{document}
