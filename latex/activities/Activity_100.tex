\documentclass{book}

\input{../activities-preamble.tex}
\begin{document}
\setcounter{cpjt}{100}
\addtocounter{cpjt}{-1}
\begin{activity}\label{activity-93}
\hypertarget{p-702}{}%
A \terminology{derangement} of a set of elements is a permutation of those elements so that no element appears in its original position.  For example, a derangement of \([4]\) might be \(3,1,4,2\), but the permutation \(3,2,4,1\) is not a derangement because \(2\) is still in the second position.%
\par
\hypertarget{p-703}{}%
Let \(d_{n}\) denote the number of derangements of \([n]\).  We will take \(d_{0} = 1\) and \(d_{1} = 0\).  Prove that \(d_{n} = (n - 1)(d_{n - 1}+ d_{n - 2})\) for \(n \geq 2\).%
\par\smallskip%
\noindent\textbf{Hint}.\hypertarget{hint-54}{}\quad%
\hypertarget{p-704}{}%
Where can \(n\) go?%
\par\smallskip%
\noindent\end{activity}

\clearpage\end{document}
