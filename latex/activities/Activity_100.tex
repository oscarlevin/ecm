\documentclass{book}

\input{../activities-preamble.tex}
\begin{document}
\setcounter{project}{100}
\addtocounter{project}{-1}
\begin{activity}[]\label{activity-93}
\hypertarget{p-730}{}%
A \terminology{derangement} of a set of elements is a permutation of those elements so that no element appears in its original position.  For example, a derangement of \([4]\) might be \(3,1,4,2\), but the permutation \(3,2,4,1\) is not a derangement because \(2\) is still in the second position.%
\par
\hypertarget{p-731}{}%
Let \(d_{n}\) denote the number of derangements of \([n]\).  We will take \(d_{0} = 1\) and \(d_{1} = 0\).  Prove that \(d_{n} = (n - 1)(d_{n - 1}+ d_{n - 2})\) for \(n \geq 2\).%
\par\smallskip%
\noindent\textbf{Hint.}\hypertarget{hint-54}{}\quad%
\hypertarget{p-732}{}%
Where can \(n\) go?%
~\hfill{\tiny\hyperlink{a-100}{[hint]}\hypertarget{q-100}{}}\par\smallskip%
\noindent\textbf{Solution.}\hypertarget{solution-65}{}\quad%
\hypertarget{p-733}{}%
In forming a derangement of \(1, 2, 3, \ldots, n\) the integer \(n\) can be placed in any of the \(n - 1\) spots \(1, 2, 3, \ldots, n - 1\), say spot \(i\). If \(i\) goes into spot \(n\) there are \(d_{n - 2}\) ways to finish it. If \(i\) does not go into spot \(n\) there are \(d_{n - 1}\) ways to complete the derangement.%
\end{activity}
\end{document}
