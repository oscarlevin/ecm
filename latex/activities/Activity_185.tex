\documentclass{book}

\input{../activities-preamble.tex}
\begin{document}
\setcounter{project}{185}
\addtocounter{project}{-1}
\begin{activity}[]\label{activity-178}
\hypertarget{p-1100}{}%
Show by example, how the bijection between Dyck words and valid parenthesizing works.  To do this, list the \(C_4 = 14\) valid Dyck words, then list the \(C_4 = 14\) ways to parenthesize \(abcde\), in the corresponding order.%
\par\smallskip%
\noindent\textbf{Solution.}\hypertarget{solution-117}{}\quad%
\hypertarget{p-1101}{}%
Some examples: HDHHDHDD corresponds with \(a((bc)(de))\).  The string HHHHDDDD corresponds with \((((ab)c)d)e\).  You should list all 14 of each.%
\end{activity}
\end{document}
