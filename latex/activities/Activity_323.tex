\documentclass{book}

\input{../activities-preamble.tex}
\begin{document}
\setcounter{project}{323}
\addtocounter{project}{-1}
\begin{activity}[]\label{activity-316}
\hypertarget{p-1674}{}%
Use generating functions to explain why the number of partitions of an integer in which each part is used an even number of times equals the number of partitions of an integer in which each part is even.%
~\hfill{\tiny\hyperlink{a-323}{[hint]}\hypertarget{q-323}{}}\par\smallskip%
\noindent\textbf{Solution.}\hypertarget{solution-261}{}\quad%
\hypertarget{p-1676}{}%
In the generating function \(\displaystyle\prod_{i=1}^\infty \sum_{j=0}^\infty q^{2ij}\), we may interpret the \(2ij\) in \(q^{2ji}\) the value of using \(2i\) as a part \(j\) times or as the value of using \(i\) as a part \(2j\) times. Therefore this is the generating function both for the number of partitions of integers into parts that are even and the number of partitions into parts that are used an even number of times. Therefore the number of partitions of \(n\) in which each part is even equals the number of partitions of \(n\) in which each part is used an even number of times. In \hyperref[partition-even-mult-even-use]{Activity~\ref{partition-even-mult-even-use}} we got the same result bijectively.%
\end{activity}
\end{document}
