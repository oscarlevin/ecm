\documentclass{book}

\input{../activities-preamble.tex}
\begin{document}
\setcounter{cpjt}{175}
\addtocounter{cpjt}{-1}
\begin{activity}\label{act-parenthesize}
\hypertarget{p-998}{}%
If we multiply \(n+1\) numbers, say \(a_1a_2\cdots a_n a_{n+1}\), we really should put in \(n\) pairs of parentheses since multiplication is a binary operation (and what if multiplication is not associative?).  How many ways can we do this?  For example, when \(n = 3\), there are 5 ways:%
\begin{equation*}
(ab)(cd)\quad ((ab)c)d \quad a(b(cd)) \quad (a(bc))d \quad a((bc)d).
\end{equation*}
%
\par\smallskip%
\noindent\textbf{Hint}.\hypertarget{hint-114}{}\quad%
\hypertarget{p-999}{}%
Try working recursively.   For a product of \(n+1\) symbols \(a_{1}a_{2}\ldots a_{n+1}\), break it at the \(k\)th symbol:%
\begin{equation*}
(a_{1}a_{2}a_{3}\ldots a_{k})(a_{k + 1}\ldots a_{n+1}).
\end{equation*}
%
\end{activity}

\clearpage\end{document}
