\documentclass{book}

\input{../activities-preamble.tex}
\begin{document}
\setcounter{project}{146}
\addtocounter{project}{-1}
\begin{activity}[]\label{activity-139}
\hypertarget{p-952}{}%
Let \((a_n)_{n \ge 1}\) be any arithmetic sequence.  Say \(a_n = dn + c\) (so \(d\) is the common difference and \(c = a_0\)).  What can we say about the closed formula for \(b_n = \sum_{i=1}^n a_i\)?%
\begin{enumerate}[font=\bfseries,label=(\alph*),ref=\alph*]
\item\label{task-179} \hypertarget{p-953}{}%
As an example, what is \(2+5+8+11+\cdots + 470\)?  You might call this sum \(S\) and calculate:%
\begin{sidebyside}{1}{0}{0}{0}
\begin{sbspanel}{1}
{\centering%
\begin{tabular}{rccccccccc}
\(S  =\)&\(2\)&\(+\)&\(5\)&\(+\)&\(8\)&\(+ \cdots +\)&\(467\)&\(+\)&470\tabularnewline[0pt]
\(+ \quad S  =\)&\(470\)&\(+\)&\(467\)&\(+\)&\(464\)&\(+ \cdots +\)&\(5\)&\(+\)&2\tabularnewline\hrulethin
\(2S  =\)&\(472\)&\(+\)&\(472\)&\(+\)&\(472\)&\(+ \cdots +\)&\(472\)&\(+\)&\(472\)
\end{tabular}
\par}
\end{sbspanel}
\end{sidebyside}
\par
\hypertarget{p-954}{}%
Why is that helpful?  What is the sum?%
\par\smallskip%
\noindent\textbf{Hint.}\hypertarget{hint-105}{}\quad%
\hypertarget{p-955}{}%
You will need to decide how many terms are on the right-hand side.%
~\hfill{\tiny\hyperlink{a-146.a}{[hint]}\hypertarget{q-146.a}{}}\item\label{task-180} \hypertarget{p-956}{}%
Generalize the previous computation to find the closed formula for \(b_n\).%
\item\label{task-181} \hypertarget{p-957}{}%
What is special about sums of arithmetic sequences that allow you to do this computation?  Explain.%
\end{enumerate}
\end{activity}

\clearpage\end{document}
