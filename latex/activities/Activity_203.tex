\documentclass{book}

\input{../activities-preamble.tex}
\begin{document}
\setcounter{project}{203}
\addtocounter{project}{-1}
\begin{activity}[]\label{activity-196}
\hypertarget{p-1149}{}%
Given a function \(f\) from a \(k\)-element set \(K\) to an \(n\)-element set, we can define a partition of \(K\) by putting \(x\) and \(y\) in the same block of the partition if and only if \(f(x)=f(y)\). How many blocks does the partition have if \(f\) is surjective? How is the number of functions from a \(k\)-element set \emph{onto} an \(n\)-element set related to a Stirling number? Be as precise in your answer as you can.\index{function!surjective!and Stirling Numbers}\index{surjective function!counting}%
~\hfill{\tiny\hyperlink{a-203}{[hint]}\hypertarget{q-203}{}}\par\smallskip%
\noindent\textbf{Solution.}\hypertarget{solution-118}{}\quad%
\hypertarget{p-1151}{}%
If \(f\) is onto, the number of blocks of the partition is \(n\). The number of onto functions from a \(k\)-element set onto an \(n\)-element set is \(S(k,n)n!\), because we have a one-to-one function from the blocks to the \(n\)-element set.%
\end{activity}
\end{document}
