\documentclass{book}

\input{../activities-preamble.tex}
\begin{document}
\setcounter{project}{312}
\addtocounter{project}{-1}
\begin{activity}[]\label{activity-305}
\hypertarget{p-1611}{}%
There is a relationship between \(p_n(k)\) and \(q_n(m)\) for some other number \(m\). Find the number \(m\) that gives you the nicest possible relationship.%
~\hfill{\tiny\hyperlink{a-312}{[hint]}\hypertarget{q-312}{}}\par\smallskip%
\noindent\textbf{Solution.}\hypertarget{solution-240}{}\quad%
\hypertarget{p-1613}{}%
The number of partitions of \(k\) into \(n\) parts is equal to the number of partitions of \(k+\binom{n}{2}\) into n distinct parts.  The bijection from partitions of \(k\) with \(n\) parts to partitions of \(k+\binom{n}{2}\) with \(n\) distinct parts that proves this is the one that takes a partition \(\lambda_n\lambda_{n-1}\cdots\lambda_1\) of \(k\) with \(\lambda_i>\lambda_{i+1}\) and adds \(i-1\) to \(\lambda_i\) to get \(\lambda'_i\). Then \(\lambda'\) is a partition into distinct parts, and the number it partitions is \(k+1+2+\cdots+n-1=k+\binom{n}{2}\). The proof that it is a bijection is the fact that subtracting \(n-i\) from the \(i\)\/th part of a partition of \(k\) into distinct parts yields a partition of \(k\), because part \(i+j\) is at least \(j\) smaller than part \(i\).%
\end{activity}
\end{document}
