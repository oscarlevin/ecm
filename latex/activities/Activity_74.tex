\documentclass{book}

\input{../activities-preamble.tex}
\begin{document}
\setcounter{cpjt}{74}
\addtocounter{cpjt}{-1}
\begin{activity}\label{act-pascalsym-dc}
\hypertarget{p-557}{}%
Think about what \(\binom{n}{k}\) counts.%
\begin{enumerate}[font=\bfseries,label=(\alph*),ref=\alph*]
\item\label{task-102} \hypertarget{p-558}{}%
Suppose you own \(n\) bow ties, and want to choose \(k\) of them to take on a trip.  How many different collections of bow ties can you take?%
\par\smallskip%
\noindent\textbf{Hint}.\hypertarget{hint-29}{}\quad%
\hypertarget{p-559}{}%
This is supposed to be very easy.%
\item\label{task-103} \hypertarget{p-560}{}%
Why is the answer to the above question also \(\binom{n}{n-k}\)?%
\par\smallskip%
\noindent\textbf{Hint}.\hypertarget{hint-30}{}\quad%
\hypertarget{p-561}{}%
What if you started with all bow ties in the suitcase and got rid of some?  How many do you need to remove?%
\item\label{task-104} \hypertarget{p-562}{}%
Why are the two parts above enough to establish the identity?%
\end{enumerate}
\end{activity}

\clearpage\end{document}
