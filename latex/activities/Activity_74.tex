\documentclass{book}

\input{../activities-preamble.tex}
\begin{document}
\setcounter{project}{74}
\addtocounter{project}{-1}
\begin{activity}[]\label{act-pascalsym-dc}
\hypertarget{p-588}{}%
Think about what \(\binom{n}{k}\) counts.%
\begin{enumerate}[font=\bfseries,label=(\alph*),ref=\alph*]
\item\label{task-105} \hypertarget{p-589}{}%
Suppose you own \(n\) bow ties, and want to choose \(k\) of them to take on a trip.  How many different collections of bow ties can you take?%
~\hfill{\tiny\hyperlink{a-74.a}{[hint]}\hypertarget{q-74.a}{}}\item\label{task-106} \hypertarget{p-592}{}%
Why is the answer to the above question also \(\binom{n}{n-k}\)?%
~\hfill{\tiny\hyperlink{a-74.b}{[hint]}\hypertarget{q-74.b}{}}\item\label{task-107} \hypertarget{p-595}{}%
Why are the two parts above enough to establish the identity?%
\end{enumerate}
\end{activity}
\end{document}
