\documentclass{book}

\input{../activities-preamble.tex}
\begin{document}
\setcounter{project}{74}
\addtocounter{project}{-1}
\begin{activity}[]\label{act-pascalsym-dc}
\hypertarget{p-573}{}%
Think about what \(\binom{n}{k}\) counts.%
\begin{enumerate}[font=\bfseries,label=(\alph*),ref=\alph*]
\item\label{task-104} \hypertarget{p-574}{}%
Suppose you own \(n\) bow ties, and want to choose \(k\) of them to take on a trip.  How many different collections of bow ties can you take?%
\par\smallskip%
\noindent\textbf{Hint.}\hypertarget{hint-29}{}\quad%
\hypertarget{p-575}{}%
This is supposed to be very easy.%
~\hfill{\tiny\hyperlink{a-74.a}{[hint]}\hypertarget{q-74.a}{}}\item\label{task-105} \hypertarget{p-576}{}%
Why is the answer to the above question also \(\binom{n}{n-k}\)?%
\par\smallskip%
\noindent\textbf{Hint.}\hypertarget{hint-30}{}\quad%
\hypertarget{p-577}{}%
What if you started with all bow ties in the suitcase and got rid of some?  How many do you need to remove?%
~\hfill{\tiny\hyperlink{a-74.b}{[hint]}\hypertarget{q-74.b}{}}\item\label{task-106} \hypertarget{p-578}{}%
Why are the two parts above enough to establish the identity?%
\end{enumerate}
\end{activity}

\clearpage\end{document}
