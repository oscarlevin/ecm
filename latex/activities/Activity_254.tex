\documentclass{book}

\input{../activities-preamble.tex}
\begin{document}
\setcounter{project}{254}
\addtocounter{project}{-1}
\begin{activity}[]\label{activity-247}
\hypertarget{p-1372}{}%
If you define \(\binom{-n}{k}\) in the way you described in \hyperref[negnchoosek]{Activity~\ref{negnchoosek}}, you can write down a version of the binomial theorem for \((x+y)^n\) that is valid for both nonnegative and negative values of \(n\). Do so. This is called the \terminology{extended binomial theorem}\index{binomial theorem!extended}\index{extended binomial theorem}. Write down a special case with \(n\) negative, like \(n=-3\), to see an interesting surprise that suggests why we do not use this formula later on.%
\end{activity}
\end{document}
