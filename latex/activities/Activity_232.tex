\documentclass{book}

\input{../activities-preamble.tex}
\begin{document}
\setcounter{project}{232}
\addtocounter{project}{-1}
\begin{activity}[]\label{activity-225}
\hypertarget{p-1283}{}%
If we roll a die eight times, we get a sequence of 8 numbers, the number of dots on top on the first roll, the number on the second roll, and so on.%
\begin{enumerate}[font=\bfseries,label=(\alph*),ref=\alph*]
\item\label{task-233} \hypertarget{p-1284}{}%
What is the number of ways of rolling the die eight times so that each of the numbers one through six appears at least once in our sequence? To get a numerical answer, you will likely need a computer algebra package.%
\item\label{task-234} \hypertarget{p-1286}{}%
What is the probability that we get a sequence in which all six numbers between one and six appear? To get a numerical answer, you will likely need a computer algebra package, programmable calculator, or spreadsheet.%
\item\label{task-235} \hypertarget{p-1288}{}%
How many times do we have to roll the die to have probability at least one half that all six numbers appear in our sequence. To answer this question, you will likely need a computer algebra package, programmable calculator, or spreadsheet.%
\end{enumerate}
\end{activity}
\end{document}
