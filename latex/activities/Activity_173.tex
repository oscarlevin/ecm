\documentclass{book}

\input{../activities-preamble.tex}
\begin{document}
\setcounter{project}{173}
\addtocounter{project}{-1}
\begin{activity}[]\label{act-hdseq}
\hypertarget{p-1044}{}%
\(2n\) people stand in line at a old-timey movie theater. Admission is 50 cents, (denoted by H), and the box office starts with no change. \(n\) of the people have H and \(n\) have \textdollar{}1 (denoted D). In how many ways can the \(2n\) people line up so that all can be admitted?%
~\hfill{\tiny\hyperlink{a-173}{[hint]}\hypertarget{q-173}{}}\par\smallskip%
\noindent\textbf{Solution.}\hypertarget{solution-109}{}\quad%
\hypertarget{p-1046}{}%
The total number of sequences of \(n\) H's and \(n\) D's is \(\binom{2n}{n}\). We delete the number of \emph{nonworkable} sequences. Each such sequence has a first snag as shown by the highlighted D in%
\begin{equation*}
\text{H\ H\ D\ D\ \alert{D}\ H\ H\ D\ D\ D\ H\ H}
\end{equation*}
Reverse each letter up to and including the snag, obtaining%
\begin{equation*}
\text{D D H H H H H D D D H H},
\end{equation*}
a sequence of \(n+1\) H's and \(n-1\) D's. For \(n = 3\) , H D \alert{D} D H H is a non-workable sequence with the snag indicated. Its mate, found by reversing the first three letters, is D H H D H H, a sequence of 2D's and 4H's. \emph{Any} arrangement of 2D's and 4H's will correspond to exactly one non-workable sequence; simply scan through and see the \emph{first} time the H's dominate the D's and then reverse through that spot. There are \(\binom{6}{3}
= 20\) arrangements of 3H's and 3D's. There are \(\binom{6}{2}
= 15\) arrangements of 2D's and 4H's each of which corresponds to a non-workable sequence. Hence there are \(\binom{6}{3}
-\binom{6}{2} = 5\) workable sequences. In general, \(\binom{2n}{n}  - \binom{2n}{n - 1} = \frac{1}{n + 1}\binom{2n}{n}\) gives the number of workable sequences.%
\end{activity}
\end{document}
