\documentclass{book}

\input{../activities-preamble.tex}
\begin{document}
\setcounter{project}{304}
\addtocounter{project}{-1}
\begin{activity}[]\label{activity-297}
\hypertarget{p-1569}{}%
The partition \((\lambda_1,\lambda_2,\ldots, \lambda_n)\) is called the \terminology{conjugate}\index{conjugate of an integer partition}\index{partition of an integer!conjugate of} of the partition \((\gamma_1,\gamma_2,\ldots, \gamma_m)\) if we obtain the Young diagram of one from the Young diagram of the other by flipping one around the line with slope -1 that extends the diagonal of the top left square. See \hyperref[conjugateYoung]{Figure~\ref{conjugateYoung}} for an example.%
\begin{figure}
\centering
\includegraphics[width=0.5\linewidth]{images/conjugateYoung}
\caption{The Ferrers diagram the partition (5,3,3,2) and its conjugate.\label{conjugateYoung}}
\end{figure}
\hypertarget{p-1570}{}%
What is the conjugate of (4,4,3,1,1)? How is the largest part of a partition related to the number of parts of its conjugate? What does this tell you about the number of partitions of a positive integer \(k\) with largest part \(m\)?%
~\hfill{\tiny\hyperlink{a-304}{[hint]}\hypertarget{q-304}{}}\par\smallskip%
\noindent\textbf{Solution.}\hypertarget{solution-228}{}\quad%
\hypertarget{p-1572}{}%
\((5,3,3,2)\). The largest part of a partition equals the number of parts of its conjugate. The number of partitions of \(k\) with largest part \(m\) equals the number of partitions of \(k\) with \(m\) parts.%
\end{activity}
\end{document}
