\documentclass{book}

\input{../activities-preamble.tex}
\begin{document}
\setcounter{cpjt}{319}
\addtocounter{cpjt}{-1}
\begin{activity}\label{genfunpartitions}
\hypertarget{p-1577}{}%
When studying partitions of integers, it is inconvenient to restrict ourselves to partitions with at most \(m\) parts or partitions with maximum part size \(m\).%
\begin{enumerate}[font=\bfseries,label=(\alph*),ref=\alph*]
\item\label{task-271} \hypertarget{p-1578}{}%
Give the generating function for the number of partitions of an integer into parts of any size. Don't forget to use \(q\) rather than \(x\) as your variable.%
\par\smallskip%
\noindent\textbf{Hint}.\hypertarget{hint-208}{}\quad%
\hypertarget{p-1579}{}%
Don't be afraid of writing down a product of infinitely many power series.%
\par\smallskip%
\noindent\item\label{task-272} \hypertarget{p-1581}{}%
Find the coefficient of \(q^4\) in this generating function.%
\par\smallskip%
\noindent\textbf{Hint}.\hypertarget{hint-209}{}\quad%
\hypertarget{p-1582}{}%
From the fifth factor on, there is no way to choose a \(q^i\) that has \(i\) nonzero and less than five from the factor.%
\par\smallskip%
\noindent\item\label{task-273} \hypertarget{p-1584}{}%
Find the coefficient of \(q^5\) in this generating function.%
\par\smallskip%
\noindent\item\label{task-274} \hypertarget{p-1586}{}%
This generating function involves an infinite product. Describe the process you would use to expand this product into as many terms of a power series as you choose.%
\par\smallskip%
\noindent\textbf{Hint}.\hypertarget{hint-210}{}\quad%
\hypertarget{p-1587}{}%
Describe to yourself how to get the coefficient of a given power of \(q\).%
\par\smallskip%
\noindent\item\label{task-275} \hypertarget{p-1589}{}%
Rewrite any power series that appear in your product as quotients of polynomials or as integers divided by polynomials.%
\par\smallskip%
\noindent\end{enumerate}
\end{activity}

\clearpage\end{document}
