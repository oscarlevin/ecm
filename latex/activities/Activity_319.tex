\documentclass{book}

\input{../activities-preamble.tex}
\begin{document}
\setcounter{project}{319}
\addtocounter{project}{-1}
\begin{activity}[]\label{genfunpartitions}
\hypertarget{p-1644}{}%
When studying partitions of integers, it is inconvenient to restrict ourselves to partitions with at most \(m\) parts or partitions with maximum part size \(m\).%
\begin{enumerate}[font=\bfseries,label=(\alph*),ref=\alph*]
\item\label{task-274} \hypertarget{p-1645}{}%
Give the generating function for the number of partitions of an integer into parts of any size. Don't forget to use \(q\) rather than \(x\) as your variable.%
~\hfill{\tiny\hyperlink{a-319.a}{[hint]}\hypertarget{q-319.a}{}}\par\smallskip%
\noindent\textbf{Solution.}\hypertarget{solution-249}{}\quad%
\hypertarget{p-1647}{}%
We start with%
\begin{equation*}
(1+q+q^2+\cdots)(1+q^2+q^4+\cdots)\cdots(1+q^i+q^{2i})\cdots,
\end{equation*}
which we can write more precisely as%
\begin{equation*}
\prod_{i=1}^\infty \sum_{j=0}^\infty q^{ij}.
\end{equation*}
%
\item\label{task-275} \hypertarget{p-1648}{}%
Find the coefficient of \(q^4\) in this generating function.%
~\hfill{\tiny\hyperlink{a-319.b}{[hint]}\hypertarget{q-319.b}{}}\par\smallskip%
\noindent\textbf{Solution.}\hypertarget{solution-250}{}\quad%
\hypertarget{p-1650}{}%
From the fifth factor on, there is no way to choose a \(q^i\) that has \(i\) nonzero and less than five from the factor. Thus we choose a 1 from each of these factors. We can choose a \(q^4\) from the fourth factor and 1 from the rest, a \(q^3\) from the third factor, a \(q\) from the first and a 1 from the rest, a \(q^2\) from the second factor, a \(q^2\) from the first and a 1 from the rest, or we can choose a \(q^4\) from the first factor and a 1 from the rest. Therefore the coefficient of \(q^4\) is five.%
\item\label{task-276} \hypertarget{p-1651}{}%
Find the coefficient of \(q^5\) in this generating function.%
\par\smallskip%
\noindent\textbf{Solution.}\hypertarget{solution-251}{}\quad%
\hypertarget{p-1652}{}%
From the sixth factor of the product on, there is no way to choose a \(q^i\) that has \(i\) nonzero and less than six from the factor, so when we compute the coefficient of \(q^5\), we can only choose the 1 from each of these terms. In the first five factors, we choose any combination of powers that adds to 5. We can choose \(q^5\) from the fifth and 1 from the rest, q from the first, \(q^4\) from the second and 1 from the rest, \(q^2\) from the second, \(q^3\) from the third and 1 from the rest, \(q^2\) from the first and \(q^3\) from the third and 1 from the rest, \(q^3\) from the first, \(q^2\) from the second, and 1 from the rest \(q^5\) from the first and 1 from the rest, and these are the only ways to get a \(q^5\) in the product. Thus the coefficient of \(q^5\) is 7.%
\item\label{task-277} \hypertarget{p-1653}{}%
This generating function involves an infinite product. Describe the process you would use to expand this product into as many terms of a power series as you choose.%
~\hfill{\tiny\hyperlink{a-319.d}{[hint]}\hypertarget{q-319.d}{}}\par\smallskip%
\noindent\textbf{Solution.}\hypertarget{solution-252}{}\quad%
\hypertarget{p-1655}{}%
To get the coefficient of \(q^k\) in this product, we look at all ways of choosing one summand from each of the infinite series and multiplying them together to get \(q^k\), and add all these products up. That is, the coefficient of \(q^k\) is the number of ways of making these choices of one summand from each series so that the product of our choices is \(q^k\). This lets us write down as many terms of the series as we want, or as we have patience for!%
\item\label{task-278} \hypertarget{p-1656}{}%
Rewrite any power series that appear in your product as quotients of polynomials or as integers divided by polynomials.%
\par\smallskip%
\noindent\textbf{Solution.}\hypertarget{solution-253}{}\quad%
\hypertarget{p-1657}{}%
We can rewrite the infinite product as \(\displaystyle\prod_{i=0}^\infty\frac{1}{1-q^i}\).%
\end{enumerate}
\end{activity}
\end{document}
