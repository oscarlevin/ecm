\documentclass{book}

\input{../activities-preamble.tex}
\begin{document}
\setcounter{project}{319}
\addtocounter{project}{-1}
\begin{activity}[]\label{genfunpartitions}
\hypertarget{p-1629}{}%
When studying partitions of integers, it is inconvenient to restrict ourselves to partitions with at most \(m\) parts or partitions with maximum part size \(m\).%
\begin{enumerate}[font=\bfseries,label=(\alph*),ref=\alph*]
\item\label{task-274} \hypertarget{p-1630}{}%
Give the generating function for the number of partitions of an integer into parts of any size. Don't forget to use \(q\) rather than \(x\) as your variable.%
~\hfill{\tiny\hyperlink{a-319.a}{[hint]}\hypertarget{q-319.a}{}}\item\label{task-275} \hypertarget{p-1633}{}%
Find the coefficient of \(q^4\) in this generating function.%
~\hfill{\tiny\hyperlink{a-319.b}{[hint]}\hypertarget{q-319.b}{}}\item\label{task-276} \hypertarget{p-1636}{}%
Find the coefficient of \(q^5\) in this generating function.%
\item\label{task-277} \hypertarget{p-1638}{}%
This generating function involves an infinite product. Describe the process you would use to expand this product into as many terms of a power series as you choose.%
~\hfill{\tiny\hyperlink{a-319.d}{[hint]}\hypertarget{q-319.d}{}}\item\label{task-278} \hypertarget{p-1641}{}%
Rewrite any power series that appear in your product as quotients of polynomials or as integers divided by polynomials.%
\end{enumerate}
\end{activity}
\end{document}
