\documentclass{book}

\input{../activities-preamble.tex}
\begin{document}
\setcounter{project}{306}
\addtocounter{project}{-1}
\begin{activity}[]\label{partition-even-mult-even-use}
\hypertarget{p-1538}{}%
Explain the relationship between the number of partitions of \(k\) into even parts and the number of partitions of \(k\) into parts of even multiplicity, i.e. parts which are each used an even number of times as in (3,3,3,3,2,2,1,1).%
\par\smallskip%
\noindent\textbf{Hint.}\hypertarget{hint-192}{}\quad%
\hypertarget{p-1539}{}%
Draw the partitions of six into even parts. Draw the partitions of six into parts used an even number of times. Look for a relationship between one set of diagrams and the other set of diagrams. If you have trouble, repeat the process using 8 or even 10 in place of 6.%
~\hfill{\tiny\hyperlink{a-306}{[hint]}\hypertarget{q-306}{}}\par\smallskip%
\noindent\textbf{Solution.}\hypertarget{solution-204}{}\quad%
\hypertarget{p-1540}{}%
The number of partitions of \(k\) into even parts equals the number of partitions of parts of even multiplicity, because if we take the Young diagram of a partition of \(k\) into even parts and conjugate it, the resulting diagram has columns of even length. Thus the difference in heights of two successive columns is an even number, but this difference is the multiplicity of one of the parts of the conjugate. Further the height of the last column of a partition is the multiplicity of the first part. Since the multiplicity of any part of a partition is either the difference in height of two successive columns of the Young diagram or the height of the last column, then each part of the conjugate has even multiplicity. This bijection can be reversed, because if all the differences in height of the columns are even and the height of the last column is even, then when we conjugate this partition, the last row will be an even length, and all differences in length of the rows will be even, so all the parts of the resulting partition will be even.%
\end{activity}
\end{document}
