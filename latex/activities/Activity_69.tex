\documentclass{book}

\input{../activities-preamble.tex}
\begin{document}
\setcounter{project}{69}
\addtocounter{project}{-1}
\begin{activity}[]\label{activity-62}
\hypertarget{p-528}{}%
Consider functions \(f:X \to Y\) where \(X\) and \(Y\) are finite sets.  In fact, let's say \(X = [4]\). If you need a refresher on basic ideas about functions, check out \hyperref[sec_background-functions]{Section~\ref{sec_background-functions}}.%
\begin{enumerate}[font=\bfseries,label=(\alph*),ref=\alph*]
\item\label{task-93} \hypertarget{p-529}{}%
Can you say anything at all about \(Y\) if you know there is some function \(f:X \to Y\)?  Give some examples of sets \(Y\) and functions \(f\).%
\item\label{task-94} \hypertarget{p-530}{}%
What if \(f:X \to Y\) is \emph{injective} (in other words, one-to-one)?  Which of the examples you found above no longer work?  What can you conclude about \(Y\)?%
\item\label{task-95} \hypertarget{p-531}{}%
What if \(f:X \to Y\) is \emph{surjective} (i.e., onto)?  Now what can you say about \(Y\)?%
\item\label{task-96} \hypertarget{p-532}{}%
What if \(f:X\to Y\) is \emph{bijective} (so both injective and surjective)?  State the general principle.%
\end{enumerate}
\end{activity}

\clearpage\end{document}
