\documentclass{book}

\input{../activities-preamble.tex}
\begin{document}
\setcounter{project}{117}
\addtocounter{project}{-1}
\begin{activity}[]\label{activity-110}
\hypertarget{p-836}{}%
The number of \(n\)-bit strings of weight \(k\) is \(\binom{n}{k}\), so a combination.  But in determining one bit string from another, all that matters is the order in which the \(k\) 1's and \(n-k\) 0's appear.  So does order matter?  In what sense does it not?%
\par\smallskip%
\noindent\textbf{Hint.}\hypertarget{hint-76}{}\quad%
\hypertarget{p-837}{}%
Think about what \(P(n,k)\) would count when building a bit string.  Why does it make sense to quotient out by \(k!\)?%
~\hfill{\tiny\hyperlink{a-117}{[hint]}\hypertarget{q-117}{}}\end{activity}
\end{document}
