\documentclass{book}

\input{../activities-preamble.tex}
\begin{document}
\setcounter{project}{61}
\addtocounter{project}{-1}
\begin{activity}[]\label{activity-54}
\hypertarget{p-486}{}%
Suppose you deal 52 regular playing cards into 13 piles of 4 cards each. Prove that you can always select one card from each pile to get one of each of the 13 card values Ace, 2, 3, \textellipsis{}, 10, Jack, Queen, and King.%
~\hfill{\tiny\hyperlink{a-61}{[hint]}\hypertarget{q-61}{}}\end{activity}
\end{document}
