\documentclass{book}

\input{../activities-preamble.tex}
\begin{document}
\setcounter{cpjt}{66}
\addtocounter{cpjt}{-1}
\begin{activity}\label{activity-59}
\hypertarget{p-493}{}%
Some definitions: A \terminology{bit} is either 0 or 1 (bit is short for ``binary digit'').  Thus a \terminology{bit string} is a string of bits.  The \terminology{length} of a bit string is the number of bits in the string; the \terminology{weight} of a bit string is the number of 1's in the string (or equivalently, the sum of the bits). A \terminology{\(n\)-bit string} means a bit string of length \(n\).%
\par
\hypertarget{p-494}{}%
We will write \(\B^n_k\) to mean the set of all \(n\)-bit strings of weight \(k\).  So for example, some of the elements in \(\B^5_3\) are,%
\begin{equation*}
11100 \qquad 10101 \qquad 01101.
\end{equation*}
%
\begin{enumerate}[font=\bfseries,label=(\alph*),ref=\alph*]
\item\label{task-82} \hypertarget{p-495}{}%
Write out all \(3\)-bit strings.  Group these by weight.  How many strings are there of each weight?%
\item\label{task-83} \hypertarget{p-496}{}%
Write out all \(4\)-bit strings.  You might want to use the list you had in the previous part as a starting point (how would you do this?).  Again, group these strings by weight and see how many there are of each type.%
\item\label{task-84} \hypertarget{p-497}{}%
How many \(5\)-bit strings of weight 3 are there?  List all of these, and say how they relate to some of the \(4\)-bit strings you found above.%
\end{enumerate}
\end{activity}

\clearpage\end{document}
