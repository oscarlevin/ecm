\documentclass{book}

\input{../activities-preamble.tex}
\begin{document}
\setcounter{project}{258}
\addtocounter{project}{-1}
\begin{activity}[]\label{substituteandsolve}
\hypertarget{p-1375}{}%
Suppose that \(a_i=3a_{i-1} + 3^i\).%
\begin{enumerate}[font=\bfseries,label=(\alph*),ref=\alph*]
\item\label{task-247} \hypertarget{p-1376}{}%
Multiply both sides by \(x^i\) and sum both the left hand side and right hand side from \(i=1\) to infinity.  In the left-hand side use the fact that%
\begin{equation*}
\sum_{i=1}^\infty a_ix^i = (\sum_{i=0}^\infty x^i) -a_0
\end{equation*}
and in the right hand side, use the fact that%
\begin{equation*}
\sum_{i=1}^\infty a_{i-1}x^i = x\sum_{i=1}^\infty a_ix^{i-1}
=x\sum_{j=0}^\infty a_jx^j =x\sum_{i=0}^\infty a_ix^i
\end{equation*}
(where we substituted \(j\) for \(i-1\) to see explicitly how to change the limits of summation, a surprisingly useful trick) to rewrite the equation in terms of the power series \(\sum_{i=0}^\infty a_ix^i\).  Solve the resulting equation for the power series \(\sum_{i=0}^\infty a_ix^i\). You can save a lot of writing by using a variable like \(y\) to stand for the power series.%
\item\label{task-248} \hypertarget{p-1378}{}%
Use the previous part to get a formula for \(a_i\) in terms of \(a_0\).%
\item\label{task-249} \hypertarget{p-1380}{}%
Now suppose that \(a_i=3a_{i-1} + 2^i\).  Repeat the previous two steps for this recurrence relation.  (There is a way to do this part using what you already know.  Later on we shall introduce yet another way to deal with the kind of generating function that arises here.)%
~\hfill{\tiny\hyperlink{a-258.c}{[hint]}\hypertarget{q-258.c}{}}\end{enumerate}
\end{activity}
\end{document}
