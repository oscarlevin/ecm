\documentclass{book}

\input{../activities-preamble.tex}
\begin{document}
\setcounter{project}{258}
\addtocounter{project}{-1}
\begin{activity}[]\label{substituteandsolve}
\hypertarget{p-1350}{}%
Suppose that \(a_i=3a_{i-1} + 3^i\).%
\begin{enumerate}[font=\bfseries,label=(\alph*),ref=\alph*]
\item\label{task-246} \hypertarget{p-1351}{}%
Multiply both sides by \(x^i\) and sum both the left hand side and right hand side from \(i=1\) to infinity.  In the left-hand side use the fact that%
\begin{equation*}
\sum_{i=1}^\infty a_ix^i = (\sum_{i=0}^\infty x^i) -a_0
\end{equation*}
and in the right hand side, use the fact that%
\begin{equation*}
\sum_{i=1}^\infty a_{i-1}x^i = x\sum_{i=1}^\infty a_ix^{i-1}
=x\sum_{j=0}^\infty a_jx^j =x\sum_{i=0}^\infty a_ix^i
\end{equation*}
(where we substituted \(j\) for \(i-1\) to see explicitly how to change the limits of summation, a surprisingly useful trick) to rewrite the equation in terms of the power series \(\sum_{i=0}^\infty a_ix^i\).  Solve the resulting equation for the power series \(\sum_{i=0}^\infty a_ix^i\). You can save a lot of writing by using a variable like \(y\) to stand for the power series.%
\par\smallskip%
\noindent\textbf{Solution.}\hypertarget{solution-158}{}\quad%
\hypertarget{p-1352}{}%
%
\begin{align*}
\sum_{i=1}^\infty a_ix^i  =\amp 3\sum_{i=1}^\infty
a_{i-1}x^i+\sum_{i=1}^i3^ix^i\\
\sum_{i=1}^\infty a_ix^i =\amp 3x\sum_{i=1}^\infty a_{i-1}x^{i-1}+
\sum_{i=0}^\infty 3^ix^i-3^0x^0\\
\sum_{i=0}^\infty a_ix^i -a_0 =\amp 3x\sum_{i=0}^\infty a_{i}x^{i}+
\frac{1}{1-3x}-1\\
(1-3x)\sum_{i=0}^\infty a_ix^i  =\amp a_0+\frac{1}{1-3x} -1\\
\sum_{i=0}^\infty a_ix^i  =\amp \frac{a_0-1}{1-3x}+\frac{1}{(1-3x)^2}
\end{align*}
%
\item\label{task-247} \hypertarget{p-1353}{}%
Use the previous part to get a formula for \(a_i\) in terms of \(a_0\).%
\par\smallskip%
\noindent\textbf{Solution.}\hypertarget{solution-159}{}\quad%
\hypertarget{p-1354}{}%
%
\begin{align*}
\sum_{i=0}^\infty a_i x^i  =\amp \frac{a_0-1}{1-3x} + \frac{1}{(1-3x)^2}\\
=\amp (a_0-1)\sum_{i=0}^\infty 3^ix^i +\sum_{i=0}^\infty \binom{i+1}{i}3^ix^i,
\end{align*}
which gives us \(a_n=(a_0-1) 3^i + (i+1)3^i=(a_0+i)3^i\).%
\item\label{task-248} \hypertarget{p-1355}{}%
Now suppose that \(a_i=3a_{i-1} + 2^i\).  Repeat the previous two steps for this recurrence relation.  (There is a way to do this part using what you already know.  Later on we shall introduce yet another way to deal with the kind of generating function that arises here.)%
\par\smallskip%
\noindent\textbf{Hint.}\hypertarget{hint-167}{}\quad%
\hypertarget{p-1356}{}%
You may run into a product of the form \(\sum_{i=0}^\infty a^ix^i\sum_{j=0}^\infty b^jx^j\). Note that in the product, the coefficient of \(x^k\) is \(\sum_{i=0}^k a^ib^{k-i} = \sum_{i=0}^k \frac{a^i}{b^i}\).%
~\hfill{\tiny\hyperlink{a-258.c}{[hint]}\hypertarget{q-258.c}{}}\par\smallskip%
\noindent\textbf{Solution.}\hypertarget{solution-160}{}\quad%
\hypertarget{p-1357}{}%
%
\begin{align*}
\sum_{i=1}^\infty a_ix^i =\amp 3\sum_{i=1}^\infty
a_{i-1}x^i +\sum_{i=1}^\infty 2^ix^i\\
\sum_{i=0}^\infty a_ix^i -a_0  =\amp 3x\sum_{i=1}^\infty a_{i-1}x^{i-1}
+\sum_{i=0}^\infty (2x)^i -1\\
\sum_{i=0}^\infty a_ix^i -a_0  =\amp 3x\sum_{i=0}^\infty a_ix^i +\frac{1}{1-2x} -1\\
(1-3x)\sum_{i=0}^\infty a_ix^i =\amp a_0 +\frac{1}{1-2x}-1\\
\sum_{i=0}^\infty a_ix^i =\amp  \frac{a_0-1}{1-3x}+\frac{1}{(1-2x)(1-3x)}\\
\sum_{i=0}^\infty a_ix^i =\amp (a_0-1)\sum_{i=0}^\infty 3^ix^i +\sum_{i=0}^\infty
2^ix^i\sum_{j=0}^\infty 3^jx^j
\end{align*}
But%
\begin{align*}
\sum_{i=0}^\infty
2^ix^i\sum_{j=0}^\infty 3^jx^j\amp=
\sum_{k=0}^\infty
\sum_{i=0}^k 2^i3^{k-i}x^k=
\sum_{k=0}^\infty3^kx^k
\sum_{i=0}^k \frac{2^i}{3^i}\\
\amp=\sum_{k=0}^\infty \frac{1-\left(\frac{2}{3}\right)^{k+1}}{1-\frac{2}{3}}3^kx^k =\sum_{k=0}^\infty
(3^{k+1}-2^{k+1})x^k\text{.}
\end{align*}
Substituting this into the equation for \(\sum_{i=0}^\infty a_ix^i\) gives us \(a_i =(a_0+2)3^i -2^{i+1}\).%
\end{enumerate}
\end{activity}

\clearpage\end{document}
