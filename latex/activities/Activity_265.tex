\documentclass{book}

\input{../activities-preamble.tex}
\begin{document}
\setcounter{project}{265}
\addtocounter{project}{-1}
\begin{activity}[]\label{partialfractions1}
\hypertarget{p-1400}{}%
In \hyperref[partialfractionsintro]{Activity~\ref{partialfractionsintro}} you may have simply guessed at values of \(c\) and \(d\), or you may have solved a system of equations in the two unknowns \(c\) and \(d\). Given constants \(a\), \(b\), \(r_1\), and \(r_2\) (with \(r_1\not= r_2\)), write down a system of equations we can solve for \(c\) and \(d\) to write%
\begin{equation*}
\frac{ax+b}{(x-r_1)(x-r_2)} = \frac{c}{x-r_1} + \frac{d}{x-r_2}\text{.}
\end{equation*}
%
\par\smallskip%
\noindent\textbf{Hint.}\hypertarget{hint-170}{}\quad%
\hypertarget{p-1401}{}%
To have%
\begin{equation*}
\frac{ax+b}{(x-r_1)(x-r_2)} = \frac{c}{x-r_1} + \frac{d}{x-r_2}
\end{equation*}
we must have%
\begin{equation*}
cx-r_2c+dx-r_1d =ax+b\text{.}
\end{equation*}
%
~\hfill{\tiny\hyperlink{a-265}{[hint]}\hypertarget{q-265}{}}\par\smallskip%
\noindent\textbf{Solution.}\hypertarget{solution-187}{}\quad%
\hypertarget{p-1402}{}%
To have%
\begin{equation*}
\frac{ax+b}{(x-r_1)(x-r_2)} =  \frac{c}{x-r_1} + \frac{d}{x-r_2}
\end{equation*}
we must have%
\begin{equation*}
cx-r_2c+dx-r_2d = ax+b.
\end{equation*}
This gives us the equations \(cx+dx=ax\) and \(-r_2c-r_1d=b\). Since \(x\) can be any value, in particular it can be nonzero, so we can divide by it. This gives us the equations \(c+d=a\) and \(r_2c+r_1d=-b\).%
\end{activity}
\end{document}
