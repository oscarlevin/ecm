\documentclass{book}

\input{../activities-preamble.tex}
\begin{document}
\setcounter{project}{91}
\addtocounter{project}{-1}
\begin{activity}[]\label{activity-84}
\hypertarget{p-674}{}%
Suppose you have 17 books strewn around your room.  You resolve to get organized.%
\begin{enumerate}[font=\bfseries,label=(\alph*),ref=\alph*]
\item\label{task-135} \hypertarget{p-675}{}%
You decide to throw 10 books in the trash.  How many choices do you have for which books to get rid of?%
\par\smallskip%
\noindent\textbf{Hint.}\hypertarget{hint-45}{}\quad%
\hypertarget{p-676}{}%
We are not necessarily looking for a numerical answer here, but you should be able to express the answer using notation we have developed already.  Or by using some sort of triangle.%
~\hfill{\tiny\hyperlink{a-91.a}{[hint]}\hypertarget{q-91.a}{}}\item\label{task-136} \hypertarget{p-677}{}%
You think twice about tossing your books, and you find a bookshelf that can hold 10 books.  How many ways can you fill the bookshelf?%
\item\label{task-137} \hypertarget{p-678}{}%
How many ways could you fill the 10 book bookshelf if the books were arranged alphabetically?%
\par\smallskip%
\noindent\textbf{Hint.}\hypertarget{hint-46}{}\quad%
\hypertarget{p-679}{}%
Should your answer be larger or smaller than if the books could be in any order?  How is this any different from throwing the books away?%
~\hfill{\tiny\hyperlink{a-91.c}{[hint]}\hypertarget{q-91.c}{}}\end{enumerate}
\end{activity}

\clearpage\end{document}
