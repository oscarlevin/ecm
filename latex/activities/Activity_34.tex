\documentclass{book}

\input{../activities-preamble.tex}
\begin{document}
\setcounter{project}{34}
\addtocounter{project}{-1}
\begin{activity}[]\label{activity-29}
\hypertarget{p-324}{}%
Consider some classic polyhedra.%
\begin{enumerate}[font=\bfseries,label=(\alph*),ref=\alph*]
\item\label{task-43} \hypertarget{p-325}{}%
An \emph{octahedron} is a regular polyhedron made up of 8 equilateral triangles (it sort of looks like two pyramids with their bases glued together). Draw a planar graph representation of an octahedron. How many vertices, edges and faces does an octahedron (and your graph) have?%
\par\smallskip%
\noindent\textbf{Solution.}\hypertarget{solution-26}{}\quad%
\hypertarget{p-326}{}%
Since there are 8 triangles, there must be 8 faces. We can count the number of edges by taking \(8 \cdot 3 = 24\), but this is double counting since each edge corresponds to two faces. Thus there are 12 edges. We can use Euler's formula to find that there are 6 vertices (and this shows that each vertex is the joining of 4 triangles).%
\par
\hypertarget{p-327}{}%
The planar representation of the graph is:%
\item\label{task-44} \hypertarget{p-328}{}%
The traditional design of a soccer ball is in fact a (spherical projection of a) truncated icosahedron. This consists of 12 regular pentagons and 20 regular hexagons. No two pentagons are adjacent (so the edges of each pentagon are shared only by hexagons). How many vertices, edges, and faces does a truncated icosahedron have? Explain how you arrived at your answers. Bonus: draw the planar graph representation of the truncated icosahedron.%
\par\smallskip%
\noindent\textbf{Solution.}\hypertarget{solution-27}{}\quad%
\hypertarget{p-329}{}%
Well, right off we know that the truncated icosahedron has \(12+20=32\) faces by counting the number of pentagons and hexagons. Now, because we know that every connected planar graph with \(V\) vertices, \(E\) edges and \(F\) faces satisfied \(V - E + F = 2\), we only really need to find out the number of edges or the number of vertices since \(V-E=-30\). So, let's maybe try to figure out the number of edges we have. If we think about the number of total edges when the pentagons and hexagons are not attached, we know that we have \(5\times 12+6\times 20=180\). But each of these edges is shared with another edge, which means that we have cut the number of edges in half. So, we have \(90\) edges, which then gives us \(60\) vertices.%
\item\label{task-45} \hypertarget{p-330}{}%
Your ``friend'' claims that he has constructed a convex polyhedron out of 2 triangles, 2 squares, 6 pentagons and 5 octagons. Prove that your friend is lying. Hint: each vertex of a convex polyhedron must border at least three faces.%
\par\smallskip%
\noindent\textbf{Solution.}\hypertarget{solution-28}{}\quad%
\hypertarget{p-331}{}%
So, let's assume for a contradiction that your friend really has constructed a convex polyhedron. Then, we would know that the polyhedron has \(15\) faces, \((2\times 3+2\times 4+6\times 5+5\times 8)/2 = 42\) edges, and \(V=2+42-15=29\) vertices. Now, using the hint, we also know that \(3V\leq F\) but \(3\times 29\) is not in fact less than \(15\). So we have a contradiction, and your friend is lying.%
\end{enumerate}
\end{activity}

\clearpage\end{document}
