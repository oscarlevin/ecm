\documentclass{book}

\input{../activities-preamble.tex}
\begin{document}
\setcounter{project}{224}
\addtocounter{project}{-1}
\begin{activity}[]\label{compunion}
\hypertarget{p-1243}{}%
Frequently when we apply the principle of inclusion and exclusion, we will have a situation like that of part (d) of \hyperref[hatcheckprobpart]{Task~\ref{hatcheck}.\ref{hatcheckprobpart}}.  That is, we will have a set \(A\) and subsets \(A_1, A_2, \ldots, A_n\) and we will want the size or the probability of the set of elements in \(A\) that are \emph{not} in the union.  This set is known as the \terminology{complement} \index{complement} of the union of the \(A_i\)s in \(A\), and is denoted by \(A \setminus \bigcup_{i=1}^n A_i\), or if \(A\) is clear from context, by \(\overline{\bigcup_{i=1}^n A_i}\). Give the fomula for \(\overline{\bigcup_{i=1}^n A_i}\).  The principle of inclusion and exclusion generall refers to both this formula and the one for the union.%
\end{activity}
\end{document}
