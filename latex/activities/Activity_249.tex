\documentclass{book}

\input{../activities-preamble.tex}
\begin{document}
\setcounter{cpjt}{249}
\addtocounter{cpjt}{-1}
\begin{activity}\label{ProductPrincipleOGF}
\hypertarget{p-1298}{}%
Suppose that we have two sets \(S_1\) and \(S_2\). Let \(v_1\) (\(v\) stands for value) be a function from \(S_1\) to the nonnegative integers and let \(v_2\) be a function from \(S_2\) to the nonnegative integers.  Define a new function \(v\) on the set \(S_1 \times S_2\) by \(v(x_1,x_2) = v_1(x_1) +v_2(x_2)\). Suppose further that \(\sum_{i=0}^\infty a_ix^i\) is the generating function for the number of elements \(x_1\) of \(S_1\) of value \(i\), that is with \(v_1(x_1)=i\). Suppose also that \(\sum_{j=0}^\infty b_j x^j\) is the generating function for the number of elements of \(x_2\) of \(S_2\) of value \(j\), that is with \(v_2(x_2) = j\).  Prove that the coefficient of \(x^k\) in%
\begin{equation*}
\left(\sum_{i=0}^\infty a_ix^i\right)\left(\sum_{j=0}^\infty
b_jx^j\right)
\end{equation*}
is the number of ordered pairs \((x_1,x_2)\) in \(S_1\times S_2\) with total value \(k\), that is with \(v_1(x_1) +v_2(x_2) =k\). This is called the \terminology{product principle for generating functions}.\index{product principle for generating functions}\index{generating function!product principle for}%
\par\smallskip%
\noindent\textbf{Hint}.\hypertarget{hint-158}{}\quad%
\hypertarget{p-1299}{}%
If this problem appears difficult, the most likely reason is because the definitions are all new and symbolic. Focus on what it means for \(\sum_{k=0}^\infty c_kx^k\) to be the generating function for ordered pairs of total value \(k\). In particular, how do we get an ordered pair with total value \(k\)? What do we need to know about the values of the components of the ordered pair?%
\par\smallskip%
\noindent\end{activity}

\clearpage\end{document}
