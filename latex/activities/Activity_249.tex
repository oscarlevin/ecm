\documentclass{book}

\input{../activities-preamble.tex}
\begin{document}
\setcounter{project}{249}
\addtocounter{project}{-1}
\begin{activity}[]\label{ProductPrincipleOGF}
\hypertarget{p-1350}{}%
Suppose that we have two sets \(S_1\) and \(S_2\). Let \(v_1\) (\(v\) stands for value) be a function from \(S_1\) to the nonnegative integers and let \(v_2\) be a function from \(S_2\) to the nonnegative integers.  Define a new function \(v\) on the set \(S_1 \times S_2\) by \(v(x_1,x_2) = v_1(x_1) +v_2(x_2)\). Suppose further that \(\sum_{i=0}^\infty a_ix^i\) is the generating function for the number of elements \(x_1\) of \(S_1\) of value \(i\), that is with \(v_1(x_1)=i\). Suppose also that \(\sum_{j=0}^\infty b_j x^j\) is the generating function for the number of elements of \(x_2\) of \(S_2\) of value \(j\), that is with \(v_2(x_2) = j\).  Prove that the coefficient of \(x^k\) in%
\begin{equation*}
\left(\sum_{i=0}^\infty a_ix^i\right)\left(\sum_{j=0}^\infty
b_jx^j\right)
\end{equation*}
is the number of ordered pairs \((x_1,x_2)\) in \(S_1\times S_2\) with total value \(k\), that is with \(v_1(x_1) +v_2(x_2) =k\). This is called the \terminology{product principle for generating functions}.\index{product principle for generating functions}\index{generating function!product principle for}%
~\hfill{\tiny\hyperlink{a-249}{[hint]}\hypertarget{q-249}{}}\end{activity}
\end{document}
