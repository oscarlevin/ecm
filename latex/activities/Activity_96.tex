\documentclass{book}

\input{../activities-preamble.tex}
\begin{document}
\setcounter{project}{96}
\addtocounter{project}{-1}
\begin{activity}[]\label{activity-89}
\hypertarget{p-725}{}%
We have already proved this identity: \(\binom{n}{0} + \binom{n}{1} + \binom{n}{2} + \ldots + \binom{n}{n} = 2^{n}\).  Now try giving a combinatorial proof that uses lattice paths.%
\par\smallskip%
\noindent\textbf{Hint.}\hypertarget{hint-50}{}\quad%
\hypertarget{p-726}{}%
You will want to consider a lot of lattice paths, not all ending at the same point.  In fact, what do all the lattice paths have in common?%
~\hfill{\tiny\hyperlink{a-96}{[hint]}\hypertarget{q-96}{}}\par\smallskip%
\noindent\textbf{Solution.}\hypertarget{solution-70}{}\quad%
\hypertarget{p-727}{}%
The question is, how many lattice paths start at (0,0) and end on the line \(x+y=n\) (in the first quadrant)?%
\end{activity}
\end{document}
