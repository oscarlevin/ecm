\documentclass{book}

\input{../activities-preamble.tex}
\begin{document}
\setcounter{project}{124}
\addtocounter{project}{-1}
\begin{activity}[]\label{bookcase}
\hypertarget{p-871}{}%
Suppose we wish to place \(k\) distinct books onto the shelves of a bookcase with \(n\) shelves. For simplicity, assume for now that all of the books would fit on any of the shelves. Also, let's imagine pushing the books on a shelf as far to the left as we can, so that we are only thinking about how the books sit relative to each other, not about the exact places where we put the books. Since the books are distinct, we can think of the first book, the second book and so on.%
\begin{enumerate}[font=\bfseries,label=(\alph*),ref=\alph*]
\item\label{task-164} \hypertarget{p-872}{}%
How many places are there where we can place the first book?%
\par\smallskip%
\noindent\textbf{Solution.}\hypertarget{solution-90}{}\quad%
\hypertarget{p-873}{}%
There are \(n\) places where we can place the first book.%
\item\label{task-165} \hypertarget{p-874}{}%
When we place the second book, if we decide to place it on the shelf that already has a book, does it matter if we place it to the left or right of the book that is already there?%
\par\smallskip%
\noindent\textbf{Solution.}\hypertarget{solution-91}{}\quad%
\hypertarget{p-875}{}%
Yes.%
\item\label{task-166} \hypertarget{p-876}{}%
How many places are there where we can place the second book?%
~\hfill{\tiny\hyperlink{a-124.c}{[hint]}\hypertarget{q-124.c}{}}\par\smallskip%
\noindent\textbf{Solution.}\hypertarget{solution-92}{}\quad%
\hypertarget{p-878}{}%
Once we have placed it, there are \(n+1\) places where we can place the second book, because on the shelf that has one book, we could put the second book to the left or to the right of the book already there.%
\item\label{task-167} \hypertarget{p-879}{}%
Once we have \(i-1\) books placed, if we want to place book \(i\)  on a shelf that already has some books, is sliding it in to the left of all the books already there different from placing it to the right of all the books already or between two books already there?%
\par\smallskip%
\noindent\textbf{Solution.}\hypertarget{solution-93}{}\quad%
\hypertarget{p-880}{}%
All of these are different.%
\item\label{task-168} \hypertarget{p-881}{}%
In how many ways may we place the \(i\)th book into the bookcase?%
~\hfill{\tiny\hyperlink{a-124.e}{[hint]}\hypertarget{q-124.e}{}}\par\smallskip%
\noindent\textbf{Solution.}\hypertarget{solution-94}{}\quad%
\hypertarget{p-883}{}%
Once we have \(i-1\) books on the shelves the \(i\)th book could go on any shelf to the left of all books there, if any, giving us \(n\) places, or it could go to the immediate right of any book already there, giving us another \(i-1\) places. Thus there are \(n+i-1\) places where we could place book  \(i\).%
\item\label{task-169} \hypertarget{p-884}{}%
In how many ways may we place all the books?%
\par\smallskip%
\noindent\textbf{Solution.}\hypertarget{solution-95}{}\quad%
\hypertarget{p-885}{}%
From this, we can see that the number of ways to place all the books is%
\begin{equation*}
\prod_{i=1}^k (n+i-1) = (n)(n+1)(n+2)\cdots(n_k-1).
\end{equation*}
%
\end{enumerate}
\end{activity}
\end{document}
