\documentclass{book}

\input{../activities-preamble.tex}
\begin{document}
\setcounter{project}{313}
\addtocounter{project}{-1}
\begin{activity}[]\label{activity-306}
\hypertarget{p-1564}{}%
Find a recurrence that expresses \(q_n(k)\) as a sum of \(q_m(k-n)\) for appropriate values of \(m\).%
\par\smallskip%
\noindent\textbf{Hint.}\hypertarget{hint-201}{}\quad%
\hypertarget{p-1565}{}%
What can you do to a Young diagram for a partition of \(k\) into \(n\) distinct parts to get a Young diagram of a partition of \(k-n\) into some number of distinct parts?%
~\hfill{\tiny\hyperlink{a-313}{[hint]}\hypertarget{q-313}{}}\par\smallskip%
\noindent\textbf{Solution.}\hypertarget{solution-206}{}\quad%
\hypertarget{p-1566}{}%
Suppose \(\lambda\) is a partition of \(k\) into \(n\) distinct parts. Either 1 is one of those parts or not. Thus if we subtract 1 from each part, we either get a partition of \(k-n\) into \(n-1\) parts or a partition of \(k-n\) into \(n\) parts. If \(\lambda\) and \(\lambda'\) are different partitions of \(k\) into \(n\) distinct parts, they go to different partitions. Each partition of \(k-n\) into \(n-1\) parts or \(n\) parts can be gotten in this way from a corresponding partition of \(k\) into \(n\) parts. Thus we have a bijective correspondence and \(q_n(k)=q_{n-1}(k-n) + q_{n}(k-n)\).%
\end{activity}

\clearpage\end{document}
