\documentclass{book}

\input{../activities-preamble.tex}
\begin{document}
\setcounter{project}{42}
\addtocounter{project}{-1}
\begin{activity}[]\label{activity-35}
\leavevmode%
\begin{enumerate}[font=\bfseries,label=(\alph*),ref=\alph*]
\item\label{task-54} \hypertarget{p-388}{}%
Find the chromatic number of the graph below.  How do you know the chromatic number is not larger or smaller than your answer?%
\begin{sidebyside}{1}{0.375}{0.375}{0}
\begin{sbspanel}{0.25}
\resizebox{\linewidth}{!}{{
\begin{tikzpicture}[scale=.9]
	\coordinate (A) at (90:2);
	\coordinate (C) at (90-36:2);
	\coordinate (B) at (90-2*36:2);
	\coordinate (D) at (90-3*36:2);
	\coordinate (E) at (90-4*36:2);
	\coordinate (F) at (90-5*36:2);
	\coordinate (G) at (90-6*36:2);
	\coordinate (I) at (90-7*36:2);
	\coordinate (H) at (90-8*36:2);
	\coordinate (J) at (90-9*36:2);

	\draw (A) -- (F) -- (D) -- (H) -- (I) (G) -- (J) -- (C) -- (F) (C) -- (G) -- (A) -- (J);
	\draw (A) -- (C) -- (F) -- (G) -- (J) -- (A) -- (F) -- (J) -- (C) -- (G) -- (A);
	\draw (A) \v -- (C) \v -- (B) \v -- (D) \v -- (E) \v -- (I) \v -- (J) \v -- (F) \v -- (G) \v (H) \v;
	\end{tikzpicture}
}
}
\end{sbspanel}
\end{sidebyside}
\par\smallskip%
\noindent\textbf{Hint.}\hypertarget{hint-15}{}\quad%
\hypertarget{p-389}{}%
If you find a proper \(5\)-coloring, could the chromatic number be more than \(5\)?  Could it be less than \(5\)?  What if you found a subgraph that you were sure had chromatic number \(5\)?%
~\hfill{\tiny\hyperlink{a-42.a}{[hint]}\hypertarget{q-42.a}{}}\item\label{task-55} \hypertarget{p-390}{}%
For any graph \(G\), if \(H\) is a subgraph with chromatic number \(k\), what can you say about the chromatic number of \(G\)?%
\item\label{task-56} \hypertarget{p-391}{}%
Define the \terminology{clique number} of a graph to be the largest \(n\) for which the graph contains a copy of \(K_n\) as a subgraph (a \terminology{clique} in a graph is a set of vertices that are pairwise adjacent, so a copy of \(K_n\) for some \(n\)).  State and prove a theorem relating the clique number and the chromatic number of a graph.%
\end{enumerate}
\end{activity}

\clearpage\end{document}
