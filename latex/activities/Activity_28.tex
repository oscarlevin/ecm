\documentclass{book}

\input{../activities-preamble.tex}
\begin{document}
\setcounter{cpjt}{28}
\addtocounter{cpjt}{-1}
\begin{activity}\label{activity-23}
\hypertarget{p-278}{}%
The other simplest graph which is not planar is \(K_{3,3}\)%
\begin{sidebyside}{1}{0.4}{0.4}{0}
\begin{sbspanel}{0.2}
\resizebox{\linewidth}{!}{{
\begin{tikzpicture}[yscale=1.2]
              \draw (-1,1) \v -- (-1,0)\v  -- (0,1) \v -- (0,0) \v -- (1,1) \v -- (1,0) \v -- (0,1) -- (-1,0) -- (1,1) (1,0) -- (-1,1) -- (0,0);
              \end{tikzpicture}
}
}
\end{sbspanel}
\end{sidebyside}
\begin{enumerate}[font=\bfseries,label=(\alph*),ref=\alph*]
\item\label{task-37} \hypertarget{p-279}{}%
Following the same proof outline as you used for \(K_5\), what value do you find for \(f\) and what can you conclude from the inequality \(3f \le 2e\)?%
\par\smallskip%
\noindent\textbf{Hint}.\hypertarget{hint-8}{}\quad%
\hypertarget{p-280}{}%
If you assume \(P\) and conclude \(Q\), but \(Q\) is true, can you say anything about \(P\)?  What row(s) of the truth table for \(P \imp Q\) are you in?%
\item\label{task-38} \hypertarget{p-281}{}%
The 3 in \(3f \le 2e\) came from the observation that the any face in \(K_5\) must be bounded by at least three edges.  Is that the best we can do for a bipartite graph?  Find and justify an improved inequality between \(f\) and \(e\).%
\end{enumerate}
\end{activity}

\clearpage\end{document}
