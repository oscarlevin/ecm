\documentclass{book}

\input{../activities-preamble.tex}
\begin{document}
\setcounter{project}{82}
\addtocounter{project}{-1}
\begin{activity}[]\label{activity-75}
\leavevmode%
\begin{enumerate}[font=\bfseries,label=(\alph*),ref=\alph*]
\item\label{task-121} \hypertarget{p-639}{}%
Explain why \hyperref[prop-prodcard]{Proposition~\ref{prop-prodcard}} is true.  Your explanation should make use of \hyperref[prop-unioncard]{Proposition~\ref{prop-unioncard}}.%
\item\label{task-122} \hypertarget{p-640}{}%
Using the same incomplete deck of cards (containing the Jack, King and Queen of diamonds, King of hearts, and 2 through 6 of spades), the magician asks you to put any face card face down under any face down black card.  How many choices do you have for this two card stack?%
\item\label{task-123} \hypertarget{p-641}{}%
How exactly does the counting question above relate to \hyperref[prop-prodcard]{Proposition~\ref{prop-prodcard}}?  How exactly does it relate to the Product Principle?  Illustrate both by listing specific outcomes.%
\item\label{task-124} \hypertarget{p-642}{}%
Make explicit the connection between the Product Principle and \hyperref[prop-prodcard]{Proposition~\ref{prop-prodcard}}.  That is, justify the Product Principle in terms of cardinality of sets.%
\end{enumerate}
\end{activity}
\end{document}
