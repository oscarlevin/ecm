\documentclass{book}

\input{../activities-preamble.tex}
\begin{document}
\setcounter{project}{247}
\addtocounter{project}{-1}
\begin{activity}[]\label{coeffinproduct2}
\hypertarget{p-1306}{}%
The point of the \hyperref[coeffinproduct]{Problems~\ref{coeffinproduct}} and \hyperref[coeffinproduct1]{Activity~\ref{coeffinproduct1}} is that so long as we are willing to assume \(a_i=0\) for \(i>n\) and \(b_j =0\) for \(j>m\), then there is a very nice formula for the coefficient of \(x^k\) in the product%
\begin{equation*}
\left(\sum_{i=0}^n a_ix^i\right)\left(\sum_{j=0}^m b_jx^j\right).
\end{equation*}
Write down this formula explicitly.%
\par\smallskip%
\noindent\textbf{Hint.}\hypertarget{hint-156}{}\quad%
\hypertarget{p-1307}{}%
Write down the formulas for the coefficients of \(x^0\), \(x^1\), \(x^2\) and \(x^3\) in%
\begin{equation*}
\left(\sum_{i=0}^n a_ix^i\right)\left(\sum_{j=0}^m b_jx^j\right)\text{.}
\end{equation*}
%
~\hfill{\tiny\hyperlink{a-247}{[hint]}\hypertarget{q-247}{}}\par\smallskip%
\noindent\textbf{Solution.}\hypertarget{solution-147}{}\quad%
\hypertarget{p-1308}{}%
\(\sum_{i=0}^k a_ib_{k-i}\).%
\end{activity}

\clearpage\end{document}
