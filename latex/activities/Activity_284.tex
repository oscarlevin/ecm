\documentclass{book}

\input{../activities-preamble.tex}
\begin{document}
\setcounter{project}{284}
\addtocounter{project}{-1}
\begin{activity}[]\label{activity-277}
\hypertarget{p-1489}{}%
What do multinomial coefficients have to do with expanding the \(k\)th power of a multinomial \(x_1+x_2+\cdots+x_n\)? This result is called the \terminology{multinomial theorem}.%
~\hfill{\tiny\hyperlink{a-284}{[hint]}\hypertarget{q-284}{}}\par\smallskip%
\noindent\textbf{Solution.}\hypertarget{solution-213}{}\quad%
\hypertarget{p-1491}{}%
When we use the distributive law to multiply out \((x_1+x_2+\cdots +x_n)^k\), we will get a sum of a bunch of terms of the form \(x_1^{i_1}x_2^{i_2}\cdots x_n^{i_n}\) where \(i_1+i_2+\cdots+ i_n=k\).  The terms with a given sequence \(i_1,i_2,\ldots, i_n\) of exponents will arise from choosing, as we apply the distributive law over and over again, \(x_1\) from \(i_1\) of the factors, \(x_2\) from \(i_2\) of the factors, and so on. Thus the number of terms \(x_1^{i_1}x_2^{i_2}\cdots x_n^{i_n}\) will be the number of ways to label \(i_1\) of the factors with a 1, \(i_2\) of the factors with a 2, \textellipsis{}, and \(i_n\) of the factors with an \(n\). The number of ways to do this is a multinomial coefficient, as we now explain. This labeling gives us a function from \([k]\) to \([n]\) as follows. If factor \(i\) is labelled \(j\) we let \(f(i) =j\). Further each function \(f\) from \([k]\) to \([n]\) gives us that maps \(i_j\) elements of \([k]\) to \(j\) will give us such a labelling. Thus the coefficient of \(x_1^{i_1}x_2^{i_2}\cdots x_n^{i_n}\) will be the multinomial coefficient \(\binom{k}{i_1,i_2,\ldots, i_n}\).%
\end{activity}
\end{document}
