\documentclass{book}

\input{../activities-preamble.tex}
\begin{document}
\setcounter{project}{253}
\addtocounter{project}{-1}
\begin{activity}[]\label{negnchoosek}
\hypertarget{p-1382}{}%
Find a formula for \((1+x)^{-n}\) as a power series whose coefficients involve binomial coefficients. What does this formula tell you about how we should define \(\binom{-n}{k}\) when \(n\) is positive?%
~\hfill{\tiny\hyperlink{a-253}{[hint]}\hypertarget{q-253}{}}\par\smallskip%
\noindent\textbf{Solution.}\hypertarget{solution-189}{}\quad%
\hypertarget{p-1385}{}%
%
\begin{equation*}
(1+x)^{-n}=(1-(-x))^{-n}=\sum_{i=0}^\infty
\binom{n+i-1}{i}(-x)^i=\sum_{i=0}^\infty (-1)^i\binom{n+i-1}{i}x^i.
\end{equation*}
We want the coefficient of \(x^k\) in \((1+x)^{-n}\) to be \(\binom{-n}{k}\), so we want \(\binom{-n}{k}= (-1)^k\binom{n+k-1}{k}\).%
\end{activity}
\end{document}
