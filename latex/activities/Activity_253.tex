\documentclass{book}

\input{../activities-preamble.tex}
\begin{document}
\setcounter{cpjt}{253}
\addtocounter{cpjt}{-1}
\begin{activity}\label{negnchoosek}
\hypertarget{p-1316}{}%
Find a formula for \((1+x)^{-n}\) as a power series whose coefficients involve binomial coefficients. What does this formula tell you about how we should define \(\binom{-n}{k}\) when \(n\) is positive?%
\par\smallskip%
\noindent\textbf{Hint 1}.\hypertarget{hint-161}{}\quad%
\hypertarget{p-1317}{}%
While you could use calculus techniques, there is a much simpler approach. Note that \(1 + x = 1 - (-x)\).%
\par\smallskip%
\noindent\textbf{Hint 2}.\hypertarget{hint-162}{}\quad%
\hypertarget{p-1318}{}%
Can you see a way to use \hyperref[multisetgenfn]{Activity~\ref{multisetgenfn}}?%
\par\smallskip%
\noindent\end{activity}

\clearpage\end{document}
