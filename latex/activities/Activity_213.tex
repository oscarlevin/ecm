\documentclass{book}

\input{../activities-preamble.tex}
\begin{document}
\setcounter{project}{213}
\addtocounter{project}{-1}
\begin{activity}[]\label{activity-206}
\hypertarget{p-1163}{}%
How does the number of partitions of \(k\) relate to the number of partitions of \(k+1\) whose smallest part is one?%
\par\smallskip%
\noindent\textbf{Hint.}\hypertarget{hint-133}{}\quad%
\hypertarget{p-1164}{}%
How can you start with a partition of \(k\) and make it into a new partition of \(k+1\) that is guaranteed to have a part of size one, even if the original partition didn't?%
~\hfill{\tiny\hyperlink{a-213}{[hint]}\hypertarget{q-213}{}}\par\smallskip%
\noindent\textbf{Solution.}\hypertarget{solution-114}{}\quad%
\hypertarget{p-1165}{}%
They are equal, because if we take two different partitions of \(k-1\) and increase the multiplicity of 1 in each (by one), they are still different; also if we take two different partitions of \(k\) that have parts of size one, and decrease the multiplicity of 1 in each (by one), they are still different.%
\end{activity}
\end{document}
