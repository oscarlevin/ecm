\documentclass{book}

\input{../activities-preamble.tex}
\begin{document}
\setcounter{project}{177}
\addtocounter{project}{-1}
\begin{activity}[]\label{act-rootedtrees}
\hypertarget{p-1054}{}%
Consider \terminology{rooted binary trees}.  Rooted trees are trees in the graph theory sense, except that one vertex is designated as the root, which puts a natural ordering on the vertices.  Vertices adjacent to the root are its \terminology{children}, and vertices adjacent to those (other than the root) are their children, and so on.  We are looking at \emph{binary} trees, so each vertex will have either two children (designated the \terminology{left child} and \terminology{right child}) or no children at all (i.e., the vertex is a leaf).%
\par
\hypertarget{p-1055}{}%
How many rooted binary trees have exactly \(n+1\) leaves?  Note that because we designate children as left or right, the two trees below are counted as distinct.%
\begin{sidebyside}{2}{0.125}{0.125}{0.25}
\begin{sbspanel}{0.25}
\resizebox{\linewidth}{!}{{
\begin{tikzpicture}
  \coordinate (r) at (0,0);
  \coordinate (rl) at (-1,1);
  \coordinate (rr) at (1,1);
  \coordinate (rll) at (-1.5,2);
  \coordinate (rlr) at (-.5, 2);
  \draw (r) \v -- (rl) \v -- (rll) \v (rl) -- (rlr) \v (r) -- (rr) \v;
\end{tikzpicture}
}
}
\end{sbspanel}
\begin{sbspanel}{0.25}
\resizebox{\linewidth}{!}{{
\begin{tikzpicture}
  \coordinate (r) at (0,0);
  \coordinate (rl) at (-1,1);
  \coordinate (rr) at (1,1);
  \coordinate (rll) at (-1.5,2);
  \coordinate (rlr) at (-.5, 2);
  \coordinate (rrl) at (.5,2);
  \coordinate (rrr) at (1.5, 2);
  \draw (r) \v -- (rr) \v -- (rrl) \v (rr) -- (rrr) \v (r) -- (rl) \v;
\end{tikzpicture}
}
}
\end{sbspanel}
\end{sidebyside}
~\hfill{\tiny\hyperlink{a-177}{[hint]}\hypertarget{q-177}{}}\end{activity}
\end{document}
