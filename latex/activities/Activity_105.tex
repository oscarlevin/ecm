\documentclass{book}

\input{../activities-preamble.tex}
\begin{document}
\setcounter{project}{105}
\addtocounter{project}{-1}
\begin{activity}[]\label{activity-98}
\hypertarget{p-740}{}%
\(1 \cdot 1! + 2 \cdot 2! + 3 \cdot 3! + \ldots + n \cdot n! = \left( n + 1 \right)! - 1\).%
\par\smallskip%
\noindent\textbf{Hint.}\hypertarget{hint-59}{}\quad%
\hypertarget{p-741}{}%
In how many ways can you arrange the \(n+1\) numbers \(0, 1, 2, \ldots, n\) so that they are \emph{not} in ascending order?%
~\hfill{\tiny\hyperlink{a-105}{[hint]}\hypertarget{q-105}{}}\par\smallskip%
\noindent\textbf{Solution.}\hypertarget{solution-60}{}\quad%
\hypertarget{p-742}{}%
In how many ways can you arrange the \(n+1\) numbers \(0, 1, 2, \ldots, n\) so that they are \emph{not} in ascending order? The answer is \(\left( n + 1 \right)! - 1\) since \(0, 1, 2, \ldots, n\) is the \emph{only} arrangement in ascending order. Now lets separate into cases. Let \(a_{0},a_{1},a_{2},\ldots,a_{n}\) represent an arrangement of these \(n+1\) numbers. If \(a_{0} \neq 0\), there are \(n\) choices left for \(a_{0}\), and then \(n!\) ways to fill out \(a_{1},a_{2},\ldots,a_{n}\) for a total of \(n \cdot n!\). Now let \(a_{0} = 0\) but \(a_{1} \neq 1\). There are \(n - 1\) choices for \(a_{1}\) and \(\left(n - 1 \right)!\) ways to complete for a total of \(\left(n - 1 \right)\left(n - 1 \right)!\). Now continue with \(a_{0} = 1,a_{1} = 1\) but \(a_{2} \neq 2\). There are \(\left(n - 2 \right)\left(n - 2 \right)!\) ways, and so on.%
\end{activity}

\clearpage\end{document}
