\documentclass{book}

\input{../activities-preamble.tex}
\begin{document}
\setcounter{project}{256}
\addtocounter{project}{-1}
\begin{activity}[]\label{activity-249}
\hypertarget{p-1393}{}%
Another way to count the number of ways to distribute 10 identical pieces of candy to 3 children so no child gets more than 4 is to use the Principle of Inclusion and Exclusion.  Do this, and compare this calculation to what you found in \hyperref[candygenfn]{Activity~\ref{candygenfn}}.%
~\hfill{\tiny\hyperlink{a-256}{[hint]}\hypertarget{q-256}{}}\par\smallskip%
\noindent\textbf{Solution.}\hypertarget{solution-192}{}\quad%
\hypertarget{p-1395}{}%
First, count the number of ways to distribute 10 candies to 3 children without restrictions, then remove the distributions in which one or more kid gets more than 4, using the Principle of Inclusion and Exclusion.  You will get%
\begin{equation*}
\mchoose{3}{10} - \left[ \binom{3}{1}\mchoose{3}{5} - \binom{3}{2}\mchoose{3}{0}\right].
\end{equation*}
This is precisely the same form as you get in \hyperref[candygenfn]{Activity~\ref{candygenfn}}.%
\end{activity}
\end{document}
