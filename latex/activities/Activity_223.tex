\documentclass{book}

\input{../activities-preamble.tex}
\begin{document}
\setcounter{project}{223}
\addtocounter{project}{-1}
\begin{activity}[]\label{activity-216}
\hypertarget{p-1239}{}%
Give a proof of your formula for the principle of inclusion and exclusion.%
\par\smallskip%
\noindent\textbf{Hint 1.}\hypertarget{hint-143}{}\quad%
\hypertarget{p-1240}{}%
Try induction.%
~\hfill{\tiny\hyperlink{a-223}{[hint]}\hypertarget{q-223}{}}\par\smallskip%
\noindent\textbf{Hint 2.}\hypertarget{hint-144}{}\quad%
\hypertarget{p-1241}{}%
We can apply the formula of \hyperref[twosetintersection]{Activity~\ref{twosetintersection}} to get%
\begin{align*}
\left|\bigcup_{i=1}^n A_i \right| \amp = \left|\left(\bigcup_{i=1}^{n-1} A_i\right) \cup A_n \right| \\
\amp = \left| \bigcup_{i=1}^{n-1} A_i\right| + |A_n| - \left|\left( \bigcup_{i=1}^{n-1} A_i\right) \cap A_n\right|\\
\amp = \left| \bigcup_{i=1}^{n-1} A_i\right| + |A_n| - \left|\bigcup_{i=1}^{n-1} A_i \cap A_n\right|
\end{align*}
%
\par\smallskip%
\noindent\textbf{Solution.}\hypertarget{solution-144}{}\quad%
\hypertarget{p-1242}{}%
The principle of inclusion and exclusion for one set says \(|A_1| = |A_1|\).  The principle of inclusion and exclusion for two sets says \(|A_1\cup A_2| = |A_1| + |A_2| - |A_1 \cap A_2|\), and was prove in our solution to \hyperref[twosetintersection]{Activity~\ref{twosetintersection}}.  Now suppose the formula is true for a union of \(n-1\) or fewer sets and \(n \ge 2\).  Since%
\begin{equation*}
A_1 \cup A_2 \cup \cdots \cup A_n = \left(A_1 \cup A_2 \cup \cdots \cup A_{n-1}\right) \cup A_n
\end{equation*}
we can apply the formula of \hyperref[twosetintersection]{Activity~\ref{twosetintersection}} to get%
\begin{align*}
\left|\bigcup_{i=1}^n A_i \right| \amp = \left|\left(\bigcup_{i=1}^{n-1} A_i\right) \cup A_n \right| \\
\amp = \left| \bigcup_{i=1}^{n-1} A_i\right| + |A_n| - \left|\left( \bigcup_{i=1}^{n-1} A_i\right) \cap A_n\right|\\
\amp = \left| \bigcup_{i=1}^{n-1} A_i\right| + |A_n| - \left|\bigcup_{i=1}^{n-1} A_i \cap A_n\right|
\end{align*}
By the inductive hypothesis, we may apply the principle of inclusion and exclusion to the first and last term in the last line above, and can rewrite it as%
\begin{gather*}
\sum_{S:S\subseteq [n-1],S\ne \emptyset} (-1)^{|S|-1}\left|\bigcap_{i:i\in S}A_i \right| + |A_n| - \sum_{S:S\subseteq [n-1],S\ne \emptyset} (-1)^{|S|-1}\left|\bigcap_{i:i\in S}A_i \cap A_n \right|\\
= \sum_{S:S\subseteq [n],S\ne \emptyset} (-1)^{|S|-1}\left|\bigcap_{i:i\in S}A_i \right|,
\end{gather*}
where the last line follows because every nonempty subset of \([n]\) is either (1) a nonempty subset of \([n-1]\), (2) a nonempty subset of \([n-1]\) with  \(n\) added in, or (3) the set \(\{n\}\).  Thus by the principle of mathematical induction, the formula for the principle of inclusion and exclusion holds for all nonnegative integers \(n\).%
\end{activity}
\end{document}
