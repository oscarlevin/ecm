\documentclass{book}

\input{../activities-preamble.tex}
\begin{document}
\setcounter{project}{210}
\addtocounter{project}{-1}
\begin{activity}[]\label{activity-203}
\hypertarget{p-1169}{}%
Explain the relationship between partitions of \(k\) into \(n\) parts and lists \(x_1,x_2\),\textellipsis{}, \(x_n\) of positive integers with \(x_1\ge x_2\ge\ldots \ge x_n\). Such a representation of a partition is called a \terminology{decreasing list}\index{partition of an integer!decreasing list} representation of the partition.%
\par\smallskip%
\noindent\textbf{Solution.}\hypertarget{solution-121}{}\quad%
\hypertarget{p-1170}{}%
There is a bijection between partitions of \(k\) into \(n\) parts and lists, in nonincreasing order, of \(n\) positive integers that add to \(k\), because each multiset of numbers that adds to \(k\) can be listed in nonincreasing order in exactly one way.%
\end{activity}
\end{document}
