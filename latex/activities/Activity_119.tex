\documentclass{book}

\input{../activities-preamble.tex}
\begin{document}
\setcounter{project}{119}
\addtocounter{project}{-1}
\begin{activity}[]\label{activity-112}
\begin{enumerate}[font=\bfseries,label=(\alph*),ref=\alph*]
\item\label{task-153} \hypertarget{p-829}{}%
How many 3-scoop cones are possible?%
\par\smallskip%
\noindent\textbf{Hint.}\hypertarget{hint-78}{}\quad%
\hypertarget{p-830}{}%
You should be able to use the multiplicative principle and nothing else here.%
~\hfill{\tiny\hyperlink{a-119.a}{[hint]}\hypertarget{q-119.a}{}}\item\label{task-154} \hypertarget{p-831}{}%
How many 3-scoop shakes are there?  Write all of them down.%
\par\smallskip%
\noindent\textbf{Hint.}\hypertarget{hint-79}{}\quad%
\hypertarget{p-832}{}%
Maybe the flavors are chocolate, vanilla, and strawberry.  Some of the outcomes are \(ccv\) and \(csv\), but notice that we would not also include \(cvc\) or \(vcs\) because in the blender, order doesn't matter.%
~\hfill{\tiny\hyperlink{a-119.b}{[hint]}\hypertarget{q-119.b}{}}\item\label{task-155} \hypertarget{p-833}{}%
Why doesn't the quotient principle apply here?  What goes wrong?  You might want to list out all 3-scoop cones and form the equivalence classes of shakes to see the issue.%
\end{enumerate}
\end{activity}
\end{document}
