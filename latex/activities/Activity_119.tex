\documentclass{book}

\input{../activities-preamble.tex}
\begin{document}
\setcounter{project}{119}
\addtocounter{project}{-1}
\begin{activity}[]\label{activity-112}
\hypertarget{p-841}{}%
The ice cream shop is down to only 3 flavors.  If you wanted a 3-scoop cone or a 3-scoop shake, made without repeated flavors, there would only be 6 cones possible and only 1 shake.  But what if you allowed repeated flavors?%
\begin{enumerate}[font=\bfseries,label=(\alph*),ref=\alph*]
\item\label{task-153} \hypertarget{p-842}{}%
How many 3-scoop cones are possible?%
~\hfill{\tiny\hyperlink{a-119.a}{[hint]}\hypertarget{q-119.a}{}}\item\label{task-154} \hypertarget{p-844}{}%
How many 3-scoop shakes are there?  Write all of them down.%
~\hfill{\tiny\hyperlink{a-119.b}{[hint]}\hypertarget{q-119.b}{}}\item\label{task-155} \hypertarget{p-846}{}%
Why doesn't the quotient principle apply here?  What goes wrong?  You might want to list out all 3-scoop cones and form the equivalence classes of shakes to see the issue.%
\end{enumerate}
\end{activity}
\end{document}
