\documentclass{book}

\input{../activities-preamble.tex}
\begin{document}
\setcounter{project}{261}
\addtocounter{project}{-1}
\begin{activity}[]\label{originalFibonacci}
\hypertarget{p-1399}{}%
In Fibonacci's original problem, each pair of mature rabbits produces one new pair at the end of each month, but otherwise the situation is the same as in \hyperref[secondorderintroduction]{Activity~\ref{secondorderintroduction}}.  Assuming that we start with one pair of baby rabbits (at the end of month 0), find the generating function for the number of pairs of rabbits we have at the end on \(n\) months.%
~\hfill{\tiny\hyperlink{a-261}{[hint]}\hypertarget{q-261}{}}\par\smallskip%
\noindent\textbf{Solution.}\hypertarget{solution-189}{}\quad%
\hypertarget{p-1401}{}%
Our recurrence becomes \(a_n=a_{n-1}+a_{n-2}\), and following the pattern of \hyperref[secondorderintroduction]{Activity~\ref{secondorderintroduction}} we get%
\begin{align*}
\sum_{n=2}^\infty a_nx^n \amp= \sum_{n=2}^\infty a_{n-1}x^n +
\sum_{n=2}^\infty a_{n-2}x^n\\
\sum_{n=0}^\infty a_nx^n-a_0-a_1x  \amp= x\left(\sum_{n=0}^\infty
a_{n}x^{n}-a_0\right) + x^2\sum_{n=0}^\infty a_{n}x^n\\
(1-x-x^2)\sum_{n=0}^\infty a_nx^n \amp= a_0+a_1x-a_0x\\
\sum_{n=0}^\infty a_nx^n \amp= \frac{a_0+a_1x-a_0x}{(1-x-x^2)}\text{.}
\end{align*}
Since now \(a_0=a_1=1\), we have \(\displaystyle \sum_{n=0}^\infty a_nx^n=
\frac{1}{1-x-x^2}\).%
\end{activity}
\end{document}
