\documentclass{book}

\input{../activities-preamble.tex}
\begin{document}
\setcounter{cpjt}{53}
\addtocounter{cpjt}{-1}
\begin{activity}\label{Ramseybound2}
\leavevmode%
\begin{enumerate}[font=\bfseries,label=(\alph*),ref=\alph*]
\item\label{task-58} \hypertarget{p-422}{}%
What does the equation in \hyperref[Ramseyrecurrence]{Activity~\ref{Ramseyrecurrence}} tell us about \(R(4,4)\)?%
\par\smallskip%
\noindent\item\label{task-59} \hypertarget{p-424}{}%
Consider 17 people arranged in a circle such that each person is acquainted with the first, second, fourth, and eighth person to the right and the first, second, fourth, and eighth person to the left.  can you find a set of four mutual acquaintances?  Can you find a set of four mutual strangers?%
\par\smallskip%
\noindent\textbf{Hint}.\hypertarget{hint-23}{}\quad%
\hypertarget{p-425}{}%
If you could find four mutual acquaintances, you could assume person 1 is among them. And by the generalized pigeonhole principle and symmetry, so are two of the people to the first, second, fourth and eighth to the right. Now there are lots of possibilities for that fourth person. You now have the hard work of using symmetry and the definition of who is acquainted with whom to eliminate all possible combinations of four people. Then you have to think about non-acquaintances.%
\par\smallskip%
\noindent\item\label{task-60} \hypertarget{p-427}{}%
What is \(R(4,4)\)?%
\end{enumerate}
\end{activity}

\clearpage\end{document}
