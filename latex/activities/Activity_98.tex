\documentclass{book}

\input{../activities-preamble.tex}
\begin{document}
\setcounter{project}{98}
\addtocounter{project}{-1}
\begin{activity}[]\label{act_anysizecommittee}
\hypertarget{p-722}{}%
Prove \(\binom{n}{1} + 2 \binom{n}{2} + 3 \binom{n}{3} + \ldots + n \binom{n}{n} = n2^{n - 1}\).%
\par\smallskip%
\noindent\textbf{Hint.}\hypertarget{hint-52}{}\quad%
\hypertarget{p-723}{}%
What if some number of a group of \(n\) people wanted to go to an escape room, and among those going, one needed to be the team captain?%
~\hfill{\tiny\hyperlink{a-98}{[hint]}\hypertarget{q-98}{}}\par\smallskip%
\noindent\textbf{Solution.}\hypertarget{solution-63}{}\quad%
\hypertarget{p-724}{}%
Given a set of \(n\) people we can select a committee of size \(k\) along with a chair from that committee in \(k \binom{n}{k}\) ways. We can select a committee (of size 1, or size 2, or \textellipsis{}) and its chair in \(\binom{n}{1} + 2 \binom{n}{2} + 3 \binom{n}{3} + \ldots + n \binom{n}{n}\) ways. Alternatively, we can explain the term \(n2^{n - 1}\) as follows: choose one of the \(n\) people to chair any of the \(2^{n - 1}\) subsets of the remaining \(n - 1\) people.%
\end{activity}
\end{document}
