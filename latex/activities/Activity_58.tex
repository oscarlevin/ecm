\documentclass{book}

\input{../activities-preamble.tex}
\begin{document}
\setcounter{project}{58}
\addtocounter{project}{-1}
\begin{activity}[]\label{activity-51}
\hypertarget{p-474}{}%
Every bipartite graph (with at least one edge) has a matching, even if it might not be perfect.  Thus we can look for the largest matching in a graph.  If that largest matching includes all the vertices, we have a perfect matching.%
\par
\hypertarget{p-475}{}%
Your ``friend'' claims that she has found the largest matching for the graph below (her matching is in bold). She explains that no other edge can be added, because all the edges not used in her partial matching are connected to matched vertices. Is she correct?%
\begin{sidebyside}{1}{0.3}{0.3}{0}
\begin{sbspanel}{0.4}
\resizebox{\linewidth}{!}{{
\begin{tikzpicture}
\foreach \x in {0,...,4} {
 \coordinate (a\x) at (\x,0);
 \coordinate (b\x) at (\x,2);
 \draw (a\x) \v (b\x) \v;
 }
 \draw[line width=2pt] (a0) -- (b0) (a1) -- (b1) (a3) -- (b2) (a4) -- (b4);
 \draw[very thin] (a0) -- (b1) (a1) -- (b2) (a2)--(b0)  (a0)--(b2) (a3) -- (b4) (a4) -- (b3);
\end{tikzpicture}
}
}
\end{sbspanel}
\end{sidebyside}
\end{activity}
\end{document}
