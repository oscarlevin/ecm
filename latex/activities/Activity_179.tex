\documentclass{book}

\input{../activities-preamble.tex}
\begin{document}
\setcounter{project}{179}
\addtocounter{project}{-1}
\begin{activity}[]\label{activity-172}
\leavevmode%
\begin{enumerate}[font=\bfseries,label=(\alph*),ref=\alph*]
\item\label{task-190} \hypertarget{p-1040}{}%
Show that the number of acceptable tableau insertions from \hyperref[act-tableau]{Activity~\ref{act-tableau}} is always a Catalan number.  That is, give a bijection between the set of Dyck words of length \(2n\) to the set of ways to insert the numbers 1 through \(2n\) into a \(2\times n\) tableau so that both rows and columns are increasing.%
\par\smallskip%
\noindent\textbf{Hint.}\hypertarget{hint-118}{}\quad%
\hypertarget{p-1041}{}%
If you put the numbers into the tableau in order, for each number, you must decide to put it in the top row or the bottom row.  Do you have any other choices?  What would constitute a mistake?%
~\hfill{\tiny\hyperlink{a-179.a}{[hint]}\hypertarget{q-179.a}{}}\item\label{task-191} \hypertarget{p-1042}{}%
Show the number of ways to parenthesize a product of \(n+1\) numbers is \(C_n\) (see \hyperref[act-parenthesize]{Activity~\ref{act-parenthesize}}).%
\par\smallskip%
\noindent\textbf{Hint.}\hypertarget{hint-119}{}\quad%
\hypertarget{p-1043}{}%
One way to parenthesize the product \(abcd\) is \(a((bc)d)\), but this is really \((a((bc)d))\), so that the outer set set of parentheses belong with the product of \(a\) and \(((bc)d)\).  Further, not that if we wrote this just as \((a((bcd\), there would be only one way to insert the right parentheses that would make the product parse correctly.%
~\hfill{\tiny\hyperlink{a-179.b}{[hint]}\hypertarget{q-179.b}{}}\end{enumerate}
\end{activity}
\end{document}
