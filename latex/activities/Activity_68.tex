\documentclass{book}

\input{../activities-preamble.tex}
\begin{document}
\setcounter{project}{68}
\addtocounter{project}{-1}
\begin{activity}[]\label{activity-61}
\leavevmode%
\begin{enumerate}[font=\bfseries,label=(\alph*),ref=\alph*]
\item\label{task-91} \hypertarget{p-535}{}%
Explain why the number of lattice paths from \((0,0)\) to \((k,n-k)\) is the same as number of \(n\)-bit strings of weight \(k\). You do not need to give a formal proof here, just explain the idea.%
\item\label{task-92} \hypertarget{p-536}{}%
Explain why the number of \(n\)-bit strings of weight \(k\) is the same as the number of \(k\)-element subsets of \([n] = \{1,2,3,\ldots, n\}\).   Again, an informal argument is fine.%
\item\label{task-93} \hypertarget{p-537}{}%
Explain why the number of \(k\)-element subsets of \([n]\) is the same as the coefficient of \(x^ky^{n-k}\) in the expansion of \((x+y)^n\).%
\par\smallskip%
\noindent\textbf{Hint.}\hypertarget{hint-26}{}\quad%
\hypertarget{p-538}{}%
This is a little harder.  It might help to notice that there is really nothing special about \([n]\) except that it contains \(n\) elements.  In fact, what can you say about the number of \(k\)-element subsets of \emph{any} \(n\)-element set?  Then, what \(n\)-element set would we look at when expanding \((x+y)^n\)?%
~\hfill{\tiny\hyperlink{a-68.c}{[hint]}\hypertarget{q-68.c}{}}\end{enumerate}
\end{activity}
\end{document}
