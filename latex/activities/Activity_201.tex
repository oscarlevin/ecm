\documentclass{book}

\input{../activities-preamble.tex}
\begin{document}
\setcounter{project}{201}
\addtocounter{project}{-1}
\begin{activity}[]\label{activity-194}
\hypertarget{p-1107}{}%
Find a recurrence for the Lah numbers \(L(k,n)\) similar to the one in \hyperref[secondstirlingrecurrence]{Activity~\ref{secondstirlingrecurrence}}.%
\par\smallskip%
\noindent\textbf{Hint.}\hypertarget{hint-127}{}\quad%
\hypertarget{p-1108}{}%
To see how many broken permutations of a \(k\) element set into \(n\) parts do not have \(k\) is a part by itself, ask yourself how many broken permutations of \([7]\) result from adding 7 to the one of the two permutations in the broken permutation \(\{14, 2356\}\).%
~\hfill{\tiny\hyperlink{a-201}{[hint]}\hypertarget{q-201}{}}\par\smallskip%
\noindent\textbf{Solution.}\hypertarget{solution-90}{}\quad%
\hypertarget{p-1109}{}%
\(L(k,n)\) is the number of broken permutations of a \(k\)-element set into \(n\) parts. Either \(k\) is in an ordered block by itself or it is not. If it is, it can go after any of the \(k-1\) other elements, or it can go at the beginning of any of the \(n\) blocks. If it is not, deleting it gives a broken permutation of a \(k-1\)-element set into \(n-1\) blocks. Thus \(L(k,n)=L(k-1,n-1) + (n+k-1)L(k,n)\).%
\end{activity}

\clearpage\end{document}
