\documentclass{book}

\input{../activities-preamble.tex}
\begin{document}
\setcounter{project}{250}
\addtocounter{project}{-1}
\begin{activity}[]\label{activity-243}
\hypertarget{p-1345}{}%
Suppose once again that \(i\) is an integer between 1 and \(n\).%
\begin{enumerate}[font=\bfseries,label=(\alph*),ref=\alph*]
\item\label{task-245} \hypertarget{p-1346}{}%
What is the generating function in which the coefficient of \(x^k\) is \(1\)? This series is an example of what is called an \terminology{infinite geometric series}.\index{geometric series}\index{series!geometric} In the next part of this problem, it will be useful to interpret the coefficient one as the number of multisets of size \(k\) chosen from the singleton set \(\{i\}\). Namely, there is only one way to choose a multiset of size \(k\) from \(\{i\}\): choose \(i\) exactly \(k\) times.%
\item\label{task-246} \hypertarget{p-1348}{}%
Express the generating function in which the coefficient of \(x^k\) is the number of multisets chosen from \([n]\) as a power of a power series.  What does \hyperref[thm-multisetsize]{Theorem~\ref{thm-multisetsize}} (in which your answer could be expressed as a binomial coefficient) tell you about what this generating function equals?%
~\hfill{\tiny\hyperlink{a-250.b}{[hint]}\hypertarget{q-250.b}{}}\end{enumerate}
\end{activity}
\end{document}
