\documentclass{book}

\input{../activities-preamble.tex}
\begin{document}
\setcounter{cpjt}{246}
\addtocounter{cpjt}{-1}
\begin{activity}\label{coeffinproduct1}
\hypertarget{p-1285}{}%
In \hyperref[coeffinproduct]{Activity~\ref{coeffinproduct}} why is there a \(b_0\) and a \(b_1\) in your expression for the coefficient of \(x^2\) but there is not a \(b_0\) or a \(b_1\) in your expression for the coefficient of \(x^4\)? What is the coefficient of \(x^4\) in%
\begin{equation*}
(a_0+a_1x+a_2x^2+a_3x^3+a_4x^4)(b_0+b_1x+b_2x^2
+b_3x^3+b_4x^4)?
\end{equation*}
%
\par
\hypertarget{p-1286}{}%
Express this coefficient in the form%
\begin{equation*}
\sum_{i=0}^4 \mbox{ something} ,
\end{equation*}
where the something is an expression you need to figure out. Now suppose that \(a_3=0\), \(a_4=0\) and \(b_4=0\). To what is your expression equal after you substitute these values? In particular, what does this have to do with \hyperref[coeffinproduct]{Activity~\ref{coeffinproduct}}?%
\par\smallskip%
\noindent\textbf{Hint}.\hypertarget{hint-155}{}\quad%
\hypertarget{p-1287}{}%
For the last two questions, try multiplying out something simpler first, say \((a_0 + a_1 x + a_2 x^2 )(b_0 + b_1 x + b_2 x^2 )\) . If this problem seems difficult the part that seems to cause students the most problems is converting the expression they get for a product like this into summation notation. If you are having this kind of problem, expand the product \((a_0 + a_1 x + a_2 x^2 )(b_0 + b_1 x + b_2 x^2 )\) and then figure out what the coefficient of \(x^2\) is. Try to write that in summation notation.%
\par\smallskip%
\noindent\end{activity}

\clearpage\end{document}
