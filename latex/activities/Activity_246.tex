\documentclass{book}

\input{../activities-preamble.tex}
\begin{document}
\setcounter{project}{246}
\addtocounter{project}{-1}
\begin{activity}[]\label{coeffinproduct1}
\hypertarget{p-1337}{}%
In \hyperref[coeffinproduct]{Activity~\ref{coeffinproduct}} why is there a \(b_0\) and a \(b_1\) in your expression for the coefficient of \(x^2\) but there is not a \(b_0\) or a \(b_1\) in your expression for the coefficient of \(x^4\)? What is the coefficient of \(x^4\) in%
\begin{equation*}
(a_0+a_1x+a_2x^2+a_3x^3+a_4x^4)(b_0+b_1x+b_2x^2
+b_3x^3+b_4x^4)?
\end{equation*}
%
\par
\hypertarget{p-1338}{}%
Express this coefficient in the form%
\begin{equation*}
\sum_{i=0}^4 \mbox{ something} ,
\end{equation*}
where the something is an expression you need to figure out. Now suppose that \(a_3=0\), \(a_4=0\) and \(b_4=0\). To what is your expression equal after you substitute these values? In particular, what does this have to do with \hyperref[coeffinproduct]{Activity~\ref{coeffinproduct}}?%
~\hfill{\tiny\hyperlink{a-246}{[hint]}\hypertarget{q-246}{}}\par\smallskip%
\noindent\textbf{Solution.}\hypertarget{solution-172}{}\quad%
\hypertarget{p-1340}{}%
There is a \(b_0\) because it can be paired with an \(a_2\) to give the term \(a_2b_0x^4\). Similarly there is a \(b_1\) because it can be paired with \(a_1\) for the same purpose. However there is no \(a_i\) that we can pair with \(b_0\) to get a coefficient of \(x^4\) and no \(a_i\) that we can pair with \(b_3\) to get a coefficient of \(x^4\).%
\par
\hypertarget{p-1341}{}%
The coefficient of \(x^4\) in%
\begin{equation*}
(a_0+a_1x+a_2x^2+a_3x^3+a_4x^4)(b_0+b_1x+b_2x^2
+b_3x^3+b_4x^4)
\end{equation*}
is \(\sum_{i=0}^4 a_ib_{4-i}\). If we substitute \(a_3=0\), \(a_4=0\), and \(b_4 =0\), we get the coefficient of \(x^4\) in \((a_0
+a_1x+a_2x^2)(b_0+b_1x+b_2x^2+b_3x^3)\). This exemplifies the idea that we can get a uniform formula for the coefficient of \(x^i\) (namely, sum all \(a_jb_{i-j}\) from \(j=0\) to \(i\)) in a product of two polynomials if we are willing to say that the coefficient of a power of \(x\) that does not appear in a polynomial is 0.%
\end{activity}
\end{document}
