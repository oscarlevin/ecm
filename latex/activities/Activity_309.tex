\documentclass{book}

\input{../activities-preamble.tex}
\begin{document}
\setcounter{cpjt}{309}
\addtocounter{cpjt}{-1}
\begin{activity}\label{activity-302}
\hypertarget{p-1531}{}%
Suppose we take a partition of \(k\) into \(n\) parts with largest part \(m\), complement it in the smallest rectangle it will fit into, complement the result in the smallest rectangle it will fit into, and continue the process until we get the partition 1 of one into one part.  What can you say about the partition with which we started?%
\par\smallskip%
\noindent\textbf{Hint 1}.\hypertarget{hint-196}{}\quad%
\hypertarget{p-1532}{}%
Suppose we take two repetitions of this complementation process. What rows and columns do we remove from the diagram?%
\par\smallskip%
\noindent\textbf{Hint 2}.\hypertarget{hint-197}{}\quad%
\hypertarget{p-1533}{}%
To deal with an odd number of repetitions of the complementation process, think of it as an even number plus 1. Thus ask what kind of partition gives us the partition of one into one part after this complementation process.%
\par\smallskip%
\noindent\end{activity}

\clearpage\end{document}
