\documentclass{book}

\input{../activities-preamble.tex}
\begin{document}
\setcounter{project}{266}
\addtocounter{project}{-1}
\begin{activity}[]\label{activity-259}
\hypertarget{p-1404}{}%
Use the method of partial fractions to convert the generating function of \hyperref[secondorderintroduction]{Activity~\ref{secondorderintroduction}} into the form%
\begin{equation*}
\frac{c}{x-r_1} + \frac{d}{x-r_2}\text{.}
\end{equation*}
Use this to find a formula for \(a_n\).%
\par\smallskip%
\noindent\textbf{Solution.}\hypertarget{solution-188}{}\quad%
\hypertarget{p-1405}{}%
\(\frac{10}{(1-x-2x^2)}=\frac{10}{(1-2x)(1+x)} = \frac{c}{1-2x} +\frac{d}{1+x}\). This gives us the equations \(c+d=10\) and \(c-2d=0\). Thus \(3d=10\), so \(d=\frac{10}{3}\), and \(c=2d\) so \(c=\frac{20}{3}\). Thus%
\begin{align*}
\sum_{i=0}^\infty a_ix^i  \amp=  \frac{10}{1-x-2x^2}\\
\amp= \frac{20/3}{1-2x} + \frac{10/3}{1+x}\\
\amp= \frac{20}{3}\sum_{i=0}^\infty (2x)^i + \frac{10}{3}\sum_{i=0}^\infty (-1)^ix^i\text{.}
\end{align*}
Thus \(a_i=\frac{20}{3}2^i +\frac{10}{3}(-1)^i\).%
\end{activity}
\end{document}
