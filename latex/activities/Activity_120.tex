\documentclass{book}

\input{../activities-preamble.tex}
\begin{document}
\setcounter{project}{120}
\addtocounter{project}{-1}
\begin{activity}[]\label{activity-113}
\hypertarget{p-851}{}%
Explain why each of the following counting problems have an answer in the form \(\mchoose{n}{k}\) (say what \(n\) and \(k\) should be).  That is, show how each of them is counting the number of \(k\)-element multisets of \([n]\) (or a different set of size \(n\)).%
\begin{enumerate}[font=\bfseries,label=(\alph*),ref=\alph*]
\item\label{task-156} \hypertarget{p-852}{}%
If you have an unlimited supply of pennies, nickles, dimes and quarters, how many handfuls of 7 coins can you grab?%
\item\label{task-157} \hypertarget{p-853}{}%
You roll 10 regular 6-sided dice (because you enjoy playing two games of Yahtzee at once).  Assuming the dice are indistinguishable, how many different outcomes are possible?%
\item\label{task-158} \hypertarget{p-854}{}%
How many functions \(f:[5] \to [7]\) are non-decreasing?  For example, one such function is \(f = \twoline{1 \amp 2 \amp 3 \amp 4 \amp 5}{2 \amp 5 \amp 5 \amp 6 \amp 7}\).%
\end{enumerate}
\end{activity}
\end{document}
