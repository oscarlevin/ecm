\documentclass{book}

\input{../activities-preamble.tex}
\begin{document}
\setcounter{project}{204}
\addtocounter{project}{-1}
\begin{activity}[]\label{Stirlingfalling}
\hypertarget{p-1129}{}%
Each function from a \(k\)-element set \(K\) to an \(n\)-element set \(N\) is a function from \(K\) onto \emph{some} subset of \(N\). If \(J\) is a subset of \(N\) of size \(j\), you know how to compute the number of functions that map onto \(J\) in terms of Stirling numbers. Suppose you add the number of functions mapping onto \(J\) over all possible subsets \(J\) of \(N\). What simple value should this sum equal? Write the equation this gives you.%
\par\smallskip%
\noindent\textbf{Hint.}\hypertarget{hint-129}{}\quad%
\hypertarget{p-1130}{}%
When you add the number of functions mapping onto \(J\) over all possible subsets \(J\) of \(N\), what is the set of functions whose size you are computing?%
~\hfill{\tiny\hyperlink{a-204}{[hint]}\hypertarget{q-204}{}}\par\smallskip%
\noindent\textbf{Solution.}\hypertarget{solution-103}{}\quad%
\hypertarget{p-1131}{}%
The sum should equal the number of functions, \(n^k\). Thus we get \(\sum_{j=0}^n \binom{n}{j}S(k,j)j! = n^k\). By using the fact that \(\binom{n}{j}= P(n,j)/j!\), this may be rewritten as \(\sum_{j=0}^n P(n,j)S(k,j) = n^k.\)%
\end{activity}
\end{document}
