\documentclass{book}

\input{../activities-preamble.tex}
\begin{document}
\setcounter{cpjt}{235}
\addtocounter{cpjt}{-1}
\begin{activity}\label{activity-228}
\hypertarget{p-1248}{}%
It is also possible to multiply infinite polynomials (i.e., power series).  As long as we are only interested in the first few terms, we can find these by multiplying a small number of terms from the two infinite polynomials.%
\begin{enumerate}[font=\bfseries,label=(\alph*),ref=\alph*]
\item\label{task-239} \hypertarget{p-1249}{}%
Multiply \((1+x+x^2+x^3 + \cdots)(1+ x + x^2 + x^3+ \cdots)\).%
\item\label{task-240} \hypertarget{p-1250}{}%
Multiply \((1+x+x^2 + x^3 + \cdots)(x + 2x + 3x^3 + 4x^4 + \cdots)\).%
\item\label{task-241} \hypertarget{p-1251}{}%
What is going on here?  Where have we seen the coefficients of these polynomials before?%
\end{enumerate}
\end{activity}

\clearpage\end{document}
