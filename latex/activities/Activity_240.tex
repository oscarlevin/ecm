\documentclass{book}

\input{../activities-preamble.tex}
\begin{document}
\setcounter{project}{240}
\addtocounter{project}{-1}
\begin{activity}[]\label{activity-233}
\hypertarget{p-1295}{}%
Notice that when we used \(A^2\) to stand for taking two apples, and \(P^3\) to stand for taking three pears, then we used the product \(A^2P^3\) to stand for taking two apples and three pears. Thus we have chosen the picture of the ordered pair (2 apples, 3 pears) to be the product of the pictures of a multiset of two apples and a multiset of three pears. Show that if \(S_1\) and \(S_2\) are sets with picture functions \(P_1\) and \(P_2\) defined on them, and if we define the picture of an ordered pair \((x_1,x_2)\in S_1\times S_2\) to be \(P((x_1,x_2))= P_1(x_1)P_2(x_2)\), then the picture enumerator of \(P\) on the set \(S_1\times S_2\) is \(E_{P_1}(S_1)E_{P_2}(S_2)\). We call this the \terminology{product principle for picture enumerators}.\index{product principle!picture enumerators}\index{picture enumerators!product principle for}%
\par\smallskip%
\noindent\textbf{Solution.}\hypertarget{solution-150}{}\quad%
\hypertarget{p-1296}{}%
%
\begin{align*}
E_P(S_1\times S_2) =\amp \sum_{(x_1,x_2)\in S_1\times
S_2} P(x_1)P(x_2)\\
=\amp
\sum_{x_1:x_1\in S_1}\sum_{x_2:x_2\in S_2} P(x_1)P(x_2)\\
=\amp \sum_{x_1\in S_1}P(x_1)\sum_{x_2\in S_2}P(x_2)\\
=\amp \sum_{x_1\in S_1} P(x_1) E_{P_2}(S_2)\\
=\amp \left(\sum_{x_1\in S_1} P(x_1)\right)E_{P_2}(S_2)\\
=\amp E_{P_1}(S_1)E_{P_2}(S_2)
\end{align*}
%
\end{activity}
\end{document}
