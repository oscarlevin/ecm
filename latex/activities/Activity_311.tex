\documentclass{book}

\input{../activities-preamble.tex}
\begin{document}
\setcounter{cpjt}{311}
\addtocounter{cpjt}{-1}
\begin{activity}\label{activity-304}
\hypertarget{p-1542}{}%
Show that the number of partitions of 7 into 3 parts equals the number of partitions of 10 into three distinct parts.%
\par\smallskip%
\noindent\textbf{Hint}.\hypertarget{hint-199}{}\quad%
\hypertarget{p-1543}{}%
While you could simply display partitions of 7 into three parts and partitions of 10 into three parts, we hope you won't. Perhaps you could write down the partitions of 4 into two parts and the partitions of 5 into two distinct parts and look for a natural bijection between them. So the hope is that you will discover a bijection from the set of partitions of 7 into three parts and the partitions of 10 into three distinct parts. It could help to draw the Young diagrams of partitions of 4 into two parts and the partitions of 5 into two distinct parts.%
\par\smallskip%
\noindent\end{activity}

\clearpage\end{document}
