\documentclass{book}

\input{../activities-preamble.tex}
\begin{document}
\setcounter{project}{311}
\addtocounter{project}{-1}
\begin{activity}[]\label{activity-304}
\hypertarget{p-1594}{}%
Show that the number of partitions of 7 into 3 parts equals the number of partitions of 10 into three distinct parts.%
~\hfill{\tiny\hyperlink{a-311}{[hint]}\hypertarget{q-311}{}}\par\smallskip%
\noindent\textbf{Solution.}\hypertarget{solution-230}{}\quad%
\hypertarget{p-1596}{}%
Given a partition \(\lambda\) of 7 in decreasing list form \(\lambda_1,\lambda_2,\lambda_3\), if we add 0 to \(\lambda_3\), \(1\) to \(\lambda_2\) and 2 to \(\lambda_1\) the resulting partition of 10 has distinct parts. If we take a partition \(\lambda'\) of 10 with distinct parts, then \(\lambda'_1\ge\lambda'_2+1\), \(\lambda'_1\ge\lambda'_2+2\), and \(\lambda'_2\ge \lambda'_3+1\). Therefore if we subtract 2 from \(\lambda'_1\) to get \(\lambda_1\), subtract 1 from \(\lambda'_2\) to get \(\lambda_2\) and let \(\lambda_3= \lambda'_3\), then \(\lambda_1,\lambda_2,\lambda_3\) is the decreasing list representation of a partition of \(10-3=7\). Thus there is a bijection between partitions of \(7\) into three parts and partitions of \(10\) into three distinct parts.%
\end{activity}
\end{document}
