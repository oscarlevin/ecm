\documentclass{book}

\input{../activities-preamble.tex}
\begin{document}
\setcounter{project}{113}
\addtocounter{project}{-1}
\begin{activity}[]\label{twocolorsofbeads}
\hypertarget{p-820}{}%
In how many ways may we attach two identical red beads and two identical blue beads to the corners of a square (with one bead per corner) free to move around in (three-dimensional) space?%
~\hfill{\tiny\hyperlink{a-113}{[hint]}\hypertarget{q-113}{}}\par\smallskip%
\noindent\textbf{Solution.}\hypertarget{solution-87}{}\quad%
\hypertarget{p-822}{}%
Two ways; either the red beads are side-by-side or diagonally opposite. If we think about partitioning lists of 2 \(R\)s and 2 \(B\)s so that two are in the same block if we get one from the other by moving the square, we get two blocks, \(\{RRBB, BRRB, BBRR, RBBR\}\) and \(\{RBRB, BRBR\}\).%
\end{activity}
\end{document}
