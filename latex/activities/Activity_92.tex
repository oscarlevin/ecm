\documentclass{book}

\input{../activities-preamble.tex}
\begin{document}
\setcounter{project}{92}
\addtocounter{project}{-1}
\begin{activity}[]\label{activity-85}
\hypertarget{p-682}{}%
How many 5-letter words can be made with distinct letters?%
\begin{enumerate}[font=\bfseries,label=(\alph*),ref=\alph*]
\item\label{task-138} \hypertarget{p-683}{}%
Explain why the answer is \(P(26,5)\).%
\item\label{task-139} \hypertarget{p-684}{}%
Explain why the answer is also \(\binom{26}{5}\cdot 5!\).%
\item\label{task-140} \hypertarget{p-685}{}%
Express \(P(26,5)\) as the quotient of two factorials.  Use this to find a numerical value for \(\binom{26}{5}\).%
\par\smallskip%
\noindent\textbf{Hint.}\hypertarget{hint-47}{}\quad%
\hypertarget{p-686}{}%
You have \(P(26,5) = \binom{26}{5}\cdot 5!\).  You know that \(P(26,5) = \frac{26!}{21!}\), so you should be able to easily ``solve'' for \(\binom{26}{5}\).%
~\hfill{\tiny\hyperlink{a-92.c}{[hint]}\hypertarget{q-92.c}{}}\end{enumerate}
\end{activity}

\clearpage\end{document}
