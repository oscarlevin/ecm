\documentclass{book}

\input{../activities-preamble.tex}
\begin{document}
\setcounter{project}{72}
\addtocounter{project}{-1}
\begin{activity}[]\label{activity-65}
\hypertarget{p-580}{}%
First, an algebraic proof.  We will see later that \(\binom{n}{k} = \frac{n!}{k!(n-k)!}\), where \(r! = r \cdot (r-1) \cdot (r-2) \cdot\cdots\cdot 2\cdot 1\) (read ``\(r\) factorial'').  Using this algebraic formula, prove the identity.%
\par\smallskip%
\noindent\textbf{Hint.}\hypertarget{hint-27}{}\quad%
\hypertarget{p-581}{}%
This should be really easy; don't let that fool you.  Note that a correct proof must start with one side of the identity and manipulate it into the other side.  The following is NOT a correct proof, because it starts by assuming the statement you wish to prove.%
\begin{align*}
\binom{n}{k}  = \amp \binom{n}{n-k}\\
\frac{n!}{k!(n-k)!} =  \amp \frac{n!}{(n-k)!(n-(n-k)!)}\\
\frac{n!}{k!(n-k)!} =  \amp \frac{n!}{(n-k)!k!)}\\
\frac{n!}{k!(n-k)!} \cdot (k!(n-k)!)=  \amp \frac{n!}{(n-k)!(n-(n-k)!)}\cdot(k!(n-k)!)\\
n! =  \amp n!
\end{align*}
%
~\hfill{\tiny\hyperlink{a-72}{[hint]}\hypertarget{q-72}{}}\par\smallskip%
\noindent\textbf{Solution.}\hypertarget{solution-54}{}\quad%
\hypertarget{p-582}{}%
We have%
\begin{equation*}
\binom{n}{k} = \frac{n!}{k!(n-k)!} = \frac{n!}{(n-k)!(n-(n-k))!} = \binom{n}{n-k}
\end{equation*}
%
\end{activity}
\end{document}
