\documentclass{book}

\input{../activities-preamble.tex}
\begin{document}
\setcounter{project}{125}
\addtocounter{project}{-1}
\begin{activity}[]\label{activity-118}
\leavevmode%
\begin{enumerate}[font=\bfseries,label=(\alph*),ref=\alph*]
\item\label{task-170} \hypertarget{p-889}{}%
In some sort of manufacturing mishap, your set of magnetic letters contains only the first 7 letters of the alphabet, and for some reason 5 identical exclamation marks.  How many ways can you arrange all 12 magnets in a single line?  (one such line is BG!AD!!!FEC!).%
\par\smallskip%
\noindent\textbf{Solution.}\hypertarget{solution-96}{}\quad%
\hypertarget{p-890}{}%
To use the quotient principle, treat the exclamation marks as distinct, and arrange all 12 magnets in \(12!\) ways.  Then group lines if the exclamation marks are in the same positions.  There are \(5!\) lines in each group, so there are \(12!/5!\) groups.%
\par
\hypertarget{p-891}{}%
Alternatively, you can see this is correct by thinking about picking one of the 12 positions to put the A into, then one of the remaining positions for the B, and so on.  This is \(P(12,7)\), just to put the letters down.  But now you have only one choice as to where to put the exclamation marks.%
\item\label{task-171} \hypertarget{p-892}{}%
What does this question have to do with placing \(7\) distinct books on \(6\) shelves?%
~\hfill{\tiny\hyperlink{a-125.b}{[hint]}\hypertarget{q-125.b}{}}\par\smallskip%
\noindent\textbf{Solution.}\hypertarget{solution-97}{}\quad%
\hypertarget{p-894}{}%
Treat each space separated by exclamation marks as a shelf.  So there is a shelf to the left of the first exclamation mark, a shelf between the first and the second, and so on, including a shelf after the last exclamation mark.%
\par
\hypertarget{p-895}{}%
Physically, you could think of shelving books as first lining up all the books with bookmarks between groups of books.  You then put everything before the first separator on the first shelf, everything between the first and second separator on the second shelf, and so on.%
\item\label{task-172} \hypertarget{p-896}{}%
Oh no! Your 5 year old left the 7 magnetic letters in the oven too long and now they are all identical blobs of plastic.  How many strings of the 12 magnets can you make now?%
~\hfill{\tiny\hyperlink{a-125.c}{[hint]}\hypertarget{q-125.c}{}}\par\smallskip%
\noindent\textbf{Solution.}\hypertarget{solution-98}{}\quad%
\hypertarget{p-898}{}%
This should be \(P(12,7)/7! = \binom{12}{7}\).  You can see this by realizing that we have really created a bit string of length 12 and weight 7 (thinking of the blobs as the 1's).%
\item\label{task-173} \hypertarget{p-899}{}%
What sort of distribution problem does the previous task correspond to?  Write a question about books and shelves that has the same answer.%
\par\smallskip%
\noindent\textbf{Solution.}\hypertarget{solution-99}{}\quad%
\hypertarget{p-900}{}%
The letters were the books before, and now they still are, but now the books are indistinguishable.  So how many ways can you distribute 7 identical books to 6 shelves?  \(\binom{12}{7}\), which by the way, is also \(\mchoose{6}{7}\).%
\end{enumerate}
\end{activity}
\end{document}
