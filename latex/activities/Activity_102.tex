\documentclass{book}

\input{../activities-preamble.tex}
\begin{document}
\setcounter{project}{102}
\addtocounter{project}{-1}
\begin{activity}[]\label{activity-95}
\hypertarget{p-740}{}%
\(\binom{2}{2} + \binom{3}{2} + \binom{4}{2} + \ldots + \binom{n}{2} = \binom{n + 1}{3}\).%
\par\smallskip%
\noindent\textbf{Hint.}\hypertarget{hint-56}{}\quad%
\hypertarget{p-741}{}%
Try bit strings here.  What length and weight are we looking at?  Where might the first (left-most) one be?%
~\hfill{\tiny\hyperlink{a-102}{[hint]}\hypertarget{q-102}{}}\par\smallskip%
\noindent\textbf{Solution.}\hypertarget{solution-67}{}\quad%
\hypertarget{p-742}{}%
The term \(\binom{n + 1}{3}\) is the number of binary strings of length \(n + 1\) consisting of three 1's (and the rest 0's). The left hand side counts these by where in the string the left-most 1 appears. Let \(a_{1}a_{2}a_{3}\ldots a_{n + 1}\) be a string of length \(n + 1\). There are \(\binom{n}{2}\) strings when \(a_{1} = 1\), \(\binom{n - 1}{2}\) strings when \(a_{2} = 1\) is the leftmost 1, and so on, up to \(\binom{2}{2}\) strings when \(a_{n - 1} = 1\) is the leftmost 1. In this last case the string looks like \(000\ldots 0111\).%
\end{activity}
\end{document}
