\documentclass{book}

\input{../activities-preamble.tex}
\begin{document}
\setcounter{project}{18}
\addtocounter{project}{-1}
\begin{activity}[]\label{activity-13}
\hypertarget{p-205}{}%
Consider the following graph:%
\begin{sidebyside}{1}{0.36}{0.36}{0}
\begin{sbspanel}{0.28}
\resizebox{\linewidth}{!}{{
\begin{tikzpicture}[scale=.7]
\foreach \x in {0, 45, ..., 315}
  \draw  (\x:2) \v -- (\x+45:2);
\draw (0,0) \v -- (45:2) (0,0) -- (135:2) (0,0) -- (225:2) (0,0) -- (315:2);
\draw (-1,0) \v -- (90:2) (-1,0) -- (180:2) (-1,0) -- (270:2);
\draw (1,0) \v -- (90:2) (1,0) -- (0:2) (1,0) -- (270:2);
\end{tikzpicture}
}
}
\end{sbspanel}
\end{sidebyside}
\begin{enumerate}[font=\bfseries,label=(\alph*),ref=\alph*]
\item\label{task-16} \hypertarget{p-206}{}%
Find a Hamilton path.  Can your path be extended to a Hamilton cycle?%
\item\label{task-17} \hypertarget{p-208}{}%
Is the graph bipartite? If so, how many vertices are in each ``part''?%
\item\label{task-18} \hypertarget{p-210}{}%
Use your answer to part (b) to prove that the graph has no Hamilton cycle.%
\item\label{task-19} \hypertarget{p-213}{}%
Suppose you have a bipartite graph \(G\) in which one part has at least two more vertices than the other.  Prove that \(G\) does not have a Hamilton path.%
\end{enumerate}
\end{activity}
\end{document}
