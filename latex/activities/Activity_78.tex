\documentclass{book}

\input{../activities-preamble.tex}
\begin{document}
\setcounter{project}{78}
\addtocounter{project}{-1}
\begin{activity}[]\label{activity-pascalrow-dc}
\hypertarget{p-595}{}%
Consider the question: how many subsets of \([n]\) are there?%
\begin{enumerate}[font=\bfseries,label=(\alph*),ref=\alph*]
\item\label{task-111} \hypertarget{p-596}{}%
How many subsets have cardinality 0?  How many have cardinality 1?  And so on.  How many would all of these give us all together?%
\par\smallskip%
\noindent\textbf{Hint.}\hypertarget{hint-34}{}\quad%
\hypertarget{p-597}{}%
Your final answer here should be a sum, since we have counted the number of subsets partitioned into disjoint collections.%
~\hfill{\tiny\hyperlink{a-78.a}{[hint]}\hypertarget{q-78.a}{}}\item\label{task-112} \hypertarget{p-598}{}%
Explain why the answer is also \(2^n\).%
\par\smallskip%
\noindent\textbf{Hint.}\hypertarget{hint-35}{}\quad%
\hypertarget{p-599}{}%
Think about how many choices you have for each element.  You could put 1 in the subset or not.  You could include \(2\) in the subset or not.%
~\hfill{\tiny\hyperlink{a-78.b}{[hint]}\hypertarget{q-78.b}{}}\end{enumerate}
\bigbreak
\hypertarget{p-600}{}%
Conclude: Since each side of the identity is the answer to the same counting question, we have established the identity.%
\end{activity}

\clearpage\end{document}
