\documentclass{book}

\input{../activities-preamble.tex}
\begin{document}
\setcounter{project}{78}
\addtocounter{project}{-1}
\begin{activity}[]\label{activity-pascalrow-dc}
\hypertarget{p-616}{}%
Consider the question: how many subsets of \([n]\) are there?%
\begin{enumerate}[font=\bfseries,label=(\alph*),ref=\alph*]
\item\label{task-112} \hypertarget{p-617}{}%
How many subsets have cardinality 0?  How many have cardinality 1?  And so on.  How many would all of these give us all together?%
~\hfill{\tiny\hyperlink{a-78.a}{[hint]}\hypertarget{q-78.a}{}}\item\label{task-113} \hypertarget{p-620}{}%
Explain why the answer is also \(2^n\).%
~\hfill{\tiny\hyperlink{a-78.b}{[hint]}\hypertarget{q-78.b}{}}\end{enumerate}
\bigbreak
\hypertarget{p-623}{}%
Conclude: Since each side of the identity is the answer to the same counting question, we have established the identity.%
\end{activity}
\end{document}
