\documentclass{book}

\input{../activities-preamble.tex}
\begin{document}
\setcounter{cpjt}{78}
\addtocounter{cpjt}{-1}
\begin{activity}\label{activity-pascalrow-dc}
\hypertarget{p-579}{}%
Consider the question: how many subsets of \([n]\) are there?%
\begin{enumerate}[font=\bfseries,label=(\alph*),ref=\alph*]
\item\label{task-109} \hypertarget{p-580}{}%
How many subsets have cardinality 0?  How many have cardinality 1?  And so on.  How many would all of these give us all together?%
\par\smallskip%
\noindent\textbf{Hint}.\hypertarget{hint-34}{}\quad%
\hypertarget{p-581}{}%
Your final answer here should be a sum, since we have counted the number of subsets partitioned into disjoint collections.%
\item\label{task-110} \hypertarget{p-582}{}%
Explain why the answer is also \(2^n\).%
\par\smallskip%
\noindent\textbf{Hint}.\hypertarget{hint-35}{}\quad%
\hypertarget{p-583}{}%
Think about how many choices you have for each element.  You could put 1 in the subset or not.  You could include \(2\) in the subset or not.%
\end{enumerate}
\bigbreak
\hypertarget{p-584}{}%
Conclude: Since each side of the identity is the answer to the same counting question, we have established the identity.%
\end{activity}

\clearpage\end{document}
