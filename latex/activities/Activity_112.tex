\documentclass{book}

\input{../activities-preamble.tex}
\begin{document}
\setcounter{cpjt}{112}
\addtocounter{cpjt}{-1}
\begin{activity}\label{tennispairings2}
\hypertarget{p-770}{}%
We first gave this problem as \hyperref[tennispairings1]{Activity~\ref{tennispairings1}}. Now we have several ways to approach the problem. A tennis club has \(2n\) members. We want to pair up the members by twos for singles matches.%
\begin{enumerate}[font=\bfseries,label=(\alph*),ref=\alph*]
\item\label{task-145} \hypertarget{p-771}{}%
In how many ways may we pair up all the members of the club? Give at least two solutions different from the one you gave in   \hyperref[tennispairings1]{Activity~\ref{tennispairings1}}. (You may not have done \hyperref[tennispairings1]{Activity~\ref{tennispairings1}}. In that case, see if you can find three solutions.)%
\par\smallskip%
\noindent\textbf{Hint}.\hypertarget{hint-67}{}\quad%
\hypertarget{p-772}{}%
You might first choose the pairs of people. You might also choose to make a list of all the people and then take them by twos from the list.%
\par\smallskip%
\noindent\item\label{task-146} \hypertarget{p-775}{}%
Suppose that in addition to specifying who plays whom, for each pairing we say who serves first.  Now in how many ways may we specify our pairs? Try to find as many solutions as you can.%
\par\smallskip%
\noindent\textbf{Hint}.\hypertarget{hint-68}{}\quad%
\hypertarget{p-776}{}%
You might first choose ordered pairs of people, and have the first person in each pair serve first. You might also choose to make a list of all the people and then take them by twos from the list in order.%
\par\smallskip%
\noindent\end{enumerate}
\end{activity}

\clearpage\end{document}
