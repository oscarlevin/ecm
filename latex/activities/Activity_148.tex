\documentclass{book}

\input{../activities-preamble.tex}
\begin{document}
\setcounter{project}{148}
\addtocounter{project}{-1}
\begin{activity}[]\label{activity-141}
\hypertarget{p-987}{}%
Consider the recurrence relation \(a_n = 5a_{n-1} - 6a_{n-2}\).%
\begin{enumerate}[font=\bfseries,label=(\alph*),ref=\alph*]
\item\label{task-183} \hypertarget{p-988}{}%
Show that \(a_n = 2^n\) and \(a_n = 3^n\) are both solutions to the recurrence relation.  That is, they are sequences for which the recurrence relation holds.%
\par\smallskip%
\noindent\textbf{Hint.}\hypertarget{hint-106}{}\quad%
\hypertarget{p-989}{}%
You could do a full proof by induction here, but really you are just plugging these in to see if the recurrence works for them.%
~\hfill{\tiny\hyperlink{a-148.a}{[hint]}\hypertarget{q-148.a}{}}\item\label{task-184} \hypertarget{p-990}{}%
Show that if \(\alpha\) and \(\beta\) are any constants, \(a_n = \alpha 2^n + \beta 3^n\) is also a solution to the recurrence relation.%
\item\label{task-185} \hypertarget{p-991}{}%
Might there be another solution to the recurrence relation?  Suppose \(a_n = r^n\) for some non-zero constant \(r\).  Show that \(r = 2\) or \(r = 3\).%
\par\smallskip%
\noindent\textbf{Hint.}\hypertarget{hint-107}{}\quad%
\hypertarget{p-992}{}%
If you do the substitution, you will get a degree \(n\) polynomial equation in the variable \(r\).  Solving that equation amounts to finding the roots of the polynomial.%
~\hfill{\tiny\hyperlink{a-148.c}{[hint]}\hypertarget{q-148.c}{}}\end{enumerate}
\end{activity}
\end{document}
