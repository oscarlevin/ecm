\documentclass{book}

\input{../activities-preamble.tex}
\begin{document}
\setcounter{project}{308}
\addtocounter{project}{-1}
\begin{activity}[]\label{activity-301}
\hypertarget{p-1582}{}%
The idea of conjugation of a partition could be defined without the geometric interpretation of a Young diagram, but it would seem far less natural without the geometric interpretation. Another idea that seems much more natural in a geometric context is this. Suppose we have a partition of \(k\) into \(n\) parts with largest part \(m\). Then the Young diagram of the partition can fit into a rectangle that is \(m\) or more units wide (horizontally) and \(n\) or more units deep. Suppose we place the Young diagram of our partition in the top left-hand corner of an \(m'\) unit wide and \(n'\) unit deep rectangle with \(m'\ge m\) and \(n' \ge n\), as in \hyperref[complementpartition]{Figure~\ref{complementpartition}}.%
\begin{figure}
\centering
\includegraphics[width=0.7\linewidth]{images/complementpartition}
\caption{To complement the partition \((5,3,3,2)\) in a 6 by 5 rectangle: enclose it in the rectangle, rotate, and cut out the original Young diagram.\label{complementpartition}}
\end{figure}
\begin{enumerate}[font=\bfseries,label=(\alph*),ref=\alph*]
\item\label{task-266} \hypertarget{p-1583}{}%
Why can we interpret the part of the rectangle not occupied by our Young diagram, rotated in the plane, as the Young diagram of another partition? This is called the \terminology{complement}\index{complement of a partition} of our partition in the rectangle.%
\par\smallskip%
\noindent\textbf{Solution.}\hypertarget{solution-231}{}\quad%
\hypertarget{p-1584}{}%
If we fill the rectangle with unit squares, those not in the Young diagram of the original partition \(\lambda\) will fall into rows.  The lengths of the rows are nonnegative, and are nondecreasing as we move down. Therefore, after we rotate through 180 degrees, these same rows will be listed in the opposite order, lined up along the left sides, and will have nonincreasing length. Thus they will be the Young diagram of a partition.%
\item\label{task-267} \hypertarget{p-1585}{}%
What integer is being partitioned by the complement?%
\par\smallskip%
\noindent\textbf{Solution.}\hypertarget{solution-232}{}\quad%
\hypertarget{p-1586}{}%
The integer being partitioned will be \(m'n'-k\).%
\item\label{task-268} \hypertarget{p-1587}{}%
What conditions on \(m'\) and \(n'\) guarantee that the complement has the same number of parts as the original one?%
~\hfill{\tiny\hyperlink{a-308.c}{[hint]}\hypertarget{q-308.c}{}}\par\smallskip%
\noindent\textbf{Solution.}\hypertarget{solution-233}{}\quad%
\hypertarget{p-1589}{}%
If \(m'>m\) and \(n'=n\). the two partitions will have the same number of parts, because we will have a nonzero number of empty squares at the end of each row of the Young diagram of \(\lambda\). If \(m'=m\) and \(n'-n\) is the multiplicity of the largest part of \(\lambda\), they will have the same number of parts. Otherwise, their numbers of parts will differ.%
\item\label{task-269} \hypertarget{p-1590}{}%
What conditions on \(m'\) and \(n'\) guarantee that the complement has the same largest part as the original one?%
~\hfill{\tiny\hyperlink{a-308.d}{[hint]}\hypertarget{q-308.d}{}}\par\smallskip%
\noindent\textbf{Solution.}\hypertarget{solution-234}{}\quad%
\hypertarget{p-1592}{}%
If \(n'\gt n\) and \(m=m'\), then the two partitions will have the same largest part. If \(n'=n\) and \(m'-m\) is the smallest part of \(\lambda\), then they will have the same largest part. Otherwise, their largest parts will differ.%
\item\label{task-270} \hypertarget{p-1593}{}%
Is it possible for the complement to have both the same number of parts and the same largest part as the original one?%
\par\smallskip%
\noindent\textbf{Solution.}\hypertarget{solution-235}{}\quad%
\hypertarget{p-1594}{}%
For the two partitions to have the same number of parts, either \(m'=m\) or \(n'=n\). If \(m'=m\) and they have the same largest part, then \(n'>n\). But this is consistent with \(n'-n\) being the multiplicity of the largest part of \(\lambda\). Thus they can have the same number of parts and the same largest part if \(m'=m\) and \(n'-n\) is the multiplicity of the largest part of \(\lambda\), or similarly if \(n=n'\) and \(m'-m\) is the smallest part of \(\lambda\).%
\item\label{task-271} \hypertarget{p-1595}{}%
If we complement a partition in an \(m'\) by \(n'\) box and then complement that partition in an \(m'\) by \(n'\) box again, do we get the same partition that we started with?%
\par\smallskip%
\noindent\textbf{Solution.}\hypertarget{solution-236}{}\quad%
\hypertarget{p-1596}{}%
If we complement a partition in an \(m'\) by \(n'\) box and then complement that partition in \emph{the same rectangle}, then we get the original partition back.%
\end{enumerate}
\end{activity}
\end{document}
