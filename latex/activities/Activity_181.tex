\documentclass{book}

\input{../activities-preamble.tex}
\begin{document}
\setcounter{project}{181}
\addtocounter{project}{-1}
\begin{activity}[]\label{activity-174}
\hypertarget{p-1073}{}%
Consider the recurrence relation%
\begin{equation*}
C_{n + 1} = \sum_{i = 0}^n C_iC_{n-i} = C_{0}C_{n} + C_{1}C_{n - 1} + \ldots + C_{n}C_{0}
\end{equation*}
with \(C_0 = 1\).%
\begin{enumerate}[font=\bfseries,label=(\alph*),ref=\alph*]
\item\label{task-192} \hypertarget{p-1074}{}%
Calculate the first 6 values of the sequence from this recurrence relation, to verify that it appears to agree with the Catalan numbers.%
\par\smallskip%
\noindent\textbf{Solution.}\hypertarget{solution-111}{}\quad%
\hypertarget{p-1075}{}%
The first six values of the sequence are 1, 2, 5, 11, 42, 132.%
\item\label{task-193} \hypertarget{p-1076}{}%
Prove that the Catalan numbers satisfy the recurrence relation.%
~\hfill{\tiny\hyperlink{a-181.b}{[hint]}\hypertarget{q-181.b}{}}\par\smallskip%
\noindent\textbf{Solution.}\hypertarget{solution-112}{}\quad%
\hypertarget{p-1078}{}%
We know that the Catalan numbers count the number of rooted binary trees.  Specifically, the number of rooted binary trees with \(n+2\) leaves is \(C_{n+1}\).  Thus we must simply argue that the number of rooted binary trees with \(n+2\) leaves is also%
\begin{equation*}
\sum_{i=0}^{n} C_iC_{n-i} = C_0C_{n} + C_1C_{n-1} + \cdots + C_{n}C_0.
\end{equation*}
%
\par
\hypertarget{p-1079}{}%
To see this, consider the result of removing the root of a rooted binary tree.  This results in two more trees (the left and right child of the root are the new roots).  These will either be rooted binary trees, or a single vertex (in the case that one of the children of the root was a leaf already).  There are \(n+1\) cases: the left tree is just the root (so the right tree has \(n+1\) leaves), the left tree has 2 leaves (so the right tree has \(n\) leaves), the left tree has 3 leaves (so the right tree has \(n-1\) leaves), and so on until the case where the left tree has \(n+1\) leaves (and the right tree is just the root).%
\par
\hypertarget{p-1080}{}%
Each of these cases corresponds to a term in the sum.  When the left tree has \(i+1\) leaves, it could be any of \(C_{i}\) trees, and for each, the right tree could be any of \(C_{n-i}\) trees (they all have \(n+2-(i+1) = n-i+1\) leaves).  Using the product and then sum prinicple, we get the right hand side of the recurrence.%
\item\label{task-194} \hypertarget{p-1081}{}%
Prove that the number of triangulations of a convex polygon with \(n+2\) sides also satisfies this recurrence relation (if you haven't already done so in \hyperref[act-catalanfirst]{Activity~\ref{act-catalanfirst}}).  Conclude that the triangulations problem is also solved by the Catalan numbers.%
~\hfill{\tiny\hyperlink{a-181.c}{[hint]}\hypertarget{q-181.c}{}}\par\smallskip%
\noindent\textbf{Solution.}\hypertarget{solution-113}{}\quad%
\hypertarget{p-1083}{}%
On one hand, we have that \(C_{n+1}\) is the number of ways to triangulate a polygon with \(n+3\) sides.  Now fix a side the of polygon and consider each of the \(n+1\) non-incident vertices.  For each vertex, we can form exactly one triangle that has our fixed side as a side.  The other two sides of this triangle will create two new polygons with the other sides of the original polygon (perhaps the polygon will only have 1 side, if the vertex is chosen adjacent to the fixed side).  These will give you your terms in the sum on the right hand side.%
\end{enumerate}
\end{activity}
\end{document}
