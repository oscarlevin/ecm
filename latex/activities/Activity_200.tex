\documentclass{book}

\input{../activities-preamble.tex}
\begin{document}
\setcounter{project}{200}
\addtocounter{project}{-1}
\begin{activity}[]\label{secondstirlingrecurrence}
\hypertarget{p-1116}{}%
Now generalize.  In a partition of the set \([k]\), the number \(k\) is either in a block by itself, or it is not.  Find a two variable recurrence for \(S(n,k)\), valid for \(n\) and \(k\) larger than one.%
\par\smallskip%
\noindent\textbf{Hint.}\hypertarget{hint-126}{}\quad%
\hypertarget{p-1117}{}%
The number of partitions of \([k]\) into \(n\) parts in which \(k\) is not in a block relates to the number of partitions of \(k-1\) into some number of blocks in a way that involves \(n\). With this in mind, review how you proved Pascal's (recurrence) equation.%
~\hfill{\tiny\hyperlink{a-200}{[hint]}\hypertarget{q-200}{}}\par\smallskip%
\noindent\textbf{Solution.}\hypertarget{solution-99}{}\quad%
\hypertarget{p-1118}{}%
The number of partitions of \([k]\) into \(n\) parts in which \(k\) is in a block with other elements of \([k]\) is equal \(n\) times the number of partitions of \([k-1]\) into \(n\) blocks, because \(k\) could be in any of the \(n\) parts, and since it is in a block with other elements of \([k-1]\), removing it leaves a partition of \([k-1]\) into \(n\) blocks. The number of partitions of \([k]\) into \(n\) blocks in which \(k\) is in a block by itself is the number of partitions of \([k]\) into \(n-1\) blocks, because you can get any such partition by deleting the block containing \(k\) from a partition of \([k]\) in which \(k\) is in a block by itself. Thus \(S(k,n) = S(k-1,n-1) + nS(k-1,n)\).%
\end{activity}
\end{document}
