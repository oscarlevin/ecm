\documentclass{book}

\input{../activities-preamble.tex}
\begin{document}
\setcounter{project}{221}
\addtocounter{project}{-1}
\begin{activity}[]\label{inclusion-exclusionunion}
\hypertarget{p-1209}{}%
Use notation something like that of \hyperref[intersectionnotation]{Equation~(\ref{intersectionnotation})} and \hyperref[notationsolution]{Equation~(\ref{notationsolution})} to express the answer to \hyperref[nsetintersection]{Activity~\ref{nsetintersection}}. Note there are many different correct ways to do this problem. Try to write down more than one and choose the nicest one you can. Say why you chose it (because your view of what makes a formula nice may be different from somebody else's). The nicest formula won't necessarily involve all the elements of \hyperref[intersectionnotation]{Equations~(\ref{intersectionnotation})} and \hyperref[notationsolution]{(\ref{notationsolution})}.%
\par\smallskip%
\noindent\textbf{Solution.}\hypertarget{solution-128}{}\quad%
\hypertarget{p-1210}{}%
%
\begin{equation*}
\left|\bigcup_{i=1}^n A_i\right| = \sum_{S:S\subseteq [n],
S\not=\emptyset}(-1)^{|S|-1}|\bigcap_{i: i\in S}A_i|
\end{equation*}
I chose this way of writing the formula partly because it is efficient with symbols; for example, it uses only one sum sign. But more importantly I chose it because it captures what I would want to say in words: ``You sum, over all ways of choosing an intersection of the sets \(A_i\), the size of the intersection times a sign factor that is -1 if you are intersecting an even number of sets and 1 if you are intersecting an odd number." If I were writing my solution out in words, I would probably assume that nobody would think about the possibility of an intersection of the empty set of the \(A_i\)s, but I had to put the \(S\not=\emptyset\) in my formula because otherwise the formula would have had us consider the possibility that \(S\) was empty.%
\end{activity}
\end{document}
