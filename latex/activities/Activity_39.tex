\documentclass{book}

\input{../activities-preamble.tex}
\begin{document}
\setcounter{cpjt}{39}
\addtocounter{cpjt}{-1}
\begin{activity}\label{activity-32}
\hypertarget{p-352}{}%
We will prove the Five Color Theorem, that every planar graph has a proper vertex coloring using five (or fewer) colors.  We proceed by induction on the number of vertices.  So assume that \(G\) is a planar graph with \(n\) vertices, and all planar graphs with fewer vertices have a proper 5-coloring.  We will use \(\{1,2,3,4,5\}\) as our set of colors.%
\begin{enumerate}[font=\bfseries,label=(\alph*),ref=\alph*]
\item\label{task-48} \hypertarget{p-353}{}%
Let \(v\) be a vertex of degree five or less (why can we do this?).  Let \(H\) be the graph resulting from deleting \(v\) and all its incident edges.  What can you say about \(H\)?%
\item\label{task-49} \hypertarget{p-354}{}%
Let \(v_1, v_2, v_3, v_4, v_5\) be the vertices adjacent to \(v\) in \(G\).  Why is it okay to assume there really are five and further that these five vertices are colored distinctly from each other?%
\par\smallskip%
\noindent\textbf{Hint}.\hypertarget{hint-12}{}\quad%
\hypertarget{p-355}{}%
If not, what could we do with \(v\)?%
\item\label{task-50} \hypertarget{p-356}{}%
Assume \(v_1, \ldots, v_5\) are drawn in the plane in clockwise order around \(v\), and further that they are colored \(1,2,3,4,5\) respectively.  Now consider the induced subgraph \(H_{1,3}\) of \(H\) consisting of all vertices colored 1 and 3 (and their edges).  Why can we assume that there is a path from \(v_1\) to \(v_3\) contained in \(H_{1,3}\)?%
\par\smallskip%
\noindent\textbf{Hint}.\hypertarget{hint-13}{}\quad%
\hypertarget{p-357}{}%
If not, what would happen if we recolor \(v_1\) with color 3, and all its adjacent vertices in \(H_{1,3}\) with color 1 and so on?  That is, could we swap colors to free up color 1 for \(v\)?%
\item\label{task-51} \hypertarget{p-358}{}%
Now consider \(H_{2,4}\), the induced subgraph of \(H\) of vertices colored 2 and 4.  Explain why their cannot be a path connecting \(v_2\) and \(v_4\), and why this completes the proof.%
\par\smallskip%
\noindent\textbf{Hint}.\hypertarget{hint-14}{}\quad%
\hypertarget{p-359}{}%
Remember where \(v_2\) and \(v_4\) are relative to the path connecting \(v_1\) and \(v_3\).%
\end{enumerate}
\end{activity}

\clearpage\end{document}
