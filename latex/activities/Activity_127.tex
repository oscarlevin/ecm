\documentclass{book}

\input{../activities-preamble.tex}
\begin{document}
\setcounter{project}{127}
\addtocounter{project}{-1}
\begin{activity}[]\label{activity-120}
\hypertarget{p-900}{}%
In how many ways may we put \(k\) identical books onto \(n\) shelves if each shelf must get at least one book?%
\par\smallskip%
\noindent\textbf{Hint.}\hypertarget{hint-85}{}\quad%
\hypertarget{p-901}{}%
We already know how to place \(k\) distinct books onto \(n\) distinct shelves so that each shelf gets at least one. Suppose we replace the distinct books with identical ones. If we permute the distinct books before replacement, does that affect the final outcome? There are other ways to solve this problem.%
~\hfill{\tiny\hyperlink{a-127}{[hint]}\hypertarget{q-127}{}}\par\smallskip%
\noindent\textbf{Solution.}\hypertarget{solution-95}{}\quad%
\hypertarget{p-902}{}%
In \hyperref[bookcaseeveryshelf]{problem~\ref{bookcaseeveryshelf}} we showed that with \(k\) distinct books we could place the books in \(k!\binom{k-1}{n-1}\) ways. We can partition these arrangements of distinct books into blocks, where each block consists of all arrangements that we get just by permuting the books among themselves. Thus each block has \(k!\) arrangements in it, and each arrangement corresponds to an arrangement of identical books. Thus there are \(\binom{k-1}{n-1}\) ways to arrange identical books.%
\end{activity}
\end{document}
