\documentclass{book}

\input{../activities-preamble.tex}
\begin{document}
\setcounter{cpjt}{110}
\addtocounter{cpjt}{-1}
\begin{activity}\label{roundtable}
\hypertarget{p-762}{}%
In how many ways may \(n\) people sit around a round table? (Assume that when people are sitting around a round table, all that really matters is who is to each person's right. For example, if we can get one arrangement of people around the table from another by having everyone get up and move to the right one place and sit back down, we get an equivalent arrangement of people. Notice that you can get a list from a seating arrangement by marking a place at the table, and then listing the people at the table, starting at that place and moving around to the right.) There are at least two different ways of doing this problem. Try to find them both, especially the one that uses the quotient principle.%
\par\smallskip%
\noindent\textbf{Hint 1}.\hypertarget{hint-63}{}\quad%
\hypertarget{p-763}{}%
The problem suggests that you think about how to get a list from a seating arrangement. Could every list of n distinct people come from a seating chart? How many lists of n distinct people are there? How many lists could we get from a given seating chart by taking different starting places?%
\par\smallskip%
\noindent\textbf{Hint 2}.\hypertarget{hint-64}{}\quad%
\hypertarget{p-764}{}%
For a different way of doing the problem, suppose that you have chosen one person, say the first one in a list of the people in alphabetical order by name. Now seat that person. Does it matter where they sit? In how many ways can you seat the remaining people? Does it matter where the second person in alphabetical order sits?%
\par\smallskip%
\noindent\end{activity}

\clearpage\end{document}
