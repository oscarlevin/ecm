\documentclass{book}

\input{../activities-preamble.tex}
\begin{document}
\setcounter{project}{110}
\addtocounter{project}{-1}
\begin{activity}[]\label{roundtable}
\hypertarget{p-802}{}%
In how many ways may \(n\) people sit around a round table? (Assume that when people are sitting around a round table, all that really matters is who is to each person's right. For example, if we can get one arrangement of people around the table from another by having everyone get up and move to the right one place and sit back down, we get an equivalent arrangement of people. Notice that you can get a list from a seating arrangement by marking a place at the table, and then listing the people at the table, starting at that place and moving around to the right.) There are at least two different ways of doing this problem. Try to find them both, especially the one that uses the quotient principle.%
~\hfill{\tiny\hyperlink{a-110}{[hint]}\hypertarget{q-110}{}}\end{activity}
\end{document}
