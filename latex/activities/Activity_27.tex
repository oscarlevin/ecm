\documentclass{book}

\input{../activities-preamble.tex}
\begin{document}
\setcounter{project}{27}
\addtocounter{project}{-1}
\begin{activity}[]\label{activity-22}
\leavevmode%
\begin{enumerate}[font=\bfseries,label=(\alph*),ref=\alph*]
\item\label{task-36} \hypertarget{p-283}{}%
The graph \(K_5\) has \(5\) vertices and 10 edges.  Explain why it would need to have \(7\) faces if it were planar.%
\par\smallskip%
\noindent\textbf{Solution.}\hypertarget{solution-19}{}\quad%
\hypertarget{p-284}{}%
The proof is by contradiction. So assume that \(K_5\) is planar. Then the graph must satisfy Euler's formula for planar graphs. \(K_5\) has 5 vertices and 10 edges, so we get%
\begin{equation*}
5 - 10 + f = 2
\end{equation*}
which says that if the graph is drawn without any edges crossing, there would be \(f = 7\) faces.%
\item\label{task-37} \hypertarget{p-285}{}%
Now get at the number of faces another way: each face must be bordered by at least three edges.  Why?  Explain why we can conclude that \(3f \le 2e\).  Where does the 2 come from?%
\par\smallskip%
\noindent\textbf{Solution.}\hypertarget{solution-20}{}\quad%
\hypertarget{p-286}{}%
Now consider how many edges surround each face. Each face must be surrounded by at least 3 edges. Let \(B\) be the total number of \emph{boundaries} around all the faces in the graph. Thus we have that \(B \ge 3f\). But also \(B = 2e\), since each edge is used as a boundary exactly twice. Putting this together we get%
\begin{equation*}
3f \le 2e
\end{equation*}
%
\item\label{task-38} \hypertarget{p-287}{}%
We now have that for \(K_5\), the number of faces is \(f = 7\), and also \(3f \le 20\).  How is this possible?  What can we conclude?%
\par\smallskip%
\noindent\textbf{Solution.}\hypertarget{solution-21}{}\quad%
\hypertarget{p-288}{}%
This is impossible, since we have already determined that \(f = 7\) and \(e = 10\), and \(21 \not\le 20\). This is a contradiction so in fact \(K_5\) is not planar.%
\end{enumerate}
\end{activity}
\end{document}
