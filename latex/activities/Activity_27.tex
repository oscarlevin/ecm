\documentclass{book}

\input{../activities-preamble.tex}
\begin{document}
\setcounter{cpjt}{27}
\addtocounter{cpjt}{-1}
\begin{activity}\label{activity-22}
\leavevmode%
\begin{enumerate}[font=\bfseries,label=(\alph*),ref=\alph*]
\item\label{task-34} \hypertarget{p-267}{}%
The graph \(K_5\) has \(5\) vertices and 10 edges.  Explain why it would need to have \(7\) faces if it were planar.%
\par\smallskip%
\noindent\item\label{task-35} \hypertarget{p-269}{}%
Now get at the number of faces another way: each face must be bordered by at least three edges.  Why?  Explain why we can conclude that \(3f \le 2e\).  Where does the 2 come from.%
\par\smallskip%
\noindent\item\label{task-36} \hypertarget{p-271}{}%
We now have that for \(K_5\), the number of faces is \(f = 7\), and also \(3f \le 20\).  How is this possible?  What can we conclude?%
\par\smallskip%
\noindent\end{enumerate}
\end{activity}

\clearpage\end{document}
