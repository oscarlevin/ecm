\documentclass{book}

\input{../activities-preamble.tex}
\begin{document}
\setcounter{project}{198}
\addtocounter{project}{-1}
\begin{activity}[]\label{act_stirlingcomputations}
\hypertarget{p-1105}{}%
Get to know the Stirling numbers by finding some.  List the set of partitions and count them.%
\begin{enumerate}[font=\bfseries,label=(\alph*),ref=\alph*]
\item\label{task-210} \hypertarget{p-1106}{}%
Find \(S(3,1)\), \(S(4,1)\) and \(S(k,1)\).%
\item\label{task-211} \hypertarget{p-1107}{}%
Compute \(S(2,2)\), \(S(3,2)\) and \(S(4,2)\).  Find a formula for \(S(k,2)\) and prove it is correct.%
\item\label{task-212} \hypertarget{p-1108}{}%
Compute \(S(3,3)\), and \(S(4,3)\).%
\item\label{task-213} \hypertarget{p-1109}{}%
Find formulas and give proofs for \(S(k,k)\) and \(S(k,k - 1)\).%
\par\smallskip%
\noindent\textbf{Hint.}\hypertarget{hint-125}{}\quad%
\hypertarget{p-1110}{}%
What are the possible sizes of parts?%
~\hfill{\tiny\hyperlink{a-198.d}{[hint]}\hypertarget{q-198.d}{}}\par\smallskip%
\noindent\textbf{Solution.}\hypertarget{solution-98}{}\quad%
\hypertarget{p-1111}{}%
If a partition has \(k-1\) parts, then one part has two elements, so once we choose those two elements from the \(k\) elements, we are done.  Therefore \(S(k,k-1) = \binom{k}{2}\).%
\end{enumerate}
\end{activity}
\end{document}
