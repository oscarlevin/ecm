\documentclass{book}

\input{../activities-preamble.tex}
\begin{document}
\setcounter{project}{143}
\addtocounter{project}{-1}
\begin{activity}[]\label{act-arithmetic}
\hypertarget{p-938}{}%
Let \((a_n)_{n \le 0}\) satisfy \(a_n = a_{n-1} + 3\) with \(a_0 = 2\).  For example, \(a_n\) might give the number of push-ups you can do \(n\) days into your training, assuming you can do 3 more push-ups each day, if you could do 2 push-ups before you started training.  Find a closed formula for \(a_n\) and justify your answer.  Show how the recurrence relation is used.%
\par\smallskip%
\noindent\textbf{Hint.}\hypertarget{hint-102}{}\quad%
\hypertarget{p-939}{}%
If you write out the sequence, it would be easy to guess at a closed formula, but the point here is to \emph{use} the recurrence relation somehow.  Consider rewriting the recurrence relation as \(a_n - a_{n-1} = 2\).  Note that \(a_n - a_0 = (a_n - a_{n-1}) + (a_{n-1} + a_{n-2}) + \cdots + (a_2 - a_1) + (a_1 - a_0)\).%
~\hfill{\tiny\hyperlink{a-143}{[hint]}\hypertarget{q-143}{}}\end{activity}

\clearpage\end{document}
