\documentclass{book}

\input{../activities-preamble.tex}
\begin{document}
\setcounter{project}{314}
\addtocounter{project}{-1}
\begin{activity}[]\label{activity-307}
\hypertarget{p-1618}{}%
Show that the number of partitions of \(k\) into distinct parts equals the number of partitions of \(k\) into odd parts.%
~\hfill{\tiny\hyperlink{a-314}{[hint]}\hypertarget{q-314}{}}\par\smallskip%
\noindent\textbf{Solution.}\hypertarget{solution-243}{}\quad%
\hypertarget{p-1620}{}%
We start by giving a function from the set of partitions of \(k\) to the set of partitions of \(k\) with (only) odd parts. Clearly such a function cannot be one to one. Then we show that when restricted to the partitions with distinct parts it is one-to-one and onto by constructing an inverse. Given a partition \(\lambda_1^{i_1}\lambda_2^{i_2}\cdots\lambda_n^{i_n}\), write \(\lambda_i=\gamma_i2^{k_i}\), where \(\gamma_i\) is odd. (Thus \(2^{k_i}\) is the highest power of 2 that is a factor of \(\lambda_i\), so it is 1 if \(\lambda_i\) is odd.). It is possible that \(\gamma_i=\gamma_j\), for example if \(\lambda_i=36\) and \(\lambda_j=18\), then \(\gamma_i=\gamma_j=9\). We construct a new partition \(\pi\) whose parts are the numbers \(\gamma_j\) as follows: Given an odd number \(p\), let the multiplicity \(m(p)\) of \(p\) in \(\pi\) be \(\sum_{j: \gamma_j=m} 2^{k_j}\). Thus \(\sum_{p: m(p)\not=0}m(p)p = k\). Therefore, \(\pi\) is a partition of \(k\) whose parts are all odd.%
\par
\hypertarget{p-1621}{}%
Now consider a partition \(\pi\) of \(k\) whose parts are all odd. Let \(\pi=\pi_1^{r_1}\pi_2^{r_2}\cdots \pi_t^{r_t}\), with \(\pi_i>\pi_{i+1}\). (In terms of the multiplicity function \(m\), \(m(\pi_i) =r_i\), and \(\sum_{i=1}^t r_i\pi_i = k\).) We are going to write the binary expansion of each \(r_i\) as \(r_i = \sum_{j= 0}^{\lfloor \log_2 r_i\rfloor} 2^{ja_{ij}}\), where \(a_{ij}\) is 1 if \(2^j\) appears in the binary expansion of \(r_i\), and 0 otherwise. All of the numbers \(\pi_i2^{ja_{ij}}\) are distinct, because a power of two times one odd number cannot equal a power of two times another odd number. The numbers \(\pi_i2^{ja_{ij}}\) add to \(k\), so they are the parts of a partition \(\pi'\) of \(k\) into distinct parts. When we apply the function constructed in the first part of the solution to \(\pi'\), we get \(\pi\), so the correspondence between \(\pi\) and \(\pi'\) is a bijection.%
\end{activity}
\end{document}
