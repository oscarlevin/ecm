\documentclass{book}

\input{../activities-preamble.tex}
\begin{document}
\setcounter{project}{90}
\addtocounter{project}{-1}
\begin{activity}[]\label{activity-83}
\hypertarget{p-663}{}%
Assuming \(k\le n\), in how many ways can we pass out \(k\) distinct pieces of fruit to \(n\) children if each child may get at most one? What is the number if \(k>n\)? Assume for both questions that we pass out all the fruit.%
\par\smallskip%
\noindent\textbf{Hint 1.}\hypertarget{hint-43}{}\quad%
\hypertarget{p-664}{}%
Ask yourself if either the sum principle or product principle applies.%
~\hfill{\tiny\hyperlink{a-90}{[hint]}\hypertarget{q-90}{}}\par\smallskip%
\noindent\textbf{Hint 2.}\hypertarget{hint-44}{}\quad%
\hypertarget{p-665}{}%
Remember that zero is a number.%
\par\smallskip%
\noindent\textbf{Solution.}\hypertarget{solution-48}{}\quad%
\hypertarget{p-666}{}%
We are asking for the number of \(k\)-element permutations of \(n\) children, which is \(\prod_{i=1}^k n-i+1\), and is zero if \(k>n\).%
\end{activity}

\clearpage\end{document}
