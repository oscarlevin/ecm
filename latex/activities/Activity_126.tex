\documentclass{book}

\input{../activities-preamble.tex}
\begin{document}
\setcounter{project}{126}
\addtocounter{project}{-1}
\begin{activity}[]\label{bookcaseeveryshelf}
\hypertarget{p-883}{}%
Suppose we wish to place the books in \hyperref[bookcase]{Activity~\ref{bookcase}} (satisfying the assumptions we made there) so that each shelf gets at least one book. Now in how many ways may we place the books? (Hint: how can you make sure that each shelf gets at least one book before you start the process described in \hyperref[bookcase]{Activity~\ref{bookcase}}?)%
\par\smallskip%
\noindent\textbf{Hint.}\hypertarget{hint-84}{}\quad%
\hypertarget{p-884}{}%
How can you make sure that each shelf gets at least one book before you start the process described in \hyperref[bookcase]{Activity~\ref{bookcase}}?%
~\hfill{\tiny\hyperlink{a-126}{[hint]}\hypertarget{q-126}{}}\par\smallskip%
\noindent\textbf{Solution.}\hypertarget{solution-84}{}\quad%
\hypertarget{p-885}{}%
Choose \(n\) books from the \(k\) books in \(\binom{k}{n}\) ways, and assign them to the \(n\) places shelves in \(n!\) ways, giving us \(k!/(k-n)!\) ways to put a book on each shelf. Now leaving these books at the far left of each shelf, place the remaining books in%
\begin{equation*}
\prod_{i=1}^{k-n}
(n+i-1)=\frac{(n+(k-n)-1)!}{(n-1)!}=\frac{(k-1)!}{(n-1)!}
\end{equation*}
ways. Thus we have%
\begin{equation*}
\frac{k!(k-1)!}{(k-n)!(n-1)!}=k!\binom{k-1}{n-1}
\end{equation*}
ways to place the books. Of course the right hand side of that equation cries out for a combinatorial explanation. Here it is. Imagine lining up the \(k\) books in a row. Then there are \(k-1\) places in between them. Choose \(n-1\) of these places, and slide a piece of paper in there as a divider. Now put the books before the first divider on shelf one, and the books after divider \(i\) on shelf \(i+1\). This gives an arrangement of the books on the shelves so that every shelf has a book!%
\end{activity}
\end{document}
