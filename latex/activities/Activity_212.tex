\documentclass{book}

\input{../activities-preamble.tex}
\begin{document}
\setcounter{project}{212}
\addtocounter{project}{-1}
\begin{activity}[]\label{activity-205}
\hypertarget{p-1185}{}%
Describe the relationship between partitions of \(k\) and lists or vectors \((x_1,x_2,\ldots,x_n)\) such that \(x_1+2x_2+\ldots kx_k = k\).  Such a representation of a partition is called a \terminology{type vector} representation of a partition, and it is typical to leave the trailing zeros out of such a representation; for example \((2,1)\) stands for the same partition as \((2,1,0,0)\). What is the decreasing list representation for this partition, and what number does it partition?\index{partition of an integer!type vector}\index{type vector for a partition of an integer}%
\end{activity}
\end{document}
