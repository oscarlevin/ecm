\documentclass{book}

\input{../activities-preamble.tex}
\begin{document}
\setcounter{project}{333}
\addtocounter{project}{-1}
\begin{investigation}[]\label{investigation-13}
\hypertarget{p-2439}{}%
Decide which of the following are valid proofs of the following statement:%
\begin{quote}\hypertarget{blockquote-13}{}
\hypertarget{p-2440}{}%
If \(a b\) is an even number, then \(a\) or \(b\) is even.%
\end{quote}
\hypertarget{p-2441}{}%
\leavevmode%
\begin{enumerate}
\item\hypertarget{li-504}{}\hypertarget{p-2442}{}%
Suppose \(a\) and \(b\) are odd. That is, \(a=2k+1\) and \(b=2m+1\) for some integers \(k\) and \(m\). Then%
\begin{align*}
ab \amp =(2k+1)(2m+1)\\
\amp =4km+2k+2m+1\\
\amp =2(2km+k+m)+1.
\end{align*}
%
\par
\hypertarget{p-2443}{}%
Therefore \(ab\) is odd.%
\item\hypertarget{li-505}{}\hypertarget{p-2444}{}%
Assume that \(a\) or \(b\) is even - say it is \(a\) (the case where \(b\) is even will be identical). That is, \(a=2k\) for some integer \(k\). Then%
\begin{align*}
ab \amp =(2k)b\\
\amp =2(kb).
\end{align*}
%
\par
\hypertarget{p-2445}{}%
Thus \(ab\) is even.%
\item\hypertarget{li-506}{}\hypertarget{p-2446}{}%
Suppose that \(ab\) is even but \(a\) and \(b\) are both odd. Namely, \(ab = 2n\), \(a=2k+1\) and \(b=2j+1\) for some integers \(n\), \(k\), and \(j\). Then%
\begin{align*}
2n \amp =(2k+1)(2j+1)\\
2n \amp =4kj+2k+2j+1\\
n \amp = 2kj+k+j+\frac{1}{2}.
\end{align*}
%
\par
\hypertarget{p-2447}{}%
But since \(2kj+k+j\) is an integer, this says that the integer \(n\) is equal to a non-integer, which is impossible.%
\item\hypertarget{li-507}{}\hypertarget{p-2448}{}%
Let \(ab\) be an even number, say \(ab=2n\), and \(a\) be an odd number, say \(a=2k+1\).%
\begin{align*}
ab \amp =(2k+1)b\\
2n \amp =2kb+b\\
2n-2kb\amp =b\\
2(n-kb)\amp =b.
\end{align*}
%
\par
\hypertarget{p-2449}{}%
Therefore \(b\) must be even.%
\end{enumerate}
%
\end{investigation}
\end{document}
