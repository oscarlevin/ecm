\documentclass{book}

\input{../activities-preamble.tex}
\begin{document}
\setcounter{project}{81}
\addtocounter{project}{-1}
\begin{activity}[]\label{activity-74}
\leavevmode%
\begin{enumerate}[font=\bfseries,label=(\alph*),ref=\alph*]
\item\label{task-117} \hypertarget{p-647}{}%
Explain why \hyperref[prop-unioncard]{Proposition~\ref{prop-unioncard}} is true.%
\item\label{task-118} \hypertarget{p-648}{}%
An incomplete deck of cards only has four face cards (the Jack, King and Queen of diamonds and the King of hearts) and five black cards (the 2 through 6 of spades).  A magician asks you to pick a card.  How many choices do you have?%
\item\label{task-119} \hypertarget{p-649}{}%
How exactly does the counting question above relate to \hyperref[prop-unioncard]{Proposition~\ref{prop-unioncard}}?  How exactly does it relate to the Sum Principle?  Illustrate both by listing specific outcomes.%
\item\label{task-120} \hypertarget{p-650}{}%
Make explicit the connection between the Sum Principle and \hyperref[prop-unioncard]{Proposition~\ref{prop-unioncard}}.  That is, justify the Sum Principle in terms of cardinality of sets.%
\end{enumerate}
\end{activity}
\end{document}
