\documentclass{book}

\input{../activities-preamble.tex}
\begin{document}
\setcounter{project}{21}
\addtocounter{project}{-1}
\begin{activity}[]\label{activity-16}
\hypertarget{p-228}{}%
What happens to the triple \((v,e,f)\) when you add an edge to a graph?%
\begin{enumerate}[font=\bfseries,label=(\alph*),ref=\alph*]
\item\label{task-22} \hypertarget{p-229}{}%
Start with a single pair of adjacent vertices (that is, the graph \(P_1\)).  What is \((v,e,f)\)?  What happens to the triple if you extend the path to \(P_2\)?  How do \(v\), \(e\), and \(f\) change?%
\item\label{task-23} \hypertarget{p-230}{}%
You could add another vertex and edge to \(P_2\) (which would either give \(P_3\) or a \emph{star} graph), or you could add an edge connecting two vertices not already adjacent.  What would these two operations do to \((v,e,f)\)?%
\item\label{task-24} \hypertarget{p-231}{}%
Generalize.  Say you know a specific triple \((a,b,c)\) describes a graph \(G\).  Give two new triples that also describe some graphs, each with one more edge than \(G\).%
\item\label{task-25} \hypertarget{p-232}{}%
Conjecture a relationship between \(v\), \(e\), and \(f\) that should hold for any connected planar graph.%
\end{enumerate}
\end{activity}
\end{document}
