\documentclass{book}

\input{../activities-preamble.tex}
\begin{document}
\setcounter{cpjt}{35}
\addtocounter{cpjt}{-1}
\begin{activity}\label{activity-30}
\hypertarget{p-318}{}%
Recall that a regular polyhedron has all of its faces identical regular polygons, and that each vertex has the same degree. Consider the cases, broken up by what the regular polygon might be.%
\begin{enumerate}[font=\bfseries,label=(\alph*),ref=\alph*]
\item\label{task-44} \hypertarget{p-319}{}%
Case 1: Each face is a triangle.  Let \(f\) be the number of faces and \(k\) the common degree of each vertex (since the polyhedron is regular).  Find formulas for the number of edges and vertices in terms of \(f\) and \(k\).  Conclude from these the only possible values for \(f\) and \(k\).%
\par\smallskip%
\noindent\item\label{task-45} \hypertarget{p-322}{}%
Case 2: each face is a square.  Again, consider the possible cases for \(f\) and \(k\) (and conclude there is only one: the cube).%
\par\smallskip%
\noindent\item\label{task-46} \hypertarget{p-326}{}%
Case 3: Each face is a pentagon.%
\par\smallskip%
\noindent\item\label{task-47} \hypertarget{p-329}{}%
Explain why it is not possible for each face to be a \(n\)-gon with \(n \ge 6\).%
\par\smallskip%
\noindent\end{enumerate}
\end{activity}

\clearpage\end{document}
