\documentclass{book}

\input{../activities-preamble.tex}
\begin{document}
\setcounter{project}{283}
\addtocounter{project}{-1}
\begin{activity}[]\label{activity-276}
\hypertarget{p-1475}{}%
Explain how to compute the number of functions from a \(k\)-element set \(K\) onto an \(n\)-element set \(N\) by using multinomial coefficients.%
\par\smallskip%
\noindent\textbf{Hint.}\hypertarget{hint-185}{}\quad%
\hypertarget{p-1476}{}%
How are the relevant \(j_i\)'s in the multinomial coefficients you use here different from the \(j_i\)'s in the previous problem.%
~\hfill{\tiny\hyperlink{a-283}{[hint]}\hypertarget{q-283}{}}\par\smallskip%
\noindent\textbf{Solution.}\hypertarget{solution-206}{}\quad%
\hypertarget{p-1477}{}%
Add the multinomial coefficients \(\binom{k}{j_1,j_2,\ldots,j_n}\) in which each \(j_i\) is different from zero. To see why, let \(N=\{y_1,y_2,\ldots,y_n\}\) and note that we are counting functions that send at least one element of \(K\) to each element \(y_i\).%
\end{activity}
\end{document}
