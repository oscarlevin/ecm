\documentclass{book}

\input{../activities-preamble.tex}
\begin{document}
\setcounter{project}{65}
\addtocounter{project}{-1}
\begin{activity}[]\label{activity-58}
\hypertarget{p-516}{}%
Recall that a \terminology{subset} \(A\) of a set \(B\) is a set all of whose elements are also elements of \(B\); we write \(A \subseteq B\).  For example, some of the subsets of \(B = \{1,2,3,4,5\}\) are \(\{2,4,5\}\), \(\{1,2,3,4,5\}\) and \(\emptyset\).  Recall also that the \terminology{cardinality} of a set is simply the number of elements in it.%
\par
\hypertarget{p-517}{}%
Since we will often consider sets of the form \(\{1,2,3,\ldots,n\}\), let's adopt the notation \([n]\) for this set.%
\begin{enumerate}[font=\bfseries,label=(\alph*),ref=\alph*]
\item\label{task-82} \hypertarget{p-518}{}%
Write out all subsets of \([3] = \{1,2,3\}\).  Then group the subsets by cardinality.  How many subsets have cardinality 0?  How many have cardinality 1?  2? 3?%
\item\label{task-83} \hypertarget{p-519}{}%
Write out all subsets of \([4]\), and determine how many have each possible cardinality.%
\item\label{task-84} \hypertarget{p-520}{}%
How many subsets of \([5]\) are there of each cardinality?  Try to answer this questions without writing out all 32 subsets.%
\end{enumerate}
\end{activity}
\end{document}
