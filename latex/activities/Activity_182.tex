\documentclass{book}

\input{../activities-preamble.tex}
\begin{document}
\setcounter{project}{182}
\addtocounter{project}{-1}
\begin{activity}[]\label{activity-175}
\hypertarget{p-1078}{}%
Recall that \(C_n\) gives the number of lattice paths from \((0,0)\) to \((n,n)\) that do not cross the line \(y = x\) (but they may touch this line).  We will compare this to all the lattice paths from \((0,0)\) to \((n,n)\).%
\begin{enumerate}[font=\bfseries,label=(\alph*),ref=\alph*]
\item\label{task-195} \hypertarget{p-1079}{}%
Explain why the number of lattice paths from \((0,0)\) to \((n,n)\) that \emph{do} cross the line \(y = x\) is the same as the number of lattice paths from \((0,0)\) to \((n,n)\) that touch or cross the line \(y = x + 1\).%
\item\label{task-196} \hypertarget{p-1081}{}%
Find a bijection between lattice paths from \((0,0)\) to \((n,n)\) that touch (or cross) the line \(y=x+1\) and lattice paths from \((-1,1)\) to \((n,n)\).%
~\hfill{\tiny\hyperlink{a-182.b}{[hint]}\hypertarget{q-182.b}{}}\item\label{task-197} \hypertarget{p-1084}{}%
Find a formula for the number of lattice paths from \((0,0)\) to \((n,n)\) that do not cross the line \(y=x\). That is, a formula for \(C_n\).%
~\hfill{\tiny\hyperlink{a-182.c}{[hint]}\hypertarget{q-182.c}{}}\end{enumerate}
\end{activity}
\end{document}
