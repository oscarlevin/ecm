\documentclass{book}

\input{../activities-preamble.tex}
\begin{document}
\setcounter{project}{182}
\addtocounter{project}{-1}
\begin{activity}[]\label{activity-175}
\hypertarget{p-1042}{}%
Recall that \(C_n\) gives the number of lattice paths from \((0,0)\) to \((n,n)\) that do not cross the line \(y = x\) (but they may touch this line).  We will compare this to all the lattice paths from \((0,0)\) to \((n,n)\).%
\begin{enumerate}[font=\bfseries,label=(\alph*),ref=\alph*]
\item\label{task-194} \hypertarget{p-1043}{}%
Explain why the number of lattice paths from \((0,0)\) to \((n,n)\) that \emph{do} cross the line \(y = x\) is the same as the number of lattice paths from \((0,0)\) to \((n,n)\) that touch or cross the line \(y = x + 1\).%
\par\smallskip%
\noindent\textbf{Solution.}\hypertarget{solution-85}{}\quad%
\hypertarget{p-1044}{}%
If a lattice path between \((0,0)\) and \((n,n)\) goes outside the triangle, it can only do so on an up-step. (A step from \((i,j)\) to \((i,j+1)\).) And an up-step must originate at a point with integer coordinates. If \(j\lt i\) an up-step from \((i,j))\) cannot leave the triangle. Thus to leave the triangle, the up-step must leave from a point of the form \((i,i)\), and go to \((i,i+1)\), which is on the line \(y=x+1\).%
\item\label{task-195} \hypertarget{p-1045}{}%
Find a bijection between lattice paths from \((0,0)\) to \((n,n)\) that touch (or cross) the line \(y=x+1\) and lattice paths from \((-1,1)\) to \((n,n)\).%
\par\smallskip%
\noindent\textbf{Hint.}\hypertarget{hint-122}{}\quad%
\hypertarget{p-1046}{}%
Given a path from \((0, 0)\) to \((n, n)\) which touches or crosses the line \(y = x + 1\), how can you modify the part of the path from \((0, 0)\) to the first touch of \(y = x + 1\) so that the modified path starts instead at \((-1, 1)\)? The trick is to do this in a systematic way that will give you your bijection.%
~\hfill{\tiny\hyperlink{a-182.b}{[hint]}\hypertarget{q-182.b}{}}\par\smallskip%
\noindent\textbf{Solution.}\hypertarget{solution-86}{}\quad%
\hypertarget{p-1047}{}%
Suppose we have a lattice path form \((0,0)\) to \((n,n)\) which touches or crosses the line \(y=x+1\). Let \((k,k+1)\) be the first point on the line \(y=x+1\) that the lattice path touches. From that point, work backwards, replacing every up-step with a step one unit to the left and every right-step with a step one unit down. The segment of the path you just changed will have moved left \(k+1\) times, so its leftmost \(x\) coordinate will be \(-1\), and it will have moved down \(k\) times, so its lowest \(y\) coordinate will be 1.  Thus we now have a lattice path from \((-1,1)\) to \((n,n)\). Further, given a lattice path from \((-1,1)\) to \((n,n)\), it must cross the line \(y=x+1\) at least once, because it starts above the line and ends below it. At the first point where such a path touches the line \(y=x+1\), say \((k',k'+1)\), work backwards replacing every up-step with a step to the left and every right-step with a step downward. The leftmost point on this path will have \(x\) coordinate 0, and the lowest point will have \(y\) coordinate 0, so the new path will be a lattice path from \((0,0)\) to \((n,n)\) that touches the line \(y=x+1\). Clearly these two processes reverse each other, and so they give us a bijection between paths form \((0,0)\) to \((n,n)\) that touch the line \(y=x+1\) and lattice lattice paths from \((-1,1)\) to \((n,n)\). Notice that geometrically what we are doing to get the bijection is to take the portion of a lattice path that goes from the initial point till the first touch of the line \(y=x+1\) and reflecting it around that line. This idea of reflection was introduced by Feller, and is called Feller's reflection principle.%
\item\label{task-196} \hypertarget{p-1048}{}%
Find a formula for the number of lattice paths from \((0,0)\) to \((n,n)\) that do not cross the line \(y=x\). That is, a formula for \(C_n\).%
\par\smallskip%
\noindent\textbf{Hint.}\hypertarget{hint-123}{}\quad%
\hypertarget{p-1049}{}%
A path either touches the line \(y = x + 1\) or it doesn't. This partitions the set of paths into two blocks.%
~\hfill{\tiny\hyperlink{a-182.c}{[hint]}\hypertarget{q-182.c}{}}\par\smallskip%
\noindent\textbf{Solution.}\hypertarget{solution-87}{}\quad%
\hypertarget{p-1050}{}%
\(C_n=\binom{2n}{n} - \binom{2n}{n+1}=\frac{1}{n+1}\binom{2n}{n}.\)%
\end{enumerate}
\end{activity}

\clearpage\end{document}
