\documentclass{book}

\input{../activities-preamble.tex}
\begin{document}
\setcounter{project}{47}
\addtocounter{project}{-1}
\begin{activity}[]\label{activity-40}
\hypertarget{p-428}{}%
Six friends decide to spend the afternoon playing chess. Everyone will play everyone else once. They have plenty of chess sets but nobody wants to play more than one game at a time. Games will last an hour (thanks to their handy chess clocks). How many hours will the tournament last?%
\par\smallskip%
\noindent\textbf{Solution.}\hypertarget{solution-46}{}\quad%
\hypertarget{p-429}{}%
Represent each player with a vertex and put an edge between two players if they will play each other. In this case, we get the graph \(K_6\):%
\begin{sidebyside}{1}{0.4}{0.4}{0}
\begin{sbspanel}{0.2}
\resizebox{\linewidth}{!}{{
\begin{tikzpicture}
	\foreach \x in {0,...,5}{

		\foreach \y in {1,...,5}{
		 \draw (\x*60:1) -- (\x*60+\y*60:1);
		}
	\draw (\x*60:1) \v;
	}
	\end{tikzpicture}
}
}
\end{sbspanel}
\end{sidebyside}
\par
\hypertarget{p-430}{}%
We must color the edges; each color represents a different hour. Since different edges incident to the same vertex will be colored differently, no player will be playing two different games (edges) at the same time. Thus we need to know the chromatic index of \(K_6\).%
\par
\hypertarget{p-431}{}%
Notice that for sure \(\chi'(K_6) \ge 5\), since there is a vertex of degree 5. It turns out 5 colors is enough (go find such a coloring). Therefore the friends will play for 5 hours.%
\end{activity}
\end{document}
