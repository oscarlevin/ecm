\documentclass{book}

\input{../activities-preamble.tex}
\begin{document}
\setcounter{cpjt}{315}
\addtocounter{cpjt}{-1}
\begin{activity}\label{activity-308}
\hypertarget{p-1555}{}%
Euler showed that if \(k\not= \frac{3j^2+j}{2}\), then the number of partitions of \(k\) into an even number of distinct parts is the same as the number of partitions of \(k\) into an odd number of distinct parts. Prove this, and in the exceptional case find out how the two numbers relate to each other.%
\par\smallskip%
\noindent\textbf{Hint}.\hypertarget{hint-203}{}\quad%
\hypertarget{p-1556}{}%
Suppose we have a partition of \(k\) into distinct parts. If the smallest part, say \(m\), is smaller than the number of parts, we may add one to each of the \(m\) largest parts and delete the smallest part, and we have changed the parity of the number of parts, but we still have distinct parts. On the other hand, suppose the smallest part, again say \(m\), is larger than or equal to the number of parts. Then we can subtract 1 from each part larger than \(m\), and add a part equal to the number of parts larger than \(m\). This changes the parity of the number of parts, but if the second smallest part is \(m+1\), the resulting partition does not have distinct parts. Thus this method does not work. Further, if it did always work, the case \(k \ne \frac{3j^2+j}{2}\) would be covered also. However you can modify this method by comparing \(m\) not to the total number of parts, but to the number of rows at the top of the Young diagram that differ by exactly one from the row above. Even in this situation, there are certain slight additional assumptions you need to make, so this hint leaves you a lot of work to do. (It is reasonable to expect problems because of that exceptional case.) However, it should lead you in a useful direction.%
\par\smallskip%
\noindent\end{activity}

\clearpage\end{document}
