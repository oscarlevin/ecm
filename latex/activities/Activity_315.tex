\documentclass{book}

\input{../activities-preamble.tex}
\begin{document}
\setcounter{project}{315}
\addtocounter{project}{-1}
\begin{activity}[]\label{activity-308}
\hypertarget{p-1596}{}%
Euler showed that if \(k\not= \frac{3j^2+j}{2}\), then the number of partitions of \(k\) into an even number of distinct parts is the same as the number of partitions of \(k\) into an odd number of distinct parts. Prove this, and in the exceptional case find out how the two numbers relate to each other.%
\par\smallskip%
\noindent\textbf{Hint.}\hypertarget{hint-203}{}\quad%
\hypertarget{p-1597}{}%
Suppose we have a partition of \(k\) into distinct parts. If the smallest part, say \(m\), is smaller than the number of parts, we may add one to each of the \(m\) largest parts and delete the smallest part, and we have changed the parity of the number of parts, but we still have distinct parts. On the other hand, suppose the smallest part, again say \(m\), is larger than or equal to the number of parts. Then we can subtract 1 from each part larger than \(m\), and add a part equal to the number of parts larger than \(m\). This changes the parity of the number of parts, but if the second smallest part is \(m+1\), the resulting partition does not have distinct parts. Thus this method does not work. Further, if it did always work, the case \(k \ne \frac{3j^2+j}{2}\) would be covered also. However you can modify this method by comparing \(m\) not to the total number of parts, but to the number of rows at the top of the Young diagram that differ by exactly one from the row above. Even in this situation, there are certain slight additional assumptions you need to make, so this hint leaves you a lot of work to do. (It is reasonable to expect problems because of that exceptional case.) However, it should lead you in a useful direction.%
~\hfill{\tiny\hyperlink{a-315}{[hint]}\hypertarget{q-315}{}}\par\smallskip%
\noindent\textbf{Solution.}\hypertarget{solution-228}{}\quad%
\hypertarget{p-1598}{}%
This solution is taken largely from the book \textsl{Introduction to Combinatorics} by Ioan Tomescu (published in London by Collet's in 1975). Tomescu calls a collection of rows in a Young diagram a ``trapezoid'' if each row contains one less cell than the row above and the number of cells in the rows above and below the trapezoid differ by two or more from the number of cells in rows of the trapezoid. Thus in (8,6,5,4,2,1) we have 3 trapezoids, the first row, the next three rows, and the last two. Since we are dealing with partitions with distinct parts, we don't have to worry about how two equal rows affect the definition of a trapezoid. We will describe a way to transform a partition with an even number of distinct parts into a partition with an odd number of distinct parts and vice versa.%
\par
\hypertarget{p-1599}{}%
First we describe a transformation on Young diagrams. Here is the first part of the description. Suppose the smallest part \(m\) of \(\lambda\) is less than or equal to the number \(j\) of rows in the top trapezoid. Suppose further that if we have only one trapezoid, then \(j>m\). Then we construct a partition with one less part by adding 1 to each of the \(m\) largest parts and discarding the part \(m\). We still have a diagram for a partition of the same integer, but now the parity of the number of parts has changed, and we \emph{may} have increased the number of trapezoids by 1. The smallest part will now be larger than the number (now \(m\)) of rows in the top trapezoid. (Notice that the construction would not work if we had only one trapezoid and \(j=m\) because we would first remove one row of the trapezoid and thus have no row to which to attach one of our squares.)%
\par
\hypertarget{p-1600}{}%
Here is the second part of the description of the transformation. Suppose now that \(m\) is larger than the number \(j\) of rows of the top trapezoid in the Young diagram. Suppose also that the Young diagram has at least two trapezoids or it has one trapezoid and \(j\ge m-2\). Take one square from each of the \(j\) rows of the top trapezoid (which is the whole diagram if there is only one trapezoid) and also add a row of \(j\) squares at the bottom of the diagram. (Since \(m>j\), this gives us a Young diagram of a partition of the same integer into distinct parts.) The parity of the number of rows has changed, and now the number of rows of the top trapezoid is at least as large as the smallest part of the partition. (Note, two previously distinct trapezoids may have joined together to form one on top.)  (Notice that if we have one trapezoid and \(j= m+1\), then the construction yields a partition with two equal parts, which is why we made the special assumption above.) Now let \(T\) be the transformation described by the two constructions above. Its domain is all Young diagrams except those with one trapezoid and \(m\le j\le m+1\). \(T^2\) is the identity, and so \(T\) is a bijection.  When restricted to partitions with an odd number of parts, \(T\) gives partitions with an even number of parts, so on its domain it gives a bijection between partitions with an even number of parts and partitions with an odd number of parts.%
\par
\hypertarget{p-1601}{}%
If \(m=j\) and the diagram has just one trapezoid, then the diagram has \(\frac{3j^2-j}{2}\) squares, and if \(m=j+1\) and the diagram has just one trapezoid, then the diagram has \(\frac{3j^2+j}{2}\) squares. Thus if \(k\ne \frac{3j^2\pm j}{2}\), the number of partitions of \(k\) into distinct even parts equals the number of partitions of \(k\) into distinct odd parts.%
\par
\hypertarget{p-1602}{}%
If \(k= \frac{3j^2\pm j}{2}\) and \(j\) is even, then there is one diagram of a partition of \(k\) that is not in the domain of the bijection and has an even number of rows, so in this case there will be one more partition with an even number of parts than with an odd number. If \(k= \frac{3j^2\pm j}{2}\) and \(j\) is odd, there is one diagram with an odd number of rows not in the domain and so in this case there is one more partition with an odd number of parts than with an even number. This completes the exceptional cases of the problem.%
\end{activity}
\end{document}
