\documentclass{book}

\input{../activities-preamble.tex}
\begin{document}
\setcounter{cpjt}{52}
\addtocounter{cpjt}{-1}
\begin{activity}\label{Ramseyrecurrence}
\hypertarget{p-419}{}%
Prove that \(R(m,n)\le R(m-1,n) + R(m,n-1)\).%
\par\smallskip%
\noindent\textbf{Hint}.\hypertarget{hint-22}{}\quad%
\hypertarget{p-420}{}%
What you need to show is that if there are \(R(m - 1, n) + R(m, n - 1)\) people in a room, then there are either \(m\) mutual acquaintances or \(n\) mutual strangers. As with earlier problems, it helps to start with a person and think about the number of people with whom this person is acquainted or nonacquainted. The generalized pigeonhole principle tells you something about these numbers.%
\par\smallskip%
\noindent\end{activity}

\clearpage\end{document}
