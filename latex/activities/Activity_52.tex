\documentclass{book}

\input{../activities-preamble.tex}
\begin{document}
\setcounter{project}{52}
\addtocounter{project}{-1}
\begin{activity}[]\label{Ramseyrecurrence}
\hypertarget{p-447}{}%
Prove that \(R(m,n)\le R(m-1,n) + R(m,n-1)\).%
\par\smallskip%
\noindent\textbf{Hint.}\hypertarget{hint-22}{}\quad%
\hypertarget{p-448}{}%
What you need to show is that if there are \(R(m - 1, n) + R(m, n - 1)\) people in a room, then there are either \(m\) mutual acquaintances or \(n\) mutual strangers. As with earlier problems, it helps to start with a person and think about the number of people with whom this person is acquainted or nonacquainted. The generalized pigeonhole principle tells you something about these numbers.%
~\hfill{\tiny\hyperlink{a-52}{[hint]}\hypertarget{q-52}{}}\par\smallskip%
\noindent\textbf{Solution.}\hypertarget{solution-49}{}\quad%
\hypertarget{p-449}{}%
If there are \(R(m-1,n) +R(m,n-1)\) people in a room, choose one person, say person \(P\). By the generalized pigeonhole principle, there are either \(R(m-1,n)\) people with whom \(P\) is acquainted or \(R(m,n-1)\) people with whom person \(P\) is unacquainted. In the first case, among the people with whom person \(P\) is acquainted, there are either \(n\) mutual strangers, in which case we are done, or there are \(m-1\) people with whom person \(P\) is acquainted. These \(m-1\) people and person \(P\) form \(m\) people who are mutually acquainted, and so we have \(m\) mutual acquaintances. On the other hand, if \(P\) is unacquainted with \(R(m,n-1)\) people, then among these people, there are either \(m\) mutually acquainted people, in which case we are done, or among these people there are \(m-1\) mutually unacquainted people, and these \(m-1\) people together with \(P\) make \(m\) mutual strangers. Thus in every case, if there are \(R(m-1,n)+R(m,n-1)\) people in a room, there are either at least \(m\) mutual acquaintances or at least \(n\) mutual strangers. Therefore \(R(m,n)\le R(m-1,n)+R(m,n-1)\).%
\end{activity}
\end{document}
