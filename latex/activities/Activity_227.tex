\documentclass{book}

\input{../activities-preamble.tex}
\begin{document}
\setcounter{project}{227}
\addtocounter{project}{-1}
\begin{activity}[]\label{activity-220}
\hypertarget{p-1279}{}%
A group of \(n\) married couples comes to a group discussion session where they all sit around a round table. In how many ways can they sit so that no person is next to his or her spouse or a person of the same sex? This problem is called the \terminology{menage problem}.\index{menage problem}%
~\hfill{\tiny\hyperlink{a-227}{[hint]}\hypertarget{q-227}{}}\par\smallskip%
\noindent\textbf{Solution.}\hypertarget{solution-163}{}\quad%
\hypertarget{p-1282}{}%
We are going to consider arrangements of the couples alternating sex around the table. This will be our set \(A\). The set \(A_i\) is the set of arrangements in which couple \(i\) sits together. We are interested in the number of arrangements that are in none of these sets. Thus for each subset \(S\) of \([n]\), we consider the number of arrangements in \(\displaystyle\bigcap_{i\colon i\in S} A_i\). We distinguish the case that \(S\) is empty from the others. The number of arrangements with \(S\) empty is just the number of ways to seat \(2n\) couples around the table, alternating sex, but with no other restrictions. We can arrange one of the sexes in a circle in \(n-1!\) ways and then assign the members of the opposite sex to the places between them in \(n!\) ways, so \(\displaystyle \left|\bigcap_{i\colon i\in \emptyset} A_i \right| = (n-1)!n!\). (Another way to get this result is to let one person sit down. This determines the sex of the person at each place of the table, so there are \((n-1)!\) ways to assign the people of the same sex of the first person, and \(n!\) ways to assign the people of the opposite. It appears that there are \(2n\) choices for where the first person sits, but we can break the seating charts up into blocks of \(2n\) seating charts, each of which gives the same circular arrangement. Thus there are \((n-1)!n!\) inequivalent seating arrangments.)%
\par
\hypertarget{p-1283}{}%
Now if \(S\) is nonempty, and has \(s\) members, we seat one of the couples that must sit together (say the first in alphabetical order), and this determines the sex of the person that must sit at each other place. There are \(2n\) pairs of adjacent seats where we can seat that couple and two ways they can sit in the pair of adjacent seats that we choose. Then we have \(s-1\) couples, \(n-s\) men and \(n-s\) women to seat in the remaining places. First we arrange the \(s-1\) couples and \(2n-2s\) identical empty chairs in places at the table in \((2n-2s+s-1)!/(2n-2s)!=(2n-s-1)!/(2n-2s)!\) ways. Each couple can sit in only one way in the places they have chosen, because the sex of the person in a given place has been determined by how the first couple sits. The sex of the person in each of the remaining chairs has been determined, so we assign the men to their seats in \((n-s)!\) ways and we assign the women to their seats in \((n-s)!\) ways. Thus we have \(2\cdot2n(2n-s-1)!(n-s)!^2/(2n-2s)!\) ways to place the people. But we can partition the placements into blocks of \(2n\) equivalent placements, because shifting everyone the same number of places to the right or left gives an equivalent placement. Thus the number of inequivalent seating arrangements is%
\begin{align*}
\frac{2(2n-s-1)!(n-s)!^2}{(2n-2s)!} =\amp \frac{2(2n-s-1)!(n-s)!^2}{2(n-s)(2n-2s-1)!}\\
=\amp \frac{(2n-s-1)!(n-s)!(n-s-1)!}{(2n-2s-1)!}.
\end{align*}
%
\par
\hypertarget{p-1284}{}%
Notice that if we take \(s=0\), this formula reduces to \((n-1)!n!\). Thus for all sets \(S\)%
\begin{equation*}
\left|\bigcap_{i\colon i\in S} A_i\right|=\frac{(2n-s-1)!(n-s)!(n-s-1)!}{(2n-2s-1)!}\text{.}
\end{equation*}
%
\par
\hypertarget{p-1285}{}%
Then from \hyperref[compunion]{Activity~\ref{compunion}}%
\begin{align*}
\left|\overline{\bigcup_{i=1}^n A_i}\right|  &= \sum_{S:S\subseteq [n]} (-1)^{|S|}\frac{(2n-|S|-1)!(n-|S|)!(n-|S|-1)!
}{(2n-2|S|-1)!}\\
=\amp \sum_{s=0}^n(-1)^s\binom{n}{s}\frac{(2n-s-1)!(n-s)!(n-s-1)!}{(2n-2s-1)!}\\
=\amp \sum_{s=0}^n(-1)^s\frac{n!}{s!(n-s)!}\frac{(2n-s-1)!(n-s)!(n-s-1)!}{(2n-2s-1)!}\\
=\amp \sum_{s=0}^n(-1)^s \frac{n!(2n-s-1)!(n-s-1)!}{s!(2n-2s-1)!}\\
=\amp \sum_{s=0}^n(-1)^s\binom{2n-s-1}{s}n!(n-s-1)!
\end{align*}
is the number of ways to seat the people, alternating sex, so that no couple sits together.%
\end{activity}
\end{document}
