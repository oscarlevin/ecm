\documentclass{book}

\input{../activities-preamble.tex}
\begin{document}
\setcounter{cpjt}{227}
\addtocounter{cpjt}{-1}
\begin{activity}\label{activity-220}
\hypertarget{p-1212}{}%
A group of \(n\) married couples comes to a group discussion session where they all sit around a round table. In how many ways can they sit so that no person is next to his or her spouse or a person of the same sex? This problem is called the \terminology{menage problem}.\index{menage problem}%
\par\smallskip%
\noindent\textbf{Hint 1}.\hypertarget{hint-148}{}\quad%
\hypertarget{p-1213}{}%
Reason somewhat as you did in \hyperref[relaxedmenage]{Activity~\ref{relaxedmenage}}, noting that if the set of couples who do sit side-by-side is nonempty, then the sex of the person at each place at the table is determined once we seat one couple in that set.%
\par\smallskip%
\noindent\textbf{Hint 2}.\hypertarget{hint-149}{}\quad%
\hypertarget{p-1214}{}%
Think in terms of the sets \(A_i\) of arrangements of people in which couple \(i\) sits side-by-side. What does the union of the sets \(A_i\) have to do with the problem?%
\par\smallskip%
\noindent\end{activity}

\clearpage\end{document}
