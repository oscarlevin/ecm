\documentclass{book}

\input{../activities-preamble.tex}
\begin{document}
\setcounter{project}{268}
\addtocounter{project}{-1}
\begin{activity}[]\label{fractionFibonacci}
\hypertarget{p-1434}{}%
Use the factors you found in \hyperref[factorFibonacci]{Activity~\ref{factorFibonacci}} to write%
\begin{equation*}
\frac{1}{x^2+x-1}
\end{equation*}
in the form%
\begin{equation*}
\frac{c}{x-r_1} + \frac{d}{x-r_2}.
\end{equation*}
%
~\hfill{\tiny\hyperlink{a-268}{[hint]}\hypertarget{q-268}{}}\par\smallskip%
\noindent\textbf{Solution.}\hypertarget{solution-205}{}\quad%
\hypertarget{p-1436}{}%
\(\frac{1}{x^2+x-1}=\frac{c}{x-r_1}+\frac{d}{x-r_2}\) gives us \(cx-cr_2+dx-dr_1=1\). Thus \(c+d=0\), and \(cr_2+dr_1 =-1\). This gives us \(d=-c\) and so \(cr_2-cr_1=-1\), which yields \(c=\frac{1}{r_1-r_2}\), and \(d=\frac{1}{r_2-r_1}\). By substitution, \(c=1/\sqrt{5}\) and \(d=-1/\sqrt{5}\). This gives us%
\begin{equation*}
\frac{1}{x^2+x-1} = \frac{1/\sqrt{5}}{x-\frac{-1+\sqrt{5}}{2}}
+ \frac{-1/\sqrt{5}}{x- \frac{-1-\sqrt{5}}{2}}\text{.}
\end{equation*}
%
\end{activity}
\end{document}
