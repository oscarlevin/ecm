\documentclass{book}

\input{../activities-preamble.tex}
\begin{document}
\setcounter{cpjt}{268}
\addtocounter{cpjt}{-1}
\begin{activity}\label{fractionFibonacci}
\hypertarget{p-1367}{}%
Use the factors you found in \hyperref[factorFibonacci]{Activity~\ref{factorFibonacci}} to write%
\begin{equation*}
\frac{1}{x^2+x-1}
\end{equation*}
in the form%
\begin{equation*}
\frac{c}{x-r_1} + \frac{d}{x-r_2}.
\end{equation*}
%
\par\smallskip%
\noindent\textbf{Hint}.\hypertarget{hint-171}{}\quad%
\hypertarget{p-1368}{}%
You can save yourself a tremendous amount of frustrating algebra if you arbitrarily choose one of the solutions and call it \(r_1\) and call the other solution \(r_2\) and solve the problem using these algebraic symbols in place of the actual roots.\footnote{We use the words roots and solutions interchangeably.\label{fn-18}} Not only will you save yourself some work, but you will get a formula you could use in other problems. When you are done, substitute in the actual values of the solutions and simplify.%
\par\smallskip%
\noindent\end{activity}

\clearpage\end{document}
