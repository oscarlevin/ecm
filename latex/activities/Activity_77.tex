\documentclass{book}

\input{../activities-preamble.tex}
\begin{document}
\setcounter{project}{77}
\addtocounter{project}{-1}
\begin{activity}[]\label{activity-70}
\hypertarget{p-610}{}%
We wish to establish this identity for all natural numbers \(n\), so it would be natural to give a proof by induction.  Do this.%
\par\smallskip%
\noindent\textbf{Hint.}\hypertarget{hint-33}{}\quad%
\hypertarget{p-611}{}%
Again, use the Pascal recurrence \(\binom{n}{k} = \binom{n-1}{k-1} + \binom{n-1}{k}\).  Doing this for all summands will give you some repeats.%
~\hfill{\tiny\hyperlink{a-77}{[hint]}\hypertarget{q-77}{}}\par\smallskip%
\noindent\textbf{Solution.}\hypertarget{solution-59}{}\quad%
\hypertarget{p-612}{}%
The base case is trivial: \(\binom{0}{0} = 1 = 2^0\).  Now fix \(n \gt 0\) and assume \(\sum_{k=0}^{n-1} \binom{n-1}{k} 2^{n-1}\).%
\par
\hypertarget{p-613}{}%
Consider the case for \(n\).  This is easier to see when written out.%
\begin{align*}
\binom{n}{0} + \amp \binom{n}{1} + \amp \binom{n}{2} \cdots + \binom{n}{n-1}  + \amp \binom{n}{n} = \\
\binom{n-1}{0} + \amp \binom{n-1}{0} + \binom{n-1}{1} + \amp \binom{n-1}{1} + \binom{n-1}{2} + \cdots + \binom{n-1}{n-2} + \binom{n-1}{n-1} + \amp \binom{n-1}{n-1}.
\end{align*}
We have replaced \(\binom{n}{0}\) with \(\binom{n-1}{0}\) and \(\binom{n}{n}\) with \(\binom{n-1}{n-1}\) since they are all 1.  Every other term is replaced using the Pascal recurrence.%
\par
\hypertarget{p-614}{}%
But notice now that for every \(k\), we have \(\binom{n-1}{k}\) appearing exactly twice.  So if we group like terms and factor out a 2, we get%
\begin{equation*}
\sum_{k=0}^n \binom{n}{k} = 2 \sum_{k=0}^{n-1} \binom{n-1}{k} = 2 \cdot 2^{n-1} = 2^n.
\end{equation*}
This completes the inductive case, and therefore the proof.%
\end{activity}
\end{document}
