\documentclass{book}

\input{../activities-preamble.tex}
\begin{document}
\setcounter{project}{111}
\addtocounter{project}{-1}
\begin{activity}[]\label{necklace}
\hypertarget{p-806}{}%
In how many ways may we string \(n\) distinct beads on a necklace without a clasp? (Perhaps we make the necklace by stringing the beads on a string, and then carefully gluing the two ends of the string together so that the joint can't be seen. Assume someone can pick up the necklace, move it around in space and put it back down, giving an apparently different way of stringing the beads that is equivalent to the first.)%
~\hfill{\tiny\hyperlink{a-111}{[hint]}\hypertarget{q-111}{}}\par\smallskip%
\noindent\textbf{Solution.}\hypertarget{solution-84}{}\quad%
\hypertarget{p-809}{}%
There are \(n!\) permutations (lists, sequences) of the beads.  By gluing the ends together we get \(n!\) necklaces, but many of these have the same design as each other.  The question is, for a given design, how many of the necklaces formed from the \(n!\) permutations have that design?%
\par
\hypertarget{p-810}{}%
We can partition the permutations according to which necklace designs they create. Two permutations are in the same block if we get one either by circularly permuting the other or by reversing the other (this corresponds to flipping the necklace over in space). Thus each necklace corresponds to \(2n\) permutations so by the quotient principle we have \(n!/2n=(n-1)!/2\) ways to string \(n\) distinct beads on a necklace.%
\end{activity}
\end{document}
