\documentclass{book}

\input{../activities-preamble.tex}
\begin{document}
\setcounter{project}{97}
\addtocounter{project}{-1}
\begin{activity}[]\label{activity-90}
\hypertarget{p-729}{}%
Prove that the number of positive integers that have their digits in strictly increasing order is \(2^{9} - 1\). Include single digit numbers.%
~\hfill{\tiny\hyperlink{a-97}{[hint]}\hypertarget{q-97}{}}\par\smallskip%
\noindent\textbf{Solution.}\hypertarget{solution-71}{}\quad%
\hypertarget{p-731}{}%
There are \(\binom{9}{1}\) single digit type, \(\binom{9}{2}\) double digit type (just select 2 of the 9 digits \(1, 2, 3, \ldots, 9\) and arrange in order), \textellipsis{}, and so on to see that there are \(\binom{9}{9}\) nine digit type. The total is \(\binom{9}{1} + \binom{9}{2} + \binom{9}{3} + \ldots + \binom{9}{9} = 2^{9} - 1\).%
\par
\hypertarget{p-732}{}%
Here is an alternative, more clever, proof. Look at 123456789. Any increasing number can be made by deleting \emph{any} subset of digits, except all of them. There are \(2^{9} - 1\) such subsets. For example, delete the subset \{2, 4, 7, 9\} and you get 13568. Combining these two approaches actually gives you a nice proof that \(\binom{9}{0} + \binom{9}{1} + \binom{9}{2} + \ldots + \binom{9}{9} = 2^{9}\).%
\end{activity}
\end{document}
