\documentclass{book}

\input{../activities-preamble.tex}
\begin{document}
\setcounter{project}{192}
\addtocounter{project}{-1}
\begin{activity}[]\label{activity-185}
\hypertarget{p-1119}{}%
Consider the graph \(P_n\) (a path with \(n\) edges and \(n+1\) vertices).  Call one of the endpoints of the path \(v\).  How many walks of length \(2n\) start and stop at \(v\)?  Recall a walk in a graph is a sequence of adjacent vertices (think of tracing along edges of the graph), that does allow for repeated vertices.%
\par\smallskip%
\noindent\textbf{Solution.}\hypertarget{solution-122}{}\quad%
\hypertarget{p-1120}{}%
We can describe a walk starting at \(v\) as a Dyck word: use \(a\) for each step away from \(v\) and \(t\) for each stem toward \(v\).  Since \(v\) is an endpoint of the path, no initial segment of this path can contain more \(t\)'s than \(a\)'s, so every path is a unique Dyck word.  Similarly, given any Dyck word on these symbols, we have a unique path starting and ending at \(v\) (there are \(n\) \(a\)'s in our words, so the farthest away you can go from \(v\) is \(a\) steps away, and the path has length \(a\)).%
\end{activity}
\end{document}
