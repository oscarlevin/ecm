\documentclass{book}

\input{../activities-preamble.tex}
\begin{document}
\setcounter{cpjt}{116}
\addtocounter{cpjt}{-1}
\begin{activity}\label{activity-109}
\hypertarget{p-789}{}%
Each counting question below asks for two answers.  Decide which answer is a combination and which is a permutation, and why that makes sense.%
\begin{enumerate}[font=\bfseries,label=(\alph*),ref=\alph*]
\item\label{task-147} \hypertarget{p-790}{}%
An ice-cream shop offers 31 flavors.  How many 3-scoop ice-cream cones are possible, assuming each scoop must be a different flavor?  How many 3-scoop milkshakes are possible, assuming each scoop must be a different flavor?%
\par\smallskip%
\noindent\textbf{Hint}.\hypertarget{hint-73}{}\quad%
\hypertarget{p-791}{}%
This is not intended to be a trick question.  In fact, this example would be a good one for thinking about how permutations and combinations are related in general.%
\item\label{task-148} \hypertarget{p-792}{}%
How many 5-digit numbers are there with distinct, non-zero digits for which the digits must be increasing?  How many are there for which the digits can come in any order?%
\par\smallskip%
\noindent\textbf{Hint}.\hypertarget{hint-74}{}\quad%
\hypertarget{p-793}{}%
The point of this question is to push back against your conception of what ``order matters'' means.  Since you know the answers must be \(\binom{9}{5}\) or \(P(9,5)\), you should be able to answer correctly by deciding which set is bigger.%
\item\label{task-149} \hypertarget{p-794}{}%
How many injective functions \(f:[k] \to [n]\) are there all   together?  How many injective functions \(f:[k] \to [n]\) are there that are (strictly) increasing?%
\par\smallskip%
\noindent\textbf{Hint}.\hypertarget{hint-75}{}\quad%
\hypertarget{p-795}{}%
You might as well assume that \(k \le n\) (otherwise the answers would both be 0).  If you are stuck, write out some examples of each using two-line notation for the functions.  What decisions do you need to make?%
\end{enumerate}
\end{activity}

\clearpage\end{document}
