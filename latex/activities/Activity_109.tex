\documentclass{book}

\input{../activities-preamble.tex}
\begin{document}
\setcounter{project}{109}
\addtocounter{project}{-1}
\begin{activity}[]\label{activity-102}
\hypertarget{p-765}{}%
Refer back to your list of 60 3-permutations of \(\{a,b,c,d,e\}\).%
\begin{enumerate}[font=\bfseries,label=(\alph*),ref=\alph*]
\item\label{task-144} \hypertarget{p-766}{}%
Define an \terminology{equivalence relation} on the permutations you listed so that permutations that ``correspond'' to the same subset are equivalent.  That is, give a rule that specifies when two permutations are ``equivalent''.%
\par\smallskip%
\noindent\textbf{Hint.}\hypertarget{hint-61}{}\quad%
\hypertarget{p-767}{}%
One rule of this type would be, two permutations are equivalent if they start with the same letter.  This is not the one you want though.%
~\hfill{\tiny\hyperlink{a-109.a}{[hint]}\hypertarget{q-109.a}{}}\item\label{task-145} \hypertarget{p-768}{}%
In any set \(S\), if you have an equivalence relation \(\sim\), you can \terminology{partition} \(S\) into \terminology{equivalence classes}: sets of elements that are equivalent under \(\sim\) (i.e., sets of the form \(\{x \in S \st x \sim a\} \) for a particular element \(a\)).%
\par
\hypertarget{p-769}{}%
Write out the equivalence classes generated by the equivalence relation you gave above.  Explain why these all have the same size.  How many equivalence classes do you have (and how does this relate to the fact that they all have the same size)?%
\item\label{task-146} \hypertarget{p-770}{}%
Find a bijection between the set of equivalence classes and the set of subsets of \(\{a,b,c,d,e\}\).  Why is this important?%
\par\smallskip%
\noindent\textbf{Hint.}\hypertarget{hint-62}{}\quad%
\hypertarget{p-771}{}%
You might think about the usual way you would write a subset.  Essentially what this question is asking is for you to pick a representative for each equivalence class.%
~\hfill{\tiny\hyperlink{a-109.c}{[hint]}\hypertarget{q-109.c}{}}\end{enumerate}
\end{activity}

\clearpage\end{document}
