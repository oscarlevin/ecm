\documentclass{book}

\input{../activities-preamble.tex}
\begin{document}
\setcounter{project}{129}
\addtocounter{project}{-1}
\begin{activity}[]\label{activity-122}
\hypertarget{p-893}{}%
Your answer in \hyperref[compositionagian]{Activity~\ref{compositionagian}} can be expressed as a binomial coefficient. This means it should be possible to interpret a composition as a subset of some set. Find a bijection between compositions of \(k\) into \(n\) parts and certain subsets of some set.  Explain explicitly how to get the composition from the subset and the subset from the composition.%
\par\smallskip%
\noindent\textbf{Hint.}\hypertarget{hint-87}{}\quad%
\hypertarget{p-894}{}%
If we line up \(k\) identical books, how many adjacencies are there in between books?%
~\hfill{\tiny\hyperlink{a-129}{[hint]}\hypertarget{q-129}{}}\par\smallskip%
\noindent\textbf{Solution.}\hypertarget{solution-87}{}\quad%
\hypertarget{p-895}{}%
If we line up \(k\) identical books, there are \(k-1\) places in between two books. If we choose \(n-1\) of these places and slip dividers into those places, then we have a first clump of books, a second clump of books, and so on. The \(i\)th element of our list is the number of books in the \(i\)th clump. Clearly using books is irrelevant; we could line up any \(k\) identical objects and make the same argument. Our bijection is between compositions and \((n-1)\)-element subsets of the set of \(k-1\) spaces between our objects.%
\end{activity}
\end{document}
