\documentclass{book}

\input{../activities-preamble.tex}
\begin{document}
\setcounter{project}{51}
\addtocounter{project}{-1}
\begin{activity}[]\label{activity-44}
\hypertarget{p-431}{}%
Find \(R(4,3)\).%
\par\smallskip%
\noindent\textbf{Hint.}\hypertarget{hint-21}{}\quad%
\hypertarget{p-432}{}%
All that is left is to decide what can happen with \(K_9\).  First explain why no matter how the edges are colored red and blue, there is at least one vertex incident to an even number of red edges and an even number of blue edges.  What could this even number of red edges be?%
~\hfill{\tiny\hyperlink{a-51}{[hint]}\hypertarget{q-51}{}}\par\smallskip%
\noindent\textbf{Solution.}\hypertarget{solution-38}{}\quad%
\hypertarget{p-433}{}%
\(R(4,3)=9\). We showed above that \(R(4,3)\) is more than 8. So we must show that if we have nine people, we either have 4 mutual acquaintances or three mutual strangers. We can argue that there is at least one person (say person A) who is acquainted with an even number of people. If person A is acquainted with six or more people, then among these six people, there are either three mutual acquaintances or three mutual strangers. If there are three mutual strangers, we are done; if there are three mutual acquaintances, they, together with Person A are four mutual acquaintances. Thus we may assume Person A is acquainted with at most four people. Thus person A is a stranger to at least four people. If two of these people are strangers, then they, together with person A form three mutual strangers and we are done. Otherwise all of these people know each other and we have at least four mutual acquaintances, and so in every possible situation, we have either four mutual acquaintances or three mutual strangers.%
\end{activity}

\clearpage\end{document}
