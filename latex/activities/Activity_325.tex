\documentclass{book}

\input{../activities-preamble.tex}
\begin{document}
\setcounter{project}{325}
\addtocounter{project}{-1}
\begin{activity}[]\label{activity-318}
\leavevmode%
\begin{enumerate}[font=\bfseries,label=(\alph*),ref=\alph*]
\item\label{task-279} \hypertarget{p-1674}{}%
Prove that%
\begin{equation*}
\frac{1}{1-x} = (1 + x + x^2 +\cdots + x^9)(1 + x^{10} + \cdots + x^{90})(1 + x^{100} + \cdots + x^{900})\cdots.
\end{equation*}
What does this tell us about numbers?%
\par\smallskip%
\noindent\textbf{Solution.}\hypertarget{solution-259}{}\quad%
\hypertarget{p-1675}{}%
If we write the right hand side using quotients, we get%
\begin{equation*}
\frac{1-x^10}{1-x}\cdot \frac{1-x^{100}}{1-x^{10}}\cdot \frac{1-x^{1000}}{1-x^{100}}\cdot\cdots.
\end{equation*}
We see that all terms will cancel excpt for%
\begin{equation*}
\frac{1}{1-x}.
\end{equation*}
%
\par
\hypertarget{p-1676}{}%
Now written as a polynomial, \(\frac{1}{1-x} = 1 + x + x^2 + x^3 + \cdots\).  So the coefficient of \(x^k\) will be 1 for all \(k\).  The right hand side though gives the genereting function for the number of ways to write a number as the sum of up to 9 ones, up to 9 tens, up to 9 one hundreds, and so on.  In other words, this generating function gives the number of ways to write a number in base 10.  And since the coefficients are all 1, we see that this number of ways is 1; that is, there is a unique base 10 representation of every natural number.%
\item\label{task-280} \hypertarget{p-1677}{}%
Prove that every non negative integer \(n\) has a unique binary representation.%
\par\smallskip%
\noindent\textbf{Solution.}\hypertarget{solution-260}{}\quad%
\hypertarget{p-1678}{}%
We repeat the previous part, but this time show that%
\begin{equation*}
\frac{1}{1-x} = (1+x)(1+x^2)(1+x^4)(1+x^8)(1+x^{16})\cdots.
\end{equation*}
Writing the right hand side as quotients, you see that again every term cancels except for the left hand side.%
\item\label{task-281} \hypertarget{p-1679}{}%
Find a simple expression for%
\begin{equation*}
(1+x+x^2)(1+x^3+x^6)(1+x^9+x^{18})(1+x^{27}+x^{54})\cdots.
\end{equation*}
How can you interpret this result?%
\par\smallskip%
\noindent\textbf{Solution.}\hypertarget{solution-261}{}\quad%
\hypertarget{p-1680}{}%
Replicating the method from above, we see that this expression simplifies to \(\frac{1}{1-x}\).  We can interpret the expression as the generating function for the number of ways to write a number in base 3.  So this result shows that the base 3 representation is unique.%
\end{enumerate}
\end{activity}
\end{document}
