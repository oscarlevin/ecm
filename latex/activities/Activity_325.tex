\documentclass{book}

\input{../activities-preamble.tex}
\begin{document}
\setcounter{project}{325}
\addtocounter{project}{-1}
\begin{activity}[]\label{activity-318}
\leavevmode%
\begin{enumerate}[font=\bfseries,label=(\alph*),ref=\alph*]
\item\label{task-279} \hypertarget{p-1680}{}%
Prove that%
\begin{equation*}
\frac{1}{1-x} = (1 + x + x^2 +\cdots + x^9)(1 + x^{10} + \cdots + x^{90})(1 + x^{100} + \cdots + x^{900})\cdots.
\end{equation*}
What does this tell us about numbers?%
\item\label{task-280} \hypertarget{p-1681}{}%
Prove that every non negative integer \(n\) has a unique binary representation.%
\item\label{task-281} \hypertarget{p-1682}{}%
Find a simple expression for%
\begin{equation*}
(1+x+x^2)(1+x^3+x^6)(1+x^9+x^{18})(1+x^{27}+x^{54})\cdots.
\end{equation*}
How can you interpret this result?%
\end{enumerate}
\end{activity}
\end{document}
