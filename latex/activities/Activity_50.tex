\documentclass{book}

\input{../activities-preamble.tex}
\begin{document}
\setcounter{project}{50}
\addtocounter{project}{-1}
\begin{activity}[]\label{activity-43}
\hypertarget{p-441}{}%
Find a way to color the edges of \(K_8\) with red and blue so that there is no red \(K_4\) and no blue \(K_3\).%
\par\smallskip%
\noindent\textbf{Hint.}\hypertarget{hint-20}{}\quad%
\hypertarget{p-442}{}%
Often when there is a counter-example, there is one with a good deal of symmetry. (Caution: there is a difference between often and always!) One way to help yourself get a symmetric example, if there is one, is to put 8 vertices into a circle. Then, perhaps, you might draw green edges in some sort of regular fashion until it is impossible to draw another green edge between any two of the vertices without creating a green triangle.%
~\hfill{\tiny\hyperlink{a-50}{[hint]}\hypertarget{q-50}{}}\end{activity}
\end{document}
