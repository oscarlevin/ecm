\documentclass{book}

\input{../activities-preamble.tex}
\begin{document}
\setcounter{project}{106}
\addtocounter{project}{-1}
\begin{activity}[]\label{activity-99}
\hypertarget{p-744}{}%
\(k \binom{n}{k} = n \binom{n - 1}{k - 1}\).%
\par\smallskip%
\noindent\textbf{Hint.}\hypertarget{hint-60}{}\quad%
\hypertarget{p-745}{}%
Look back at \hyperref[act_anysizecommittee]{Activity~\ref{act_anysizecommittee}}.  This one is easier.%
~\hfill{\tiny\hyperlink{a-106}{[hint]}\hypertarget{q-106}{}}\par\smallskip%
\noindent\textbf{Solution.}\hypertarget{solution-61}{}\quad%
\hypertarget{p-746}{}%
Suppose you have a group of \(n\) people and you wish to form a subcommittee of \(k\) people with one of those \(k\) people to serve as chair. Choose the subcommittee in \(\binom{n}{k}\) ways and the chair in \(k\) ways. The product rule gives \(k \binom{n}{k}\) as the number of ways of selecting such a chaired subcommittee.%
\par
\hypertarget{p-747}{}%
Alternatively, you could first choose any one of the \(n\) people to serve as chair and then fill out the committee in \(\binom{n - 1}{k - 1}\) ways. There are \(n \binom{n - 1}{k - 1}\) ways to select a chaired subcommittee. Hence \(k \binom{n}{k} = n \binom{n - 1}{k - 1}\).%
\end{activity}

\clearpage\end{document}
