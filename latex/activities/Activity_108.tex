\documentclass{book}

\input{../activities-preamble.tex}
\begin{document}
\setcounter{project}{108}
\addtocounter{project}{-1}
\begin{activity}[]\label{activity-101}
\hypertarget{p-784}{}%
Let's count the number of subsets of \(\{a,b,c,d,e\}\) of size 3.  Of course we already know the answer should be \(\binom{5}{3}\), which we can see in Pascal's triangle is equal to 10.  What might you try if you didn't already know this?%
\begin{enumerate}[font=\bfseries,label=(\alph*),ref=\alph*]
\item\label{task-142} \hypertarget{p-785}{}%
You might first guess that the answer is \(5\cdot 4 \cdot 3\) since there are 5 choices for which element you put in your subset first, then  4 choices for the next element, and 3 choices for the last element.  Write down all 60 of the outcomes you get by counting the ``subsets'' this way.\footnote{This might seem like a lot of busy work, but your efforts will be rewarded.\label{fn-9}}%
\par\smallskip%
\noindent\textbf{Solution.}\hypertarget{solution-82}{}\quad%
\leavevmode%
\begin{table}
\centering
\begin{tabular}{llllll}
\(abc\)&\(acb\)&\(bac\)&\(bca\)&\(cab\)&\(cba\)\tabularnewline[0pt]
\(abd\)&\(adb\)&\(bad\)&\(bda\)&\(dab\)&\(dba\)\tabularnewline[0pt]
\(abe\)&\(aeb\)&\(bae\)&\(bea\)&\(eab\)&\(eba\)\tabularnewline[0pt]
\(acd\)&\(adc\)&\(cad\)&\(cda\)&\(dac\)&\(dca\)\tabularnewline[0pt]
\(ace\)&\(aec\)&\(cae\)&\(cea\)&\(eac\)&\(eca\)\tabularnewline[0pt]
\(ade\)&\(aed\)&\(dae\)&\(dea\)&\(ead\)&\(eda\)\tabularnewline[0pt]
\(bcd\)&\(bdc\)&\(cbd\)&\(cdb\)&\(dbc\)&\(dcb\)\tabularnewline[0pt]
\(bce\)&\(bec\)&\(cbe\)&\(ceb\)&\(ebc\)&\(ecb\)\tabularnewline[0pt]
\(bde\)&\(bed\)&\(dbe\)&\(deb\)&\(ebd\)&\(edb\)\tabularnewline[0pt]
\(cde\)&\(ced\)&\(dce\)&\(dec\)&\(ecd\)&\(edc\)
\end{tabular}
\caption{The \(3\)-element permutations of \(\{a,b,c,d,e\}\) organized by which \(3\)-element set they permute.\label{tab_permsof3}}
\end{table}
\item\label{task-143} \hypertarget{p-786}{}%
One of the subsets we are actually interested in is \(\{a,c,d\}\).  How many of the outcomes you listed above correspond to this set?  How many outcomes correspond to the set \(\{c,d,e\}\)?%
\item\label{task-144} \hypertarget{p-787}{}%
Explain why every subset corresponds to the same number of permutations.  Then use this to count the total number of subsets correctly.%
\end{enumerate}
\end{activity}
\end{document}
