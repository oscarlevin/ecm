\documentclass{book}

\input{../activities-preamble.tex}
\begin{document}
\setcounter{project}{282}
\addtocounter{project}{-1}
\begin{activity}[]\label{activity-275}
\hypertarget{p-1472}{}%
Explain how to compute the number of functions from a \(k\)-element set \(K\) to an \(n\)-element set \(N\) by using multinomial coefficients.%
\par\smallskip%
\noindent\textbf{Hint.}\hypertarget{hint-184}{}\quad%
\hypertarget{p-1473}{}%
The sum principle will help here.%
~\hfill{\tiny\hyperlink{a-282}{[hint]}\hypertarget{q-282}{}}\par\smallskip%
\noindent\textbf{Solution.}\hypertarget{solution-205}{}\quad%
\hypertarget{p-1474}{}%
Add the multinomial coefficients \(\binom{k}{j_1,j_2,\ldots,j_n}\) over all possible nonnegative values of the \(j_i\)'s that add to \(k\).  To see why, let \(N=\{y_1,y_2,\ldots,y_n\}\) and apply the definition of multinomial coefficients.%
\end{activity}
\end{document}
