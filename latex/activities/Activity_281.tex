\documentclass{book}

\input{../activities-preamble.tex}
\begin{document}
\setcounter{project}{281}
\addtocounter{project}{-1}
\begin{activity}[]\label{activity-274}
\hypertarget{p-1494}{}%
In how many ways may we label the elements of a \(k\)-element set with \(n\) distinct labels (numbered 1 through \(n\)) so that label \(i\) is used \(j_i\) times? (If we think of the labels as \(y_1, y_2, \ldots, y_n\), then we can rephrase this question as follows.  How many functions are there from a \(k\)-element set \(K\) to a set \(N=\{y_1,y_2,\ldots y_n\}\) so that \(y_i\) is the image of \(j_i\) elements of \(K\)?) This number is called a \terminology{multinomial coefficient}\index{multinomial coefficient}\index{coefficient!multinomial} and denoted by%
\begin{equation*}
\binom{k}{j_1,j_2,\ldots, j_n}.
\end{equation*}
%
~\hfill{\tiny\hyperlink{a-281}{[hint]}\hypertarget{q-281}{}}\par\smallskip%
\noindent\textbf{Solution.}\hypertarget{solution-220}{}\quad%
\hypertarget{p-1497}{}%
If the \(j_i\)'s don't add to \(k\), it is zero.  Otherwise \(\binom{k}{j_1,j_2,\ldots, j_n} =
\frac{k!}{j_1!j_2!\cdots j_n!}\). We get this either as the product of binomial coefficients%
\begin{equation*}
\binom{k}{j_1}\binom{k-j_1}{j_2}\binom{k-j_1-j_2}{j_3}\cdots\binom{j_n}{j_n},
\end{equation*}
or more elegantly, by lining up the elements of the domain in \(k!\) ways, taking the first \(j_1\) elements to \(y_1\), the next \(j_2\) elements to \(y_2\) and so on.  However the order of the \(j_i\) elements that go to \(y_i\) is irrelevant, so \(j_1!j_2!\cdots j_n!\) lists all correspond to the same function, giving us \(\frac{k!}{j_1!j_2!,\cdots j_n!}\) functions.%
\end{activity}
\end{document}
