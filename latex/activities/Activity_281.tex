\documentclass{book}

\input{../activities-preamble.tex}
\begin{document}
\setcounter{cpjt}{281}
\addtocounter{cpjt}{-1}
\begin{activity}\label{activity-274}
\hypertarget{p-1427}{}%
In how many ways may we label the elements of a \(k\)-element set with \(n\) distinct labels (numbered 1 through \(n\)) so that label \(i\) is used \(j_i\) times? (If we think of the labels as \(y_1, y_2, \ldots, y_n\), then we can rephrase this question as follows.  How many functions are there from a \(k\)-element set \(K\) to a set \(N=\{y_1,y_2,\ldots y_n\}\) so that \(y_i\) is the image of \(j_i\) elements of \(K\)?) This number is called a \terminology{multinomial coefficient}\index{multinomial coefficient}\index{coefficient!multinomial} and denoted by%
\begin{equation*}
\binom{k}{j_1,j_2,\ldots, j_n}.
\end{equation*}
%
\par\smallskip%
\noindent\textbf{Hint 1}.\hypertarget{hint-182}{}\quad%
\hypertarget{p-1428}{}%
What if the \(j_i\)'s don't add to \(k\)?%
\par\smallskip%
\noindent\textbf{Hint 2}.\hypertarget{hint-183}{}\quad%
\hypertarget{p-1429}{}%
Think about listing the elements of the \(k\)-element set and labeling the first \(j_1\) elements with label number 1.%
\par\smallskip%
\noindent\end{activity}

\clearpage\end{document}
