\documentclass{book}

\input{../activities-preamble.tex}
\begin{document}
\setcounter{cpjt}{85}
\addtocounter{cpjt}{-1}
\begin{activity}\label{act-latticepaths2}
\hypertarget{p-624}{}%
We have seen that the number of lattice paths from \((0,0)\) to \((x,y)\) is given by \(\binom{x+y}{x}\).  Now with the sum and product principles, we can answer lots of questions about lattice paths.%
\begin{enumerate}[font=\bfseries,label=(\alph*),ref=\alph*]
\item\label{task-127} \hypertarget{p-625}{}%
How many lattice paths from \((0,0)\) to \((10,10)\) pass through the point \((4,7)\)?%
\par\smallskip%
\noindent\textbf{Hint}.\hypertarget{hint-37}{}\quad%
\hypertarget{p-626}{}%
Many students want this sort of question to be an example of the sum principle (presumably because to get an acceptable path you travel to \((4,7)\) and then add on a path that gets you the rest of the way.  But notice that you are really concatenating two paths, and there are lots of choices for the first path and lots for the second.  This sounds more like an ordered pair.%
\item\label{task-128} \hypertarget{p-627}{}%
How many lattice paths from \((0,0)\) to \((10,10)\) do NOT pass through the point \((4,7)\)?  In what way does this problem use the sum principle?%
\par\smallskip%
\noindent\textbf{Hint}.\hypertarget{hint-38}{}\quad%
\hypertarget{p-628}{}%
If you knew the answer to this part and the previous part, what is another way you could compute the number of paths from \((0,0)\) to \((10,10)\)?%
\item\label{task-129} \hypertarget{p-629}{}%
How many lattice paths from \((0,0)\) to \((10,10)\) pass through \((4,7)\) or \((7,4)\)?%
\item\label{task-130} \hypertarget{p-630}{}%
Explain why can you NOT use the sum principle to answer this question: How many lattice paths from \((0,0)\) to \((10,10)\) pass through \((4,7)\) or \((7,8)\)?  What is the answer?%
\par\smallskip%
\noindent\textbf{Hint}.\hypertarget{hint-39}{}\quad%
\hypertarget{p-631}{}%
Compare to the question about how many playing cards are either red or a face card, and why the answer is not \(26 + 12\).%
\end{enumerate}
\end{activity}

\clearpage\end{document}
