\documentclass{book}

\input{../activities-preamble.tex}
\begin{document}
\setcounter{project}{303}
\addtocounter{project}{-1}
\begin{activity}[]\label{activity-296}
\hypertarget{p-1527}{}%
Draw the Young diagram of the partition (4,4,3,1,1). Describe the geometric relationship between the Young diagram of (5,3,3,2) and the Young diagram of (4,4,3,1,1).%
\par\smallskip%
\noindent\textbf{Hint.}\hypertarget{hint-189}{}\quad%
\hypertarget{p-1528}{}%
Draw a line through the top-left corner and bottom-right corner of the top-left box.%
~\hfill{\tiny\hyperlink{a-303}{[hint]}\hypertarget{q-303}{}}\par\smallskip%
\noindent\textbf{Solution.}\hypertarget{solution-201}{}\quad%
\begin{sidebyside}{1}{0.42}{0.42}{0}
\begin{sbspanel}{0.16}
\includegraphics[width=1\linewidth]{images/Young44311}
\end{sbspanel}
\end{sidebyside}
\par
\hypertarget{p-1529}{}%
We get the Young diagram of \((5,3,3,2)\) by flipping the Young diagram of \((4,4,3,1,1)\) around a line that includes the diagonal of the upper left box; if we think of the top left corner of the diagram as being at the origin, we flip around the line \(y=-x\).%
\end{activity}
\end{document}
