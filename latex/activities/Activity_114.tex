\documentclass{book}

\input{../activities-preamble.tex}
\begin{document}
\setcounter{project}{114}
\addtocounter{project}{-1}
\begin{activity}[]\label{activity-107}
\hypertarget{p-823}{}%
We have used the quotient principle to explain the formula \(\binom{n}{k} = \frac{n!}{(n-k)!k!}\) by thinking of this as \(\binom{n}{k} = \frac{P(n,k)}{k!}\).  What if we don't involve \(P(n,k)\) at all?%
\par
\hypertarget{p-824}{}%
Describe a set of outcomes that has size \(n!\) that can be partitioned into blocks of size \((n-k)!k!\) so that each block corresponds to something \(\binom{n}{k}\) counts.%
~\hfill{\tiny\hyperlink{a-114}{[hint]}\hypertarget{q-114}{}}\end{activity}
\end{document}
