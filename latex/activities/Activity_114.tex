\documentclass{book}

\input{../activities-preamble.tex}
\begin{document}
\setcounter{project}{114}
\addtocounter{project}{-1}
\begin{activity}[]\label{activity-107}
\hypertarget{p-798}{}%
We have used the quotient principle to explain the formula \(\binom{n}{k} = \frac{n!}{(n-k)!k!}\) by thinking of this as \(\binom{n}{k} = \frac{P(n,k)}{k!}\).  What if we don't involve \(P(n,k)\) at all?%
\par
\hypertarget{p-799}{}%
Describe a set of outcomes that has size \(n!\) that can be partitioned into blocks of size \((n-k)!k!\) so that each block corresponds to something \(\binom{n}{k}\) counts.%
\par\smallskip%
\noindent\textbf{Hint 1.}\hypertarget{hint-70}{}\quad%
\hypertarget{p-800}{}%
You might try the next problem first to get an idea.%
~\hfill{\tiny\hyperlink{a-114}{[hint]}\hypertarget{q-114}{}}\par\smallskip%
\noindent\textbf{Hint 2.}\hypertarget{hint-71}{}\quad%
\hypertarget{p-801}{}%
One thing that \(\binom{n}{k}\) counts is all bit strings of length \(n\) and weight \(k\).  What if instead of bit strings, we wanted all strings made up out of some number of distinct symbols that come in two types?  How can you make this be counted by \(n!\)?%
\end{activity}

\clearpage\end{document}
