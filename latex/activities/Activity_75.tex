\documentclass{book}

\input{../activities-preamble.tex}
\begin{document}
\setcounter{cpjt}{75}
\addtocounter{cpjt}{-1}
\begin{activity}\label{act-pascalsym-bij}
\hypertarget{p-564}{}%
Consider the set \([n] = \{1,2,\ldots,n\}\) and its power set \(\pow([n])\) of all subsets of \([n]\).  Define the function \(f:\pow([n]) \to \pow([n])\) by \(f(A) = [n]\setminus A\) for any \(A \in \pow([n])\) (that is, \(f(A)\) is the complement of \(A\) in \([n]\)).%
\begin{enumerate}[font=\bfseries,label=(\alph*),ref=\alph*]
\item\label{task-105} \hypertarget{p-565}{}%
Prove that \(f\) is a bijection.  In fact, \(f\) is an \terminology{involution} in that it is its own inverse.%
\item\label{task-106} \hypertarget{p-566}{}%
For any set \(A \in \pow([n])\), if \(\card{A} = k\), what is the \(\card{f(A)}\)?%
\item\label{task-107} \hypertarget{p-567}{}%
Is \(f\) still a bijection if we restrict it to just the subsets of \([n]\) that have size \(k\)?  Explain.%
\item\label{task-108} \hypertarget{p-568}{}%
Why is the above enough to establish the identity?%
\par\smallskip%
\noindent\textbf{Hint}.\hypertarget{hint-31}{}\quad%
\hypertarget{p-569}{}%
Recall the bijection principle tells us that if \(f:X \to Y\) is a bijection, then \(\card{X} = \card{Y}\).  What is the cardinality of each of the subsets of \(\pow([n])\) that we are considering?%
\end{enumerate}
\end{activity}

\clearpage\end{document}
