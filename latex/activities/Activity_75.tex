\documentclass{book}

\input{../activities-preamble.tex}
\begin{document}
\setcounter{project}{75}
\addtocounter{project}{-1}
\begin{activity}[]\label{act-pascalsym-bij}
\hypertarget{p-598}{}%
Consider the set \([n] = \{1,2,\ldots,n\}\) and its power set \(\pow([n])\) of all subsets of \([n]\).  Define the function \(f:\pow([n]) \to \pow([n])\) by \(f(A) = [n]\setminus A\) for any \(A \in \pow([n])\) (that is, \(f(A)\) is the complement of \(A\) in \([n]\)).%
\begin{enumerate}[font=\bfseries,label=(\alph*),ref=\alph*]
\item\label{task-108} \hypertarget{p-599}{}%
Prove that \(f\) is a bijection.  In fact, \(f\) is an \terminology{involution} in that it is its own inverse.%
\item\label{task-109} \hypertarget{p-600}{}%
For any set \(A \in \pow([n])\), if \(\card{A} = k\), what is the \(\card{f(A)}\)?%
\item\label{task-110} \hypertarget{p-601}{}%
Is \(f\) still a bijection if we restrict it to just the subsets of \([n]\) that have size \(k\)?  Perhaps we should also restrict our codomain (to what?).  Explain.%
\item\label{task-111} \hypertarget{p-602}{}%
Why is the above enough to establish the identity?%
~\hfill{\tiny\hyperlink{a-75.d}{[hint]}\hypertarget{q-75.d}{}}\end{enumerate}
\end{activity}
\end{document}
