\documentclass{book}

\input{../activities-preamble.tex}
\begin{document}
\setcounter{project}{209}
\addtocounter{project}{-1}
\begin{activity}[]\label{activity-202}
\hypertarget{p-1167}{}%
How many solutions are there in the positive integers to the equation \(x_1+x_2+x_3 =7\) with \(x_1\ge x_2\ge x_3\)?%
\par\smallskip%
\noindent\textbf{Solution.}\hypertarget{solution-120}{}\quad%
\hypertarget{p-1168}{}%
This problem is asking for \(p_3(7)\) and suggests an organized way to go about finding it: list the partitions starting with the largest part and work down. \(7=5+1+1\), \(7=4+2+1\), \(7=3+3+1\), \(7=3+2+2\), and if we have three numbers that add to seven, one must be larger than two, so there are four such solutions.%
\end{activity}
\end{document}
